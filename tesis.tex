\documentclass[12pt, a4paper,twoside]{tesi_upf}


%CODIFICATION
\usepackage[latin1]{inputenc}
\usepackage[T1]{fontenc}    % use 8-bit T1 fonts
\usepackage{url} 

%IMPORTED PACKAGES BY GUILLERMO
\usepackage{booktabs}       % professional-quality tables
\usepackage{amsfonts}       % blackboard math symbols
\usepackage{nicefrac}       % compact symbols for 1/2, etc.
\usepackage{microtype}      % microtypography
\usepackage{xcolor}         % colors
\usepackage{mathtools}
\usepackage[noend]{algpseudocode}
\usepackage{amsthm}
\usepackage{adjustbox}
\usepackage{todonotes}
\usepackage{amssymb}
\usepackage{comment}
\usepackage{caption}
\usepackage{subcaption}
\usepackage{xspace}
\usepackage{booktabs}
\usepackage{tabularx}
\usepackage{multirow}
\usepackage{amsmath}
\usepackage{natbib}
\usepackage{tikz}
\usepackage{algorithm}
\usepackage[noend]{algpseudocode}
% \usepackage{mathspec} 
\usepackage{adjustbox}
\usepackage{dsfont}
\usepackage{thmtools}
\usepackage{thm-restate}
\usepackage{setspace}
\usepackage{comment}
\usepackage[toc,page]{appendix}

\usepackage[hyperfootnotes=false]{hyperref}
\hypersetup{
  citecolor=blue,}


\usetikzlibrary{arrows,automata,positioning}
\usetikzlibrary{shapes.multipart}
\usetikzlibrary{decorations.markings}
\usetikzlibrary{decorations.pathreplacing}

%LENGUAGE
\usepackage[english,catalan]{babel}

%ONLY TO OBTAIN MARK BANK INDEX INDICATION A4
\usepackage[cam,a4,center,frame]{crop}

%INCLUDE GRAPHICS AND THE LOGO OF THE UPF
\usepackage{graphicx}

%FONTS TIMES OR GARAMOND, 
%\usepackage{times}
\usepackage{ebgaramond}

%WITHOUT HEADINGS: NO MODIFICATION
\pagestyle{plain}

%FOR THE INDEX OF SUBJECTS
\usepackage{makeidx}
\makeindex


\usepackage{enumitem}
\setlist[itemize]{topsep=-0.5pt}

\usepackage{rotating}

%SELECT LANGUAGE
\selectlanguage{english}

%THE TABLE OF CONTENTS IS TITLE CONTENTS
\addto\captionscatalan
  {\renewcommand{\contentsname}{\Large \sffamily Sumari}}


%COMMANDS CREATED BY GUILLERMO

\DeclareMathOperator*{\argmax}{arg\,max}

% macros:

\newcommand{\guillermo}[1]{{\color{blue}#1}}

\newcommand{\tS}{\text{t}}   % for terminal states
\newcommand{\tSi}{\tau}   % for terminal states in the subtasks
\newcommand{\cA}{\mathcal{A}}
\newcommand{\cB}{\mathcal{B}}
\newcommand{\cC}{\mathcal{C}}
\newcommand{\cD}{\mathcal{D}}
\newcommand{\cE}{\mathcal{E}}
\newcommand{\cF}{\mathcal{F}}
\newcommand{\cG}{\mathcal{G}}
\newcommand{\cH}{\mathcal{H}}
\newcommand{\cI}{\mathcal{I}}
\newcommand{\cJ}{\mathcal{J}}
\newcommand{\cK}{\mathcal{K}}
\newcommand{\cL}{\mathcal{L}}
\newcommand{\cM}{\mathcal{M}}
\newcommand{\cO}{\mathcal{O}}
\newcommand{\cP}{\mathcal{P}}
\newcommand{\cR}{\mathcal{R}}
\newcommand{\cS}{\mathcal{S}}
\newcommand{\cT}{\mathcal{T}}
\newcommand{\cU}{\mathcal{U}}
\newcommand{\cV}{\mathcal{V}}
\newcommand{\cX}{\mathcal{X}}

\newcommand{\kernel}{\mathbb{P}}

\newcommand{\indicator}[1]{\mathds{I}\{#1\}}



\newcommand{\EE}[1]{\mathbb{E}\left[#1\right]}
\newcommand{\EEc}[2]{\mathbb{E}\left[#1\;\middle\lvert\;#2\right]}
\newcommand{\EEcp}[3]{\mathbb{E}_{#3}\left[#1\;\middle\lvert\;#2\right]}
\newcommand{\KL}[2]{\mathrm{KL}\left(#1 \lVert #2\right)}
\newcommand{\pa}[1]{\left(#1\right)}

\newcommand{\norm}[1]{\left\|#1\right\|}
\newcommand{\onenorm}[1]{\norm{#1}_1}
\newcommand{\infnorm}[1]{\norm{#1}_\infty}

\newcommand{\diag}{\text{diag}}
\newcommand{\real}{\mathbb{R}}

\newcommand{\w}{\mathbf{w}}

\definecolor{mygreen}{HTML}{569e34}
\newcommand{\checklist}[1]{{\color{mygreen}#1}}

\newcommand{\boldpsi}{\boldsymbol{\mathbf{\psi}}}

\DeclareMathOperator*{\argmin}{argmin}

\newtheorem{theorem}{Theorem}
\newtheorem{lemma}[theorem]{Lemma}
\newtheorem{corollary}[theorem]{Corollary}
\newtheorem{proposition}[theorem]{Proposition}
\newtheorem{fact}[theorem]{Fact}
\newtheorem{definition}{Definition}
\newtheorem{assumption}[theorem]{Assumption}

% Define Example environment
\definecolor{myblue}{HTML}{060963}

\newcounter{example}[section]
\newenvironment{example}[1][]{%
  \color{myblue}
  \refstepcounter{example}
  \par\medskip\noindent
  %~\theHchapter.\theexample}
  \if
  \relax\detokenize{#1}\relax
  \else
  {\bfseries#1.}
  \fi
  \enspace
  \ignorespaces
}{\par\medskip}

\usepackage{tikz}
\usetikzlibrary{shapes.geometric}
\usetikzlibrary{positioning,automata,arrows}
\usepackage{pifont}
\usepackage{fontawesome,wasysym,marvosym}

\newcommand{\coffee}[0]{{\color{black}\Coffeecup}\xspace}
\newcommand{\taxi}[0]{{\color{purple}\faTaxi}\xspace}
\newcommand{\person}[0]{{\color{black}\Gentsroom}\xspace}

\newcommand{\mail}[0]{\Letter}

\newcommand{\agent}{\resizebox{4mm}{!}{\begin{tikzpicture}\node[draw, thick, shape border rotate=90, isosceles triangle, isosceles triangle apex angle=60, fill=violet!70!white, fill opacity=1.0, node distance=1cm,minimum height=1.5em] at (0,0) {};\end{tikzpicture}}\xspace}

\newcommand{\miniagent}{\resizebox{2mm}{!}{\begin{tikzpicture}\node[draw, thick, shape border rotate=90, isosceles triangle, isosceles triangle apex angle=60, fill=violet!70!white, fill opacity=1.0, node distance=1cm,minimum height=1.5em] at (0,0) {};\end{tikzpicture}}\xspace}


% Allow page breaks in aligns
\allowdisplaybreaks

%ADD YOUR DATA
\title{Compositionality for Hierarchical Reinforcement Learning}
\subtitle{}
\author{Guillermo Infante Molina}
\thyear{2024}
\department{de Tecnologies de la Informaci\'o i les Comunicacions}
\supervisor{Anders Jonsson, Vicen\c{c} G\'omez}


\begin{document}


\frontmatter

\maketitle

\cleardoublepage


%%%%%% Dedication

\rightline{\large A mis padres, Amparo y Agust\'in, y mi hermano Jorge.}

\cleardoublepage

%%%%%% End dedication


%%%%%% Thanks
\input{chapters/ack}
\cleardoublepage

%%%%%% End of thanks

%ABSTRACT IN TWO LEGUAGES.
\selectlanguage{english}
\chapter*{Abstract}

Recent breakthroughs in machine intelligence, such as AlphaZero or ChatGPT, have proven that Reinforcement Learning can be used successfully to solve complex sequential decision problems with super-human performance level. However, despite the successful applications that use function approximation techniques, one key challenge for RL algorithms is data efficiency. In this line, hierarchical methods have been historically utilised for simplyfying the solutions. The philosiphycal motivation in favor of hierarchical systems is several-fold. First, human tend to think and reason at different levels of abstractions. Second, solving several smaller problem could be easier and more straightforward than solving the original problem. And third, decomposing a problem into subtasks or skills allows agents to exploit composability properties of different methods.

We present three different frameworks in which agent can exploit compositionality. Instead of applu

\cleardoublepage
%END OF ABSTRACT

%PREFACE. 
% {\bf Preface}

\cleardoublepage
%END OF PREFACE


%TABLE OF CONTENTS: REQUIRED
\tableofcontents

%lIST OF FIGURES; ONLY IF THERE ARE FIGURES
\listoffigures
%TO APPER THE LIST OF FIGURES IN THE TABLE OF CONTENTS 
\addcontentsline{toc}{chapter}{List of figures}

%LIST OF TABLES; ONLY IF THERE ARE TABLES
\listoftables
%TO APPEAR THE LIST OF TABLES IN THE TABLE OF CONTENTS
\addcontentsline{toc}{chapter}{List of tables}

% LENGTH OF SPACES BETWEEN PARAGRAPHS
\setlength{\parskip}{0.5\baselineskip}
\setlength{\parindent}{0pt}
\raggedbottom
%START THE TEXT
\mainmatter

\chapter{Introduction}
\chapter{Introduction}

\section{Thesis structure}
\section{Contributions}

\chapter{Background}
In this chapter we introduce and clarify the technical concepts to necessary understand the later chapters. We also introduce the following notation that is used throughout the document:
\begin{itemize}
  \item Given a finite set $\cX$, let $\Delta(\cX)=\{p\in\real^\cX:\sum_x p(x)=1, p(x)\geq 0\;(\forall x)\}$ we denote the probability simplex on $\cX$. 
  \item Given a probability distribution $p\in\Delta(\cX)$, we let $\cB(p)=\{x\in\cX:p(x) > 0\}\subseteq\cX$ denote the support of $p$.
  \item Given a set $\cX$, let$\lvert\cX\rvert$ we denote the cardinality of set $\cX$.
  \item We use capital, non-caligraphic letters to express random variables while their lowercase counterpart represents a realization of such a random variable. E.g: $S_t$ is a random variable that denotes a state at timestep $t$ and $s_0$ is the realization of such random variable at timestep $0$.
\end{itemize}
\section{Markov decision processes}
Stochastic sequential decision problems in which a learning agent interacts sequentially with an environment are usually modeled as Markov decision processes (MDP). Without loss of generality, we restrict our attention to countable MDPs which are formally defined as the tuple $\cM=\langle\cS,\cA,\cR,\kernel,\kernel_0\rangle$, where:
\begin{itemize}
  \item $\cS$ is a countable set of states.
  \item $\cA$ is a finite set of actions.
  \item $\cR:\cS\times\cA\times\cS\rightarrow\real$ is a reward function. Though there exists multiple ways of defining a reward function (such as $\cR:\cS\times\cA\rightarrow\real$ or even $\linebreak\cR:\cS\rightarrow\real$), we stick to its most general form that associates rewards to triplets $\linebreak(s, a, s')\in\cS\times\cA\times\cS$. If necessary, we will redefine the reward functions accordingly.
  \item $\kernel:\cS\times\cA\rightarrow\Delta(\cS)$ is a transition probablity distribution that encondes $\kernel(\cdot\lvert s, a)$ for every $(s, a)\in\cS\times\cA$.
  \item $\kernel_0:\Delta(\cS)$ is the initial state distribution. We assume that $\kernel_0(s)>0\;\forall s\in\cS$.
\end{itemize}

The most general form of the interaction is depicted in Figure~\ref{fig:rl_loop}. At timestep $t$, the agent observes a state $S_t$ and chooses an action $A_t\sim\pi(S_t)$ according to a decision rule $\pi$. Such a decision rule is called the policy and it is a function 
\begin{equation}
\pi:\cS\rightarrow\Delta(\cA).
\label{eq:def_policy}
\end{equation}
This means the policy is a function that maps states to a distribution over actions. Once the agent executes action $A_t$ in the environment, it receives a reward $\linebreak R_{t+1} = \cR(S_t, A_t, S_{t+1})$ and a new state $S_{t+1}\sim\kernel(\cdot\lvert\cS_t, A_t)$. The process repeats indefinitely. Initially, the start state is sampled from the initial state distribution $S_0\sim\kernel_0$.

\begin{figure}
  \centering
  \includegraphics[width=0.65\textwidth]{figures/RL_loop.png}
  \caption{Interaction loop in a Markov decision process.}
  \label{fig:rl_loop}
\end{figure}


The quadruplet $(s_t, a_t, r_{t+1}, s_{t+1})$ defines a transition while the sequence of transitions up to a certain timestep, \[
  h_t = (s_0, s_0, r_1, s_1,\dots,r_t,s_t),
\] is called history. We let $\cH$ denote the set of all possible histories. Note that the interaction of an agent with the environment can be described by means of trajectories $h\in\cH$.

These processes are so-called Markov because the Markov property holds both in the reward and transition probability functions. This indicates that at timestep $t$ the reward and next state depends solely on the current state (or state and action), and not the full history. Formally this means that
\begin{align*}
  \kernel(s_{t+1}\lvert h_t) &= \kernel(s_{t+1}\lvert s_t, a_t), \\
  \cR(s_t, a_t, s_{t+1}\lvert h_t) &=  \cR(s_t, a_t, s_{t+1}) 
\end{align*}

There are many ways to design policies, but our definition in~\ref{eq:def_policy} implies that we will consider policies that:
\begin{itemize}
  \item are Markovian in the sense that the choice of the action depends on the current state $s_t\in\cS$ and not the whole trajectory $h_t\in\cH$, 
  \item are stationary, because they remain the same over time. To be more precise, the choice of the action given a state does not the depend on the timestep $t$ in which the state is observed,
  \item are stochastic as they represent a distribution over actions. More concretely a distribution $\pi(\cdot\lvert s)$ consitioned on the state $s\in\cS$. 
\end{itemize}
Unless otherwise specified, the policies considered satisfy the characteristics just mentioned. We will also need the notion of deterministic policies. In that case policies are functions $\pi:\cS\rightarrow\cA$ that return a single action for a given state, i.e~$a=\pi(s)$.


The aim of the learning agent is to come up with policies such that they optimize some numerical objective. Now, we turn our attention to two optimality cases: the discounted and the average-reward frameworks. In the following subsections we describe how we can solve each case when $\kernel$ and $\cR$ are fully disclosed to the agent via dynamic-programming. This setting, in which the reward and transition functions are known, is also referred to as planning in MDPs.


\subsection{The discounted case}
In the discounted setting, an optimal policy is such that it maximizes the expected discounted return. The discounted return is given by the sum of discounted rewards, 
\begin{equation}
  G_t = \sum_{i=t} ^\infty \gamma ^ {i-t} \cR(S_i, A_i, S_{i+1}).
  \label{eq:return}
\end{equation}
Here, $0<\gamma<1$ is the discount factor. There exists several reasons to use the discount factor, such as giving more credit to rewards closer in time. Notwithstanding, it is mainly used so as to make the return (Equation~\ref{eq:return}) have a finite value.

Given a policy $\pi$, we define its value function, $v^\pi:\cS\rightarrow\real$, as the expected discounted return of being at state $s\in\cS$ and following policy $\pi$,
\begin{equation}
  v^\pi(s) = \EEcp{\sum_{i=t} ^\infty \gamma ^ {i-t} R_i}{S_t = s}{\pi}\;\forall s\in\cS, 
\end{equation}
where $R_i$ is a shorthand for $\cR(S_i, A_i, S_{i+1})$. We also define the action-value function, $q^\pi:\cS\times\cA\rightarrow\real$, of a state-action pair as the expected discounted return of being at state $s\in\cS$, choosing action $a\in\cA$ and following policy $\pi$ thereafter,
\begin{equation}
  q^\pi(s, a) = \EEcp{\sum_{i=t} ^\infty \gamma ^ {i-t} R_i}{S_t = s, A_t = a}{\pi}\;\forall (s, a)\in\cS\times\cA.
\end{equation}
This value function is known to satisfy the following Bellman equations
\begin{equation}
  v^\pi(s) = \sum_a \pi(a\lvert s)\bigg[\sum_{s'} \cR(s, a, s') + \gamma \kernel(s'\lvert s, a)v^\pi(s')\bigg]\;\forall s\in\cS.
\end{equation}
Analogously, the action-value function satisfies the Bellman recursion given by 
\begin{equation}
  q^\pi(s, a) = \sum_{s'} \cR(s, a, s') + \gamma \kernel(s'\lvert s, a)v^\pi(s')\;\forall (s, a)\in\cS\times\cA.
\end{equation}
The goal of the agent is to compute a policy $\pi^*$ that maximizes the expected discounted return. We denote the optimal value and action-value functions attained by an optimal policy as $v^*$ and $q^*$, respectively. The question is how to obtain such optimal value functions so we can derive the optimal policy.

First, we describe how we can obtain the value function associated with a policy $\pi$. This procedure is usually known as policy evaluation. We declare the following Bellman operator $T^\pi:\real^\cS\rightarrow\real^\cS$ as 
\begin{equation}
  (T^\pi v^\pi)(s) = \sum_a \pi(a\lvert s)\bigg[\sum_{s'} \cR(s, a, s') + \gamma \kernel(s'\lvert s, a)v^\pi(s')\bigg]\;\forall s\in\cS.
  \label{eq:bo}
\end{equation}
By looking at $v^\pi$ as a vector of the appropiate size, Equation~\ref{eq:bo} can be rewritten in vector form as 
\begin{equation}
  T^\pi v^\pi = v^\pi.
\end{equation}
Therefore, computing the true value of $v^\pi$ requires finding the fixed-point solution of the previous system of linear equations. 

We further introduce the Bellman optimality operator $T^*:\real^\cS\rightarrow \real^\cS$ defined as
\begin{equation}
  (T^* v^*)(s) = \max_a \bigg[\sum_{s'} \cR(s, a, s') + \gamma \kernel(s'\lvert s, a)v^*(s')\bigg]\;\forall s\in\cS.
  \label{eq:boo}
\end{equation}
We can again rewrite~\ref{eq:boo} in vector form:
\begin{equation}
  T ^* v^* = v^*.
  \label{eq:boo_vector}
\end{equation}
The optimal value function $v^*$ is the fixed-point solution of the previous system of equations, which in this case is not linear as it requires a $\max$ operation. We can derive tge first dynamic-programming algorithm called value iteration from Equation~\ref{eq:boo_vector}. Here, we keep an estimate $v_k$ of the optimal value function function which updated iteratively as \[v_{k+1}\leftarrow T^* v_k,\] for some arbitrary initialization $v_0$. Value iteration can be shown to converge thanks to the contraction mapping theorem (see Appendix).

The idea of the operators can be also applied to the action-value function. Thus, we define the Bellman operator $T^\pi:\real^{\cS\times\cA}\rightarrow\real^{\cS\times\cA}$ and the Belmman optimality operator $T^*:\real^{\cS\times\cA}\rightarrow\real^{\cS\times\cA}$ over action-value functions in the following way:
\begin{align}
(T^\pi q^\pi)(s, a) &= \sum_{s'} \cR(s, a, s') + \gamma \kernel(s'\lvert s, a)v^\pi(s')\;\forall (s, a)\in\cS\times\cA. \\
(T^* q^*)(s, a) &= \sum_{s'} \cR(s, a, s') + \gamma \kernel(s'\lvert s, a)v^*(s')\;\forall (s, a)\in\cS\times\cA.
\end{align}
We also introduce the greedy operator $\cG:\real^{\cS}\rightarrow\cA^\cS$ over action-value functions which is defined as 
\begin{equation}
  (\cG q^\pi)(s, \cdot) = \argmax_a q^\pi(s, a) 
\end{equation}
and returns a deterministic, greedy policy with respect to the action-value function.
Now, we can describe the procedure called policy iteration, which consists of interleaved steps of policy evaluation and policy improvement operations. This algorithm works as follows
 \begin{enumerate}[label=(\arabic*)]
  \item Fix some initial policy $\pi_0$.
  \item At each iteration solve $T^{\pi_k} q^{\pi_k} = q^{\pi_k}$ (policy evaluation).
  \item Then derive new policy $\pi_{k+1}\leftarrow\cG q^{\pi_k}$ (policy improvement).
  \item Repeat (2) and (3) until convergence.
\end{enumerate}



\subsection{The average-reward case}
A better way to model continuing tasks is the average-reward setting. Here, the agent seeks a policy that maximizes 
\begin{equation}
  \rho^\pi(s) =\lim_{T\rightarrow\infty}\frac 1 T \mathbb{E}_\pi\left[\sum_{i=t}^T\cR(S_i, A_i, S_{i+1})\right] \;\forall s\in\cS
\end{equation}
which is known as the average-reward per step or gain.

This setting is arguably more complex than the discounted one and, usually, the following assumptions are made.

\begin{assumption}
  The MDP $\cM$ is communicating~\citep{Puterman1994}: for each pair of states $s,s'\in\cS$, there exists a policy $\pi$ that has non-zero probability of reaching $s'$ from $s$.
  \label{ass:mdp_communicating}
\end{assumption}

\begin{assumption}
  The ALMDP $\cM$ is unichain~\citep{Puterman1994}: the transition probability distribution induced by all stationary policies admit a single recurrent class.
  \label{ass:mdp_unichain}
\end{assumption}

In other words, we assume that$\dots$. A consquence of the aforeintroduced assumptions that simplify the analysis and algorithms is that the gain of any policy does not depend on the state, thus, $\rho^\pi(s) = \rho^\pi(s') = \rho^\pi = $. We use the latter term to denote the gain of a policy $\pi$.

Unlike the discounted case, the value functions could be now unbounded and, instead, relative value functions are used. are known to satisfy
\begin{align}
  v^\pi(s) &= \sum_a \pi(a\lvert s)\bigg[\sum_{s'} \cR(s, a, s') - \rho^\pi + \gamma \kernel(s'\lvert s, a)v^\pi(s')\bigg]\;\forall s\in\cS. \\
  q^\pi(s, a) &= \sum_{s'} \cR(s, a, s') - \rho^\pi + \gamma \kernel(s'\lvert s, a)v^\pi(s');\forall (s,a)\in\cS\times\cA. 
\end{align}

\section{Reinforcement learning}
When the agent is alien to the reward funtion $\cR$ and the transition function $\kernel$ of the MDP, the learning happens through direct interaction with the environment. Reinforcement learning (RL) proposes a learning paradigm in which the agent tries to learn the optimal behaviour by means of samples of that interaction. There is a vast collection of RL algorithms


\subsection{The discounted setting}
\subsection{The average-reward setting}

% \begin{figure}[t!h]
%     \centering
%     \includegraphics[width=0.65\textwidth]{figures/RL_loop.png}
%     \caption{Caption}
%     \label{fig:rl_loop}
% \end{figure}


\guillermo{
\section{Successor features} DEVELOP THIS FURTHER .. 

\label{section:successor_features}

Successor features (SFs)~\citep{Dayan1993, Barreto2017} is a widely used RL representation framework that assumes the reward function is linearly expressible with respect to a feature vector,
\begin{equation}
  \cR^\w(s, a, s') = \w^\intercal\boldsymbol\phi(s, a , s').
  \label{eq:reward_sf}
\end{equation}

Here, $\boldsymbol\phi:\cS\times\cA\times\cS\rightarrow\real^{d}$ maps transitions to feature vectors and $\w\in\real^d$ is a weight vector. Every weight vector~$\w$ induces a different reward function and, thus, a task. The SF vector of a state-action pair $(s,a)\in\cS\times\cA$ under a policy $\pi$ is the expected discounted sum of future feature vectors: 
\begin{equation}
  \boldpsi^\pi(s, a) = \EEcp{\sum_{i=t}^\infty \gamma^{i-t} \boldsymbol\phi_i}{S_t = s, A_t = a}{\pi},
  \label{eq:sf}
\end{equation}
where $\boldsymbol\phi_i = \boldsymbol\phi(S_{i}, A_{i}, S_{i+1})$. The action value function for a state-action pair $(s, a)$ under policy $\pi$ can be efficiently represented using the SF vector. Due to the linearity of the reward function, the weight vector can be decoupled from the Bellman recursion. Following the definition of Equations~\eqref{eq:qfunction}~ and \eqref{eq:reward_sf}, the action value function in the SF framework can be rewritten as
\begin{align}
  Q^\pi_\w(s, a) &= \EEcp{\sum_{i=t} ^\infty \gamma^{i-t} \w^\intercal\boldsymbol\phi_i}{S_t = s, A_t = a}{\pi} \nonumber \\
                 & = \w^\intercal\EEcp{\sum_{i=t} ^\infty \gamma^{i-t} \boldsymbol\phi_i}{S_t = s, A_t = a}{\pi} \nonumber \\
                 &=  \w^\intercal \boldpsi^\pi(s, a) .
\label{eq:qfunction_sf}
\end{align}

The SF representation leads to \textit{generalized policy evaluation} (GPE) over multiple tasks~\citep{Barreto2020a}, and similarly, to \textit{generalized policy improvement} (GPI) to obtain new better policies~\citep{Barreto2017}.

A family of MDPs is defined as the set of MDPs that share all the components, except the reward function. This set is formally defined as 
\begin{equation*}
    \cM^{\boldsymbol{\phi}}\equiv\{\langle\cS,\cE,\cA,\cR_\w,\mathbb{P}_0, \mathbb{P},\gamma\rangle \lvert \cR_\w = \w^\intercal \boldsymbol{\phi}, \forall\w\in\real^d\}.
\end{equation*}

Transfer learning on families of MDPs is possible thanks to GPI. Given a set of policies $\Pi$, learned on the same family~$\cM^{\boldsymbol{\phi}}$, for which their respective SF representations have been computed, and a new task $\w'\in\real^d$, a GPI policy $\pi_{\text{GPI}}$ for any $s\in\cS$ is derived as 
\begin{equation}
    \pi_{\text{GPI}}(s) \in \argmax_{a\in\cA} \max_{\pi\in\Pi} Q^\pi_{\w'}(s, a).
    \label{eq:gpi}
\end{equation}

However, there is no guarantee of optimality for $\w'$.
A fundamental question to solve the so-called \textit{optimal policy transfer learning problem} is which policies should be included in the set of policies $\Pi$ so an optimal policy for any weight vector $\w\in\real^d$ can be obtained with GPI. 
}

\section{Linearly-solvable Markov decision processes}

Linearly-solvable Markov decision processes~\citep{Todorov2006, Kappen2005} are a restricted class of the more general MDPs where the Bellman optimality equations are linear. This makes the computation of optimal value functions more efficient. In continuous state space domains or contexts of optimal control as probablistic inference, they frequently appear under the names of path-integral or Kullback-Leibler control~\citep{Kappen2012}. 

Even though this formulation is arguably restricted and only applicable to domains with deterministic dynamics, the intuition of entropy-regularization~\citep{Neu2017}, that lies at the core of LMDPs, is fundamental in RL as it is one of the building blocks of current state-of-the-art deep reinforcement learning algorithms such as trust region policy optimization~\citep{Schulman2015}, soft actor-critic (SAC)~\citep{Haarnoja2018} or Manchausen RL~\citep{Vieillard2020}. In what follows we introduce linearly-solvable Markov decision processes in the finite-horizon (first-exit) setting, its extension to infinite-horizon (average-reward) setting and we briefly discuss how they can be used along with function approximation.

\subsection{First-exit linearly-solvable Markov decision processes}

We define a first-exit linearly-solvable Markov decision process, or just LMDP, as a tuple $\cL=\langle\cS,\cT,\kernel,\cR,\cJ\rangle$, where: \begin{itemize}
  \item $\cS$ is a set of non-terminal states.
  \item $\cT$ is a set of terminal states.
  \item $\kernel:\cS\rightarrow\Delta(\cS^+)$ is an uncontrolled transition function, also known as passive dynamics or passive controls.
  \item $\cR:\cS\rightarrow\real$ is a reward function for non-terminal states.
  \item $\cJ:\cT\rightarrow\real$ is a reward function for terminal states.
\end{itemize}

\noindent We let $\cS^+=\cS\cup\cT$ denote the full set of states and $B=\max_{s\in\cS}|\cB(\kernel(\cdot|s))|$ an upper bound on the support of $\kernel$. 

Similarly to the stardard RL learning loop, the agent interacts with the environment in a sequential manner. Nontheless, there are no explict actions, and now the learning agent now follows a policy $\pi:\cS\rightarrow\Delta(\cS^+)$. Such a policy chooses, for each non-terminal state $s\in\cS$, a probability distribution over next states in the support of $\kernel(\cdot|s)$, i.e.~$\pi(\cdot|s)\in\Delta(\cB(\kernel(\cdot|s))$. There is also no explicit mention to the initial state distribution, which is asummed to be uniform over the set of non-terminal states (this is $\kernel_0 = \text{unif}(\cS)$). 

At each timestep $t$, the learning agent observes a state $s_t\in\cS^+$. If $s_t$ is non-terminal, the agent transitions to a new state $s_{t+1}\sim\pi(\cdot|s_t)$ and receives an immediate, regularized reward
\[
\cR(s_t, s_{t+1},\pi) = \cR(s_t) - \frac 1 \eta \log \frac {\pi(s_{t+1}|s_t)} {\kernel(s_{t+1}|s_t)},
\] 
where $\cR(s_t)$ is the reward associated with state $s_t$, and $\eta$ is a temperature parameter. 
%  $\mathrm{KL}(\pi(\cdot|s_t)\Vert\, \kernel(\cdot|s_t))$ is the Kullback-Leibler divergence between $\pi(\cdot|s_t)$ and $\kernel(\cdot|s_t)$ defined as 
% \begin{equation*}
%   \mathrm{KL}(\pi(\cdot|s)\Vert\, \kernel(\cdot|s)) = \sum_{s'} \pi(s'\lvert s) \log \frac{\pi(s'\lvert s)}{\kernel(s'\lvert s)}
% \end{equation*}

Hence the agent can set the probability $\pi(s_{t+1}|s_t)$ freely, but gets penalized by the entropy-regularization term for deviating from the passive controls $\kernel(s_{t+1}|s_t)$, and this penalty is modulated by the temperature parameter $\eta$. On the other hand, if $s_t$ is terminal, the agent receives reward $\cJ(s_t)$ and then the current episode ends. The aim of the agent is to compute a policy $\pi$ that maximizes the expected future \textbf{total reward}. For each non-terminal state $s\in\cS$, the value function is defined as
\[
v^\pi_\eta(s) = \EEc{\sum_{i=t}^{T-1} \cR(S_i, S_{i+1},\pi) + \cJ(S_T)}{S_t = s, \pi}.
\]

Here, $T$ is a random variable representing the time at which the current episode ends, and $S_t$ is a random variable representing the state at time $t$. The expectation is over the stochastic choice of next state $S_{t+1}\sim\pi(\cdot|S_t)$ at each time~$t$, and the time $T$ it takes for the episode to end. It is assumed that the reward of all non-terminal states is negative, i.e.~$\cR(s)<0$ for each $s\in\cS$. As a consequence, $\cR(s,\pi)<0$ holds for any policy $\pi$, and the value $v^\pi_\eta(s)$ has a well-defined upper bound. Note that the value function is computed with respect to a concrete value of the temperature parameter $\eta$.  
%As an alternative to the assumption $\cR(s)<0$, we could instead assume that each policy terminates with probability 1 within a fixed time horizon $H$.


We are interested in finding the optimal policy which implies computing the optimal value function ${v^*_\eta:\cS\rightarrow\real}$, i.e.~the maximum expected future total reward among all policies. The value function is extended to each terminal state $\tau\in\cT$ by defining $v^*_\eta(\tau)\equiv\eta\cJ(\tau)$. The value function $v^*_\eta$ satisfies the Bellman equations
\begin{align*}
  \eta v^*_\eta(s) &= \eta \max_\pi \left[ \cR(s,\pi) + \mathbb{E}_{s'\sim\pi(\cdot|s)} v^*_\eta(s') \right] \\
  &= \eta \cR(s) + \max_\pi \mathbb{E}_{s'\sim\pi(\cdot|s)} \left[ \eta v^*_\eta(s') - \log \frac {\pi(s'|s)} {\kernel(s'|s)} \right] \;\; \forall s.
\end{align*}
In expectation, the regularization term in reduces to the Kullback-Liebler divergence between $\pi$ and $\kernel$,
\begin{equation*}
  \mathbb{E}_{s'\sim\pi(\cdot|s)} \log \frac {\pi(s'|s)} {\kernel(s'|s)} = \sum_{s'}\kernel(s'\lvert s)\log\frac{\pi(s'\lvert s)}{\kernel(s'\lvert s)} = \mathrm{KL}\big(\pi(\cdot|s)\Vert\, \kernel(\cdot|s)\big).
\end{equation*}

The maximization in the Bellman equations can be resolved analytically~\citep{Todorov2006}, and the optimal value function can be rewritten as
\begin{align*}
  v^*_\eta(s)      &=  \frac 1 \eta \log{\sum_{s'\in\cS}\kernel(s'\lvert s)e^{\eta(\cR(s) + v^*_\eta(s'))})} \;\;\forall s\in\cS, \\
                   &=  \cR(s) + \frac 1 \eta \log{\sum_{s'\in\cS}\kernel(s'\lvert s)e^{\eta v^*_\eta(s')}} \;\;\forall s\in\cS, \\
  \eta v^*_\eta(s) &=  \eta \cR(s) + \log{\sum_{s'\in\cS}\kernel(s'\lvert s)e^{\eta v^*_\eta(s')}} \;\;\forall s\in\cS.
\end{align*}
Now, we introduce the notation $z(s)=e^{\eta v^*_\eta(s)}$ for each $s\in\cS^+$. We often abuse this notation by referring to $z(s)$ as the (optimal) value of $s$. After exponentiating, the previous system yields the following Bellman optimality equations that are linear in $z$:
\begin{equation}\label{eq:boe_z_lmdp}
z(s) = e^{\eta\cR(s)} \sum_{s'}\kernel(s'|s)z(s').
\end{equation}
\subsubsection{Solving a first-exit LMDP}
The Bellman equation can be expressed in matrix form by defining an $\lvert\cS\rvert\times\lvert\cS\rvert$ diagonal reward matrix $R=\diag(e^{\eta\cR(\cdot)})$ and an $\lvert\cS\rvert\times \lvert\cS^+\rvert$ stochastic transition matrix $P$ whose entries $(s,s')$ equal $\kernel(s'|s)$. We define a vector $\bf z$ that stores the values $z(s)$ for each non-terminal state $s\in\cS$, and a vector $\bf z^+$ extended to all states in $\cS^+$. Now we can write the Bellman equations in matrix form as:
\begin{equation}\label{eq:eigen_lmdp}
{\bf z} = R P {\bf z^+}.
\end{equation}
Given $z$, the optimal policy $\pi$ is given by the following expression for each pair of states $(s,s')$:
\begin{equation}
\label{eq:lmdp_optimal_policy}
\pi(s'|s) =  \frac {\kernel(s'|s) e^{\eta v(s')} } {\sum_{s''} \kernel(s''|s) e^{\eta v(s'')} } = \frac {\kernel(s'|s)z(s')} {\sum_{s''} \kernel(s''|s)z(s'')}.
\end{equation}

The solution for $z$ corresponds to the largest eigenvector of $RP$.
If the dynamics $\kernel$ and $\cR$ are known, we can iterate~\eqref{eq:eigen_lmdp}~\citep{Todorov2006}.

Alternatively, we can learn an estimate $\widehat{z}$ incrementally using stochastic updates based on state transitions sampled from the uncontrolled dynamics $(s_t,r_t,s_{t+1})$
% use an online algorithm called Z-learning to compute an estimate  \citep{TodorovNIPS2007}. After observing each transition , the update rule of Z-learning is given by
\[
\widehat{z}(s_t) \leftarrow (1-\alpha_t)\widehat{z}(s_t) + \alpha_t e^{\eta r_t}\widehat{z}(s_{t+1}),
\]
where $\alpha_t$ is a learning rate. 
The previous update rule is called \emph{Z-learning}~\citep{Todorov2006} and suffers from slow convergence in very large state spaces and when the optimal policy differs substantially from the uncontrolled dynamics $\kernel$.
%assumes that we sample next states using the uncontrolled transition function $\cP$, which is typically no better than a random walk. 
A better choice is importance sampling, which uses samples from the estimated policy $\widehat{\pi}$ derived from the estimated values $\widehat{z}$ and \eqref{eq:lmdp_optimal_policy} and updates $\widehat{z}$ according to the following update
%to obtain samples that are corrected using the following update~\citet{TodorovPNAS2009}:
%to sample In this case,  suggested a corrected update rule based on importance sampling:
\begin{align}\label{eqn:zlearning-imp}
\widehat{z}(s_t) \leftarrow (1-\alpha_t)& \widehat{z}(s_t) + \alpha_t e^{\eta r_t}\widehat{z}(s_{t+1})\frac {\kernel(s_{t+1}|s_t)} {\widehat{\pi}(s_{t+1}|s_t)}.
\end{align}
However, this requires local knowledge of $\kernel(\cdot|s_t)$ to correct for the different sampling distribution.
%, i.e.~the set of possible next states and their associated uncontrolled probabilities.
Though this seems like a strong assumption, in practice $\kernel$ usually has a simple form, e.g.~uniform distribution.
Further, as shown in \citep{Jonsson2016}, the corrected update rule in \eqref{eqn:zlearning-imp} can also be used to perform off-policy updates in case transitions are sampled using a policy different from $\widehat{\pi}$.
%leading to simultaneous learning of different tasks.
%Note that the expression for the policy $\widehat{\pi}$ in \eqref{eq:pi} and the corrected update rule in \eqref{eqn:zlearning-imp} require local knowledge of $\cP(\cdot|s_t)$, i.e.~the set of possible next states and their associated uncontrolled probabilities. Though this seems like a strong assumption, in practice $\cP$ usually has a simple form, e.g.~uniform. \citet{conf/icaps/Jonsson16} showed that the corrected update rule in \eqref{eqn:zlearning-imp} can also be used to perform off-policy updates in case transitions are sampled using a policy different from $\widehat{\pi}$.

\subsubsection{Compositionality}
\label{section:compositionality}
\citet{Todorov2009a} introduces the concept of compositionality for LMDPs. Let us consider a set of LMDPs $\{\cL_1,\ldots,\cL_n\}$, where each LMDP $\cL_i=\langle\cS,\cT,\kernel,\cR,\cJ_i\rangle$ has the same components $\cS,\cT,\kernel,\cR$ and only differ in the reward $\cJ_i(\tau)$ of each terminal state $\tau\in\cT$, as well as its exponentiated value $z_i(\tau)=e^{\eta \cJ_i(\tau)}$.

Now consider a new LMDP $\cL=\langle\cS,\cT,\kernel,\cR,\cJ\rangle$ with the same components as the $n$ LMDPs above, except for $\cJ$. Assume that there exist weights $w_1,\ldots,w_n$ such that the exponentiated value of each terminal state $\tau\in\cT$ can be written as
\[
e^{\eta \cJ(\tau)} = z(\tau) = w_1z_1(\tau) + \ldots + w_nz_n(\tau) = \sum_{k=1}^n w_kz_k(\tau).
\]
Since the Bellman optimality equation of each non-terminal state $s\in\cS$ is linear in $z$, the optimal value of $s$ satisfies the same equation:
\[
z(s) = \sum_{k=1}^n w_kz_k(s).
\]
Consequently, if the optimal values $z_1,\ldots,z_n$ of the $n$ LMDPs are previously computed and the weights $w_1,\ldots,w_n$ are known, we immediately obtains the optimal values of the new LMDP $\cL$ without further learning.


\subsection{Average-reward linearly-solvable Markov decision processes}

LMDPs can be extended to the average-reward setting. We say an average-reward Linearly-solvable Markov decision process (ALMDP) is a tuple
$\cL = \langle\cS,\kernel,\cR\rangle$, where $\cS$ is a set of states, $\kernel:\cS\rightarrow\Delta(\cS)$ is the passive dynamics, and $\cR$ is the reward function. ALMDPs represent infinite-horizon (continuing) tasks and, unlike first-exit LMDPs, there are no terminal states. Consequently, there is no reward function for terminal states.  

Throughout this thesis, we make the following assumptions about ALMDPs. These are common assumptions in the average-reward RL setting to avoid ill-defined value functions.

\begin{assumption}
  The ALMDP $\cL$ is communicating~\citep{Puterman1994}: for each pair of states $s,s'\in\cS$, there exists a policy $\pi$ that has non-zero probability of reaching $s'$ from $s$.
  \label{ass:communicating}
\end{assumption}

\begin{assumption}
  The ALMDP $\cL$ is unichain~\citep{Puterman1994}: the transition probability distribution induced by all stationary policies admit a single recurrent class.
  \label{ass:unichain}
\end{assumption}
%\todo{I think this assumption is enough: no need to introduce the notation for stationary distributions over the state space since we do not use it at any point. This assumption as is is enough to make the gain not to be conditioned by the initial state.}
%{\color{red} \begin{assumption}
%  For all states $s\in\cS$ the reward $\cR(s)$ is bounded in $\left(-\infty, 0\right]$.
%  \label{ass:rewards}
%\end{assumption}}

In the average-reward setting, the value function is defined as the expected \textbf{average reward} when following a policy $\pi:\cS\rightarrow\Delta(\cS)$ starting from a state $s\in\cS$. This is expressed as
\begin{equation}
  v_\eta^\pi(s) = \underset{T\rightarrow\infty}\lim \EEc{\frac{1}{T} \sum_{i=t}^T \cR(S_i, S_{i+1}, \pi)}{S_t = s, \pi},
  \label{eq:value_function_almdp}
\end{equation}
where $\cR(s_t, s_{t+1}, \pi)$ is defined as for first-exit LMDPs.
Again, the goal is to obtain the optimal value function $v^*_\eta$. Under Assumption~\ref{ass:unichain}, the Bellman optimality equations can be written as
\begin{equation}
  v^*_\eta(s) = \frac 1 \eta \log{\sum_{s'\in\cS}\kernel(s'\lvert s)e^{\eta(\cR(s) - \rho + v^*_\eta(s'))}} \;\;\forall s\in\cS,
  \label{eq:boe_almdp}
\end{equation}
where $\rho$ is the optimal one-step average reward (i.e.~gain), which is state-independent for unichain\todo{Add derivation of the independece of the gain?} ALMDPs~\citep{Todorov2006}. Exponentiating yields
\begin{equation}
  z(s) = e^{\eta(\cR(s) - \rho)} \sum_{s'\in\cS}\kernel(s'\lvert s)z(s') \;\;\forall s\in\cS.
  \label{eq:boe_z_almdp}
\end{equation}
For the optimal value function $z$, the optimal policy is given by the same expression as in~\eqref{eq:lmdp_optimal_policy}.

\subsubsection{Solving an ALMDP}

We let $\Gamma=e^{\eta\rho}$ denote the exponentiated gain. Similar to the first-exit case, we can express Equation~\eqref{eq:boe_z_almdp} in matrix form as
\begin{equation}\label{eq:rel_opt}
  \Gamma {\bf z} = R P {\bf z},
\end{equation}
where the matrices $P\in \real^{\lvert\cS\rvert\times\lvert\cS\rvert}$ and $R\in \real^{\lvert\cS\rvert\times\lvert\cS\rvert}$ are appropiately defined as in~\eqref{eq:eigen_lmdp}. The exponentiated gain $\Gamma$ can be shown to correspond to the largest eigenvalue of $RP$~\citep{Todorov2009}.
An ALMDP can be solved using {\em relative value iteration} by selecting a reference state $s^*\in\cS$, initializing $\widehat{\bf z}_0={\bf 1}$ and iteratively applying
\begin{equation*}
  \widehat{\bf z}_{k+\frac 1 2} \gets R P \widehat{\bf z}_k, \quad \quad \widehat{\bf z}_{k+1} \gets \widehat{\bf z}_{k+\frac 1 2} / \widehat z_{k+\frac 1 2}(s^*).
\end{equation*}
The reference state $s^*$ satisfies $z(s^*)=1$, which makes the optimal value $z$ unique (else any constant shift preserves optimality). After convergence, the exponentiated gain equals $\Gamma=\widehat z_{k+\frac 1 2}(s^*)$. Under Assumption~\ref{ass:communicating}, relative value iteration converges to the unique optimal value $z$~\citep{Todorov2009}.

Analogously to first-exit LMDPs, when $\kernel$ and $\cR$ are not known, the agent can learn estimates $\widehat z$ and $\widehat\Gamma$ of the optimal value function and the exponentiated gain in an online manner, using samples $(s_t, r_t, s_{t+1})$ generated when following the estimated policy $\widehat\pi$. The update rules for the so-called \textit{differential Z-learning} algorithm are given by
\begin{align}
  \widehat{z}_{t+1}(s_t)  \gets \widehat{z}_{t}(s_{t}) & + \alpha_t \Bigg(\frac{e^{\eta r_t}}{\widehat\Gamma_t}\dfrac{\kernel(s_{t+1}\lvert s_t)}{\widehat\pi_t(s_{t+1}\lvert s_t)}  - \widehat z_t(s_t)\Bigg),\label{eq:diff_update_z}  \\   %\delta^z_t \\
  \widehat\Gamma_{t+1}    \gets \widehat\Gamma_t       & + \beta_t \Bigg(\frac{e^{\eta r_t}}{\widehat z_t(s_t)}\dfrac{\kernel(s_{t+1}\lvert s_t)}{\widehat\pi_t(s_{t+1}\lvert s_t)}  - \widehat \Gamma_t\Bigg).\label{eq:diff_update_gamma}
\end{align}
The learning rates $\alpha_t$ and $\beta_t$ can be chosen independently.

Unlike the first-exit case, the compositionality property does not hold in the average-reward case.

\subsection{Function approximation in LMDPs}

So far, we have introduced methods for solving (A)LMDPs in the tabular case, where the full state space can be enumerated and stored in memory. However, in cases where the state space is considerably large, tabular methods are unfeasible and the value function is approximated.

In this line,~\cite{Todorov2010} prescibes two families of methods for adapting LMDPs in the average-reward setting to function approximation. The first one lies in the family of policy gradients while the second ones can be considered critic-only approaches. Here a valid parameterization of the policy $\pi(s'\lvert s, \w)$ is considered.

The Bellman equation~\eqref{eq:boe_almdp} can be rewritten as follows
\begin{equation}
\label{eq:boe_for_pg}
\rho + v^*(s) =\cR(s) +\log{\sum_{s'\in\cS}\kernel(s'\lvert s)e^{v^*(s')}} \;\;\forall s\in\cS,
\end{equation}
for the sake of simplicity we assume that $\eta=1$. 

The first flavor of methods builds on the policy gradient theorem for ALMDPs~\citep[cf.~Theorem 1]{Todorov2010} that states that the gradient of the gain in ALMDPs is
\begin{equation}
  \nabla_\w \rho = \sum_{s}\mu(s, \w) \sum_{s'} \nabla_\w \pi(s'\lvert s, \w)\Big( \log \frac{\pi(s'\lvert s, \w)}{\kernel(s'\lvert s)} + \widetilde v(s', \mathbf{r})\Big),
\end{equation}
where $\mu(x, \w)$ is the stationary distribution induced by $\pi$, and $\widetilde v(s,\mathbf{r})$ is an approximation to the (optimal) value function with parameterization $\mathbf r$.

For the second type of methods one could use \textit{approximate} value iteration or \textit{approximate} policy iteration to iteratively improve the weight vector $\w$, even though this results in a biased estimator. An even more direct approach is to use Gauss-Newton method to fit the weight vector $\w$ by directly optimizing~\eqref{eq:boe_for_pg}.

Despite the correctness of the theoretical results, which guarantee that function approximation can be used along with LMDPs, they are of little practical use. Even in the linear function approximation (LFA) case, obtaining a correct parameterization for $\widetilde v(s,\mathbf{r})$ requires of an intricate process in the policy gradient approach. Additionally, the representation of the policy still depends on the size of the whole state space. This makes the approach intractable for real-world problems as it does not scale with large state spaces.


\section{Hierarchical Reinforcement Learning}
Hierarchical Reinforcement learning embodies the idea of divide-and-conquer in sequential decision problems. 
\subsection{Optimality of HRL algorithms}
\cite{Dietterich2000} identifies two types of optimality for hierarchical methods in reinforcement learning, namely \textit{hierarchical optimality} and \textit{recursive optimality}. Hierarchical optimality implies optimality with regard a constrained space of policies, given by the hierarchy structure. Hierarchies of Abtract Machines (HAMs,~\cite{Parr1997}) and the Options Framework~\citep{Sutton1999} lie in this category) 
\subsection{The options framework}
\label{section:options}

\section{Non-Markovian task specification}
\label{section:non_markovian}
 {\color{blue}
 
 \begin{itemize}
 \item Difficulty about expressing some tasks in Marokovian terms 
 \item Finite State Autamatons and Reward Machines 
 \item Reward Machines and their relationship with logics
 \item Comment on algorithms given in the 
  
\end{itemize}
 
 }




\chapter{Globally Optimal Hierarchical Reinforcement Learning for Linearly-solvable Markov Decision Processes}
\section{Introduction}
A major challenge in reinforcement learning is to design agents that are able to learn efficiently and to adapt their existing knowledge to solve new tasks.

One way to reduce the complexity of learning is hierarchical reinforcement learning~\textit{Sutton1999, Dietterich2000, Barto2003}. By decomposing a task into subtasks, each of which can be solved independently, a solution to the original task can then be composed of the solutions to the subtasks. If each subtask is easier to solve than the original task, this may significantly reduce the learning effort of an agent that is learning to perform the task.

We consider Linearly-solvable Markov decision processes (LMDPs), a class of control problems %for reinforcement learning
whose Bellman optimality equations are linear in the (exponentiated) value function~\textit{Todorov2006, Kappen2012}. Because of this, solution methods for LMDPs are more efficient than those for general Markov decision processes (MDPs). 
%, both in the model-based and model-free settings. 
Though not as expressive as MDPs, LMDPs can nevertheless model a wide range of decision problems, and there exist methods for approximating MDPs with LMDPs~\textit{Todorov2006}.

LMDPs frequently appear under the names of path-integral or Kullback-Leibler control in the context of optimal control as probabilistic inference~\textit{Kappen2012,Dvijotham2012,Kappen2005}. LMDPs are also strongly related to maximum-entropy reinforcement learning, which is known to have favorable properties and is quickly becoming the state-of-the-art for reinforcement learning~\textit{Ziebart2010,Mnih2016,Haarnoja2018,Levine2018,Vieillard2020,BasSerrano2021}.

%In the continuous setting, LMDPs frequently appear under the names path-integral control or Kullback-Leibler (KL) control~\textit{KappenML2012}. LMDPs are also strongly related to maximum-entropy reinforcement learning, which is known to have favorable properties and is quickly becoming the state-of-the-art for reinforcement learning~\textit{pmlr-v48-mniha16,pmlr-v80-haarnoja18b,NEURIPS2020_2c6a0bae,pmlr-v130-bas-serrano21a}.

%On the other hand, hierarchical reinforcement learning offers a principled framework for reducing the complexity of learning~\textit{sutton1999between,dietterich2000hierarchical,conf/nips/Wen20}. By decomposing a task into subtasks, each of which can be solved independently, a solution to the original task can then be composed of the solutions to the subtasks. If each subtask is easier to solve than the original task, this may significantly reduce the learning effort of an agent that is learning to perform the task.

One of the computational advantages of LMDPs is compositionality, which allows for zero-shot learning of new skills by linearly combining previously learned base skills which only differ in their cost or reward at boundary states~\textit{Todorov2009,Silva2009}.
%the boundary states of the composite skill and rewards at of the new skill needs to be expressed as a weighted combination of the rewards of the base skills.

%More precisely, one can express the value function at any state of a new task if its cost or reward at boundary states is expressed as a weighted combination of a set of base tasks.
%using the same weighted combination of the value functions of the base tasks. This allows for zero-shot learning of new tasks by linearly combining previously learned base skills~\textit{TodorovNIPS2009,animation}.

In this paper we propose a novel approach to hierarchical reinforcement learning in LMDPs that takes advantage of the compositionality of LMDPs. Our approach assumes that the state space is partitioned into subsets, and the subtasks consist in moving between these partitions. The subtasks are parameterized on the current value estimates of boundary states. Instead of solving the subtasks each time the value estimates change, we take advantage of compositionality to express the solution to an arbitrary subtask as a linear combination of a set of base LMDPs. The result is a form of value function decomposition which allows us to express an estimate of the optimal value of an arbitrary state as a combination of multiple value functions with smaller domains.

Concretely, our work makes the following contributions:
\begin{itemize}
%\item We define a novel scheme based on compositionality for solving subtasks, defining local reward functions that constitute a convenient basis for composite reward functions.
\item We define a novel scheme based on compositionality for solving subtasks, defining local rewards that constitute a convenient basis for composite rewards.
\item The subtask decomposition is at the level of the value function, not of the actual policy. Hence our approach does not suffer from non-stationarity in the online setting, unlike approaches that select among subtasks whose associated policies are being learned.
\item Even though the subtasks have local reward functions, under mild assumptions our approach converges to the globally optimal value function.
\item We analyze experimentally our proposed learning algorithm and show in two classical domains 
that it is more sample efficient compared to a flat learner and similar hierarchical approaches when the set of boundary states is smaller than the entire state space.
\end{itemize}


\section{Related Work}
Several authors have recently exploited concurrent compositionality of tasks in the context of transfer learning.~\textit{Niekerk2019} use the linear compositionality of LMDPs to solve new tasks that can be expressed as combinations of a series of existing base tasks. They show that, while disjunctions of base tasks (OR-compositionality) can be performed exactly, the AND composition (when the goals of base tasks partially overlap) can only be performed approximately.

\textit{Haarnoja2018a} exploit a similar idea to transfer knowledge from existing tasks to new tasks by averaging their reward functions.~\textit{Hunt2019} further extended this by introducing the so-called compositional optimism, and apply divergence correction in case compositionality does not transfer well.

More recently, \textit{NangueTasse2020} derive a formal characterization of union and intersection of tasks in terms of Boolean algebra. They show that learning (extended) value functions that account for all achievable goals, exact zero-shot transfer learning using both AND- and OR- compositionality is possible, achieving an exponential increase in skills compared to the previous works.

All the aforementioned results are derived for general MDPs with \emph{deterministic} dynamics and, possibly, entropy regularization. This setting is no more general than the class of LMDPs or path-integral control.

%~\textit{van2019composing} use compositionality in systems with deterministic dynamics to solve new tasks that can be expressed as combinations of a series of existing tasks. The authors distinguish between AND-compositionality and OR-compositionality, depending on whether the features of a new task are present in all existing tasks or only some.
%and present a method that approximates AND-compositionality and 

%The above forms of compositionality do not guarantee that the resulting policy is optimal, unlike compositionality for LMDPs which is exact.

In this work, we aim to integrate both concurrent task composition, as done in the above approaches, together with hierarchical composition, where skills are chained in a temporal sequence, under the framework of LMDPs.

Several authors have proposed hierarchical versions of LMDPs.~\textit{Jonsson2016} extend MAXQ~\textit{Dietterich2000} to LMDPs by defining subtasks that represent high-level decisions. The top-level policy chooses multi-step transitions, which introduces non-stationarity in the high-level decision process if subtasks are learned concurrently, and also prevents global optimality. The authors discuss the idea of compositionality, but do not explore the concept further.~\textit{Saxe2017} propose a hierarchical multi-task architecture that does exploit compositionality.

Their Multitask LMDP maintains a parallel distributed representation of tasks, reducing the complexity through stacking. However, the approach requires to augment the state space with many additional boundary (subtask) states. Further, the stacking introduces additional costs (cf. their Equation 10), and does not provide global optimality.

The Options Keyboard \textit{Barreto2019} combines a successor feature representation with generalized policy improvement to obtain subtask policies from a set of base subtasks without learning, similar to our use of subtask compositionality. However, unlike in our approach, the composition weights have to be set manually, and although the composed policy is guaranteed to be better than the individual base policies, it is not guaranteed to be optimal.

%\textit{multipl}

Our work is similar to that of \textit{Wen2020} in that we define a hierarchical decomposition based on a partition of the state space, and exploit the equivalence of subtasks to reduce the learning effort. Unlike previous work, however, our approach is not restricted to single initial states, does not suffer from non-stationarity in the online setting, proposes a more general definition of equivalence that captures more structure, and guarantees convergence to the optimal value function for stochastic dynamics.

The concept of equivalent subtasks is strongly related to factored (L)MDPs, which capture conditional independence among a set of state variables~\textit{Boutilier1995, Koller2000}. Equivalence arises whenever a subset of state variables are conditionally independent of another subset. Several authors have shown how to automatically discover the structure of factored MDPs from experience~\textit{Strehl2007,Kolobov2012}, which in turn could be used to define equivalence classes of subtasks.

%{\color{red} Mention here the other papers \textit{hunt2019composing}\textit{van2019composing},\textit{algebra}}

\section{Hierarchical LMDPs}

In this section we describe our novel approach to hierarchical LMDPs. We first describe the particular form of hierarchical decomposition that we consider, and then present algorithms for solving a decomposed LMDP.

\subsection{Hierarchical Decomposition}

Our hierarchical decomposition is similar to that of~\textit{Wen2020}. Formally, given an LMDP $\cL=\langle\cS,\cT,\cP,\cR,\cJ\rangle$, the set of non-terminal states $\cS$ is partitioned into $L$ subsets $\{\cS_i\}_{i=1}^L$. For each such subset $\cS_i$, we define an induced subtask $\cL_i=\langle\cS_i,\cT_i,\cP_i,\cR_i,\cJ_i\rangle$, i.e.~an LMDP whose components are defined as follows:
\begin{itemize}
\item The set of non-terminal states is $\cS_i$.
\item The set of terminal states $\cT_i=\{\tSi \in\cS^+\setminus\cS_i:\exists s\in \cS_i \; \text{s.t.} \; \tSi \in \cB(\cP(\cdot|s))\}$ includes all states in $\cS^+\setminus\cS_i$ (terminal or non-terminal) that are reachable in one step from a state in $\cS_i$.
\item $\cP_i:\cS_i\rightarrow\Delta(\cS_i^+)$ and $\cR_i:\cS_i\rightarrow\real$ are the restrictions of $\cP$ and $\cR$ to $\cS_i$, where $\cS_i^+=\cS_i\cup\cT_i$ denotes the full set of subtask states.
%\item The reward of a terminal state $\tS\in\cT_i$ equals $\cJ_i(\tS)=\cJ(\tS)$ if $\tS\in\cT$, and $\cJ_i(\tS)=\hat{v}(\tS)$ otherwise, where $\hat{v}(\tS)$ is the estimated value in $\cL$ of the non-terminal state in  $\tS\in\cS \setminus \cS_i$.
\item The reward of a terminal state $\tSi \in\cT_i$ equals $\cJ_i(\tSi)=\cJ(\tSi)$ if $\tSi\in\cT$, and $\cJ_i(\tSi)=\hat{v}(\tSi)$ otherwise, where $\hat{v}(\tSi)$ is the estimated value in $\cL$ of the non-terminal state in  $\tSi\in\cS \setminus \cS_i$.
\end{itemize}

Intuitively, if the reward $\cJ_i(\tSi)$ of each terminal state $\tSi\in\cT_i$ equals its optimal value $v(\tSi)$ for the original LMDP~$\cL$, then solving the subtask $\cL_i$ yields the optimal values of the states in $\cS_i$.
In practice, however, we only have access to an estimate $\hat{v}(\tSi)$ of the optimal value.
% VG : what do you mean in practice: during learning? in the presence of function approximation?
In this case, the subtask $\cL_i$ is {\em parameterized} on the value estimate $\hat{v}$ of terminal states in $\cT_i$, and each time the value estimate changes, we can solve $\cL_i$ to obtain a new value estimate
$\hat{v}(s)$ for each state $s\in\cS_i$.



We define a set of {\em exit states} $\cE=\cup_{i=1}^L\cT_i$, i.e.~the union of the terminal states of each subtask in $\{\cL_1,\ldots,\cL_L\}$. For convenience, we use $\cE_i=\cE\cap\cS_i$ to denote the set of (non-terminal) exit states in the subtask $\cL_i$. We also introduce the notation $K=\max_{i=1}^L|\cS_i|$, $N=\max_{i=1}^L|\cT_i|$ and $E=|\cE|$.

Just like \textit{Wen2020}, we define a notion of equivalent subtasks.
\begin{definition}
Two subtasks $\cL_i$ and $\cL_j$ are equivalent if there exists a bijection $f:\cS_i\rightarrow\cS_j$ such that the transition probabilities and rewards of non-terminal states are equivalent through $f$.
\end{definition}
Unlike \textit{Wen2020}, we do {\em not} require the sets of terminal states $\cT_i$ and $\cT_j$ to be equivalent. Instead, for each class of equivalent subtasks, our approach is to define a single subtask whose set of terminal states is the {\em union} of the sets of terminal states of subtasks in the class.

Formally, we define a set of equivalence classes $\cC=\{\cC_1,\ldots,\cC_C\}$, $C\leq L$, i.e.~a partition of the set of subtasks $\{\cL_1,\ldots,\cL_L\}$ such that all subtasks in a given partition are equivalent. We represent a single subtask $\cL_j=\langle\cS_j,\cT_j,\cP_j,\cR_j,\cJ_j\rangle$ per equivalence class $\cC_j\in\cC$. The components $\cS_j,\cP_j,\cR_j$ are shared by all subtasks in the equivalence class, while the set of terminal states is $\cT_j=\bigcup_{\cL_i\in\cC_j} \cT_i$, where the union is taken w.r.t. the bijection $f$ relating all equivalent subtasks. As before, the reward $\cJ_j$ of terminal states is parameterized on a given value estimate $\hat{v}$. We assume that each non-terminal state $s\in\cS$ can be easily mapped to its subtask $\cL_i$ and equivalence class $\cC_j$.


\begin{figure}[!t]
\begin{center}
\begin{adjustbox}{width=\columnwidth}
\begin{tikzpicture}
\draw[step=0.4,thin,shift={(0.2,0.2)}] (0.8,0.8) grid (4.8,4.8);
\draw[ultra thick] (1,1) rectangle (5,5);
\draw[ultra thick] (3,1) -- (3,1.8);
\draw[ultra thick] (3,2.2) -- (3,3.8);
\draw[ultra thick] (3,4.2) -- (3,5);
\draw[ultra thick] (1,3) -- (1.8,3);
\draw[ultra thick] (2.2,3) -- (3.8,3);
\draw[ultra thick] (4.2,3) -- (5,3);

%\draw[fill] (0.6,1.8) rectangle (1,2.2);
%\draw[fill] (0.6,3.8) rectangle (1,4.2);
%\draw[fill] (1.8,5) rectangle (2.2,5.4);
%\draw[fill] (1.8,0.6) rectangle (2.2,1);
%\draw[fill] (3.8,0.6) rectangle (4.2,1);
%\draw[fill] (5,1.8) rectangle (5.4,2.2);
%\draw[fill] (5,3.8) rectangle (5.4,4.2);
%\draw[fill] (3.8,5) rectangle (4.2,5.4);

\draw[ultra thick] (4.2,4.2) rectangle (4.6,4.6);
\draw[ultra thick] (3.8,2.6) rectangle (4.2,3.4);
\draw[ultra thick] (1.8,2.6) rectangle (2.2,3.4);
\draw[ultra thick] (2.6,3.8) rectangle (3.4,4.2);
\draw[ultra thick] (2.6,1.8) rectangle (3.4,2.2);

\node at (4.4,4.4) {\small $F$};
\node at (2,3.2) {\small $3^T$};
\node at (2,2.8) {\small $1^B$};
\node at (4,3.2) {\small $4^T$};
\node at (4,2.8) {\small $2^B$};
\node at (2.8,4) {\small $2^L$};
\node at (2.8,2) {\small $4^L$};
\node at (3.2,4) {\small $1^R$};
\node at (3.2,2) {\small $3^R$};

\draw[step=0.4,thin,shift={(0.2,0)}] (8.799,1.999) grid (10.8,4);
\draw[ultra thick] (9,3.2) -- (8.6,3.2) -- (8.6,2.8) -- (9,2.8) -- (9,2) -- (9.8,2);
\draw[ultra thick] (9,3.2) -- (9,4) -- (9.8,4) -- (9.8,4.4) -- (10.2,4.4) -- (10.2,4);
\draw[ultra thick] (10.2,4) -- (11,4) -- (11,3.2) -- (11.4,3.2) -- (11.4,2.8) -- (11,2.8);
\draw[ultra thick] (9.8,2) -- (9.8,1.6) -- (10.2,1.6) -- (10.2,2) -- (11,2) -- (11,2.8);
\draw[ultra thick] (10.2,3.2) rectangle (10.6,3.6);

\node at (10.4,3.4) {\small $F$};
\node at (8.8,3)    {\small $L$};
\node at (11.2,3)   {\small $R$};
\node at (10,1.8)   {\small $B$};
\node at (10,4.2)   {\small $T$};

\node at (3,0.7) {\Large a)};
\node at (10,0.7) {\Large b)};
\end{tikzpicture}
\end{adjustbox}
\end{center}

\caption{a) A 4-room LMDP, with a terminal state $F$ and 8 other exit states; b) a single subtask with 5 terminal states $F,L,R,T,B$ that is equivalent to all 4 room subtasks. Rooms are numbered 1 through 4, left-to-right, then top-to-bottom, and exit state $1^B$ refers to the exit $B$ of room $1$, etc.}
\label{fig:ex}
\end{figure}


\paragraph{Example 1:} Figure~\ref{fig:ex}a) shows an example 4-room LMDP with a single terminal state marked $F$, separate from the room but reachable in one step from the highlighted location. The rooms are only connected via a single doorway; hence if we partition the states by room, the subtask corresponding to each room has two terminal states in other rooms, plus the terminal state $F$ for the top right room. The 9 exit states in $\cE$ are highlighted and correspond to states next to doorways, plus $F$. Figure~\ref{fig:ex}b) shows a single subtask that is equivalent to all four room subtasks, since dynamics is shared inside rooms and the set of terminal states is the union of those of the subtasks.
Hence the number of equivalent subtasks is $C=1$, the number of non-terminal and terminal states of subtasks is $K=25$ and $N=5$, respectively, and the number of exit states is $E=9$.

\subsection{Subtask Compositionality}

During learning, the value estimate $\hat{v}$ changes frequently, and it is inefficient to solve all subtasks after each change. Instead, our approach is to use compositionality to obtain solutions to the subtasks without learning. The idea is to introduce several base LMDPs for each subtask $\cL_j$ such that {\em any} reward function $\cJ_j$ can be expressed as a combination of the reward functions of the base LMDPs.
% VG: and learn them simultaneously?

Given a subtask $\cL_j=\langle\cS_j,\cT_j,\cP_j,\cR_j,\cJ_j\rangle$ as defined above, assume that the set $\cT_j$ contains $n$ states, i.e.~$\cT_j=\{\tSi_1,\ldots,\tSi_n\}$. We define $n$ base LMDPs $\cL_j^1,\ldots,\cL_j^n$, where each base LMDP is given by $\cL_j^k=\langle\cS_j,\cT_j,\cP_j,\cR_j,\cJ_j^k\rangle$. Hence the base LMDPs only differ in the reward of terminal states.
Concretely, we define the exponentiated reward as $z_j^k(\tSi)=1$ if $\tSi=\tSi_k$, and $z_j^k(\tSi)=0$ otherwise.
This corresponds to an actual reward of $\cJ_j^k(\tSi)=0$ for $\tSi=\tSi_k$, and $\cJ_j^k(\tSi)=-\infty$ otherwise.

Even though the exit reward $\cJ_j^k(\tSi)$ equals negative infinity for terminal states different from $\tSi_k$, this does not cause computational issues in the exponentiated space, since the value $z_j^k(\tSi)=0$ is well-defined in \eqref{eq:matrixz} and \eqref{eq:pi}.
Moreover, there are two good reasons for defining the rewards in this way. The first is that the rewards form a convenient basis that allows us to express {\em any} value estimate on the terminal states in $\cT_j$ as a linear combination of $z_j^1,\ldots,z_j^n$.
The second is that a value estimate $\hat{z}(\tSi)=0$ can be used to {\em turn off} terminal state $\tSi$, since the definition of the optimal policy in \eqref{eq:pi} assigns probability $0$ to any transition that leads to a state $\tSi$ with $\hat{z}(\tSi)=0$. % departing from it?
This is the reason that we do not need the sets of terminal states to be equal for equivalent subtasks.

Now assume that we solve the base LMDPs to obtain the optimal value functions $z_j^1,\ldots,z_j^n$. Also assume a given value estimate $\hat{v}$ for the terminal states in $\cT_j$, i.e.~$\cJ_j(\tSi)=\hat{v}(\tSi)$ for each $\tSi\in\cT_j$. Then we can write the exponentiated reward $\hat{z}(\tSi)=e^{\hat{v}(\tSi)/\lambda}$ of each terminal state as
\begin{equation}\label{eq:comp}
\hat{z}(\tSi) = \sum_{k=1}^n w_kz_j^k(\tSi) = \sum_{k=1}^n \hat{z}(\tSi_k) z_j^k(\tSi),
\end{equation}
where each weight is simply given by $w_k=\hat{z}(\tSi_k)$. This is because for a given terminal state $\tSi_\ell\in\cT_j$, the value $z_j^k(\tSi_\ell)$ equals $0$ for $k\neq \ell$, so the weighted sum simplifies to $w_\ell z_j^\ell(\tSi_\ell)=w_\ell\cdot 1=\hat{z}(\tSi_\ell)$.

Due to compositionality, we can now write the estimated value of each non-terminal state $s\in\cS_i$ as
\begin{align}\label{eq:comp3}
\hat{z}(s)=\sum_{k=1}^n \hat{z}(\tSi_k) z_j^k(s) \;\; \forall s\in\cS_i,\forall\cL_i\in\cC_j.
\end{align}
Here, the terminal states $\tSi_1,\ldots,\tSi_n$ are by definition exit states in $\cE$. If we have access to a value estimate $\hat{z}_\cE:\cE\rightarrow\mathbb{R}$ on exit states, as well as the value functions $z_j^1,\ldots,z_j^n$ of all base LMDPs, we can thus use \eqref{eq:comp3} to express the value estimate of each other state without learning. Hence~\eqref{eq:comp3} is a form of value function decomposition, allowing us to express the values of arbitrary states in $\cS$ in terms of value functions with smaller domains. Concretely, there are $O(CN)$ base LMDPs, each with $O(M)$ values, so in total we need $O(CMN+E)$ values for the decomposition.

\paragraph{Example 1:} %Returning to the example in Figure~\ref{fig:ex},
In the 4-room example, there are five base LMDPs with value functions $z^F$, $z^L$, $z^R$, $z^T$ and $z^B$, respectively. Given an initial value estimate $\hat{z}_\cE$ for each exit state in $\cE$, a value estimate of any state in the top left room is given by $\hat{z}(s)=\hat{z}_\cE(1^B) z^B(s) + \hat{z}_\cE(1^R) z^R(s)$, where we use $\hat{z}_\cE(F)=\hat{z}_\cE(L)=\hat{z}_\cE(T)=0$ to indicate that the terminal states $F$, $L$ and $T$ are not present in the top left room. We need $CMN = 125$ values to store the value functions of the 5 base LMDPs, and $E=9$ values to store the value estimates of all exit states. Although this is more than the 100 states of the original LMDP, if we increase the number of rooms to $X\times Y$, the term $CMN$ is a constant as long as all rooms have equivalent dynamics, and the number of exit states is $E=(2X-1)(2Y-1)$, which is much smaller than the $25XY$ total states. For $10\times 10$ rooms, the value function decomposition requires $486$ values to represent the values of $2{,}500$ states.\\

%The example in Figure~\ref{fig:ex} is brittle, in the sense that changing the configuration and size of the rooms may break the assumption of equivalence, which in turn makes the hierarchical approach less powerful. However, the notion of equivalence is naturally associated with factored (L)MDPs, in which the state is factored into a set of variables $\cV=\{v_1,\ldots,v_m\}$, i.e.~$\cS=\cD(v_1)\times\cdots\times\cD(v_m)$, where $D(v_i)$ is the domain of variable $v_i$, $1\leq i\leq m$.
The 4-room example is limited in the sense that changing the configuration and size of the rooms may break the assumption of equivalence, which in turn makes the hierarchical approach less powerful. However, the notion of equivalence is naturally associated with factored (L)MDPs, in which the state is factored into a set of variables $\cV=\{v_1,\ldots,v_m\}$, i.e.~$\cS=\cD(v_1)\times\cdots\times\cD(v_m)$, where $D(v_i)$ is the domain of variable $v_i$, $1\leq i\leq m$.
Concretely, if there is a subset of variables $\cU\subset\cV$ such that the transitions among $\cU$ are independent of the variables in $\cV\setminus\cU$, then it is natural to partition the states based on their assignment to the variables in $\cV\setminus\cU$. Consequently, there is a single equivalent subtask whose set of states is $\times_{v\in\cU}D(v)$, i.e.~all partial states on the variables in $\cU$.

\paragraph{Example 2:} The Taxi domain~\textit{Dietterich2020} is described by three variables: the location of the taxi ($v_1$), and the location and destination of the passenger ($v_2$ and $v_3$). Since the location of the taxi is independent of the other two, it is natural to partition the states according to the location and destination of the passenger. Each partition consists of the possible locations of the taxi, defining a unique equivalent subtask whose terminal states are the locations at which the taxi can pick up or drop off passengers. Since there are 16 valid combinations of passenger location and destination, there are 16 such equivalent subtasks.
\textit{Dietterich2000} calls this condition {\em max node irrelevance}, where ``max node'' refers to a given subtask.

\subsection{Eigenvector Approach}

If the dynamics $\cP$ and the state costs $\cR, \cJ$ are known, we can use the power method to solve the original LMDP $\cL$ by composing individual solutions of the subtask LMDPs $\cL_i$.
In this case, we define Bellman equations in \eqref{eq:matrixz} to solve the base LMDPs of all equivalence classes.
To compute the values of the original LMDP $\cL$ for the exit states in $\cE$, the compositionality relation in~\eqref{eq:comp3} provides us with an additional system of linear equations, one for each non-terminal exit state. We can reformulate this additional system of equations in matrix form defined for the exit states ${\bf z}_\cE$:
\begin{equation}\label{eq:exits}
{\bf z}_\cE=G{\bf z}_\cE.
\end{equation}
Here, the matrix $G$ contains the values of the base LMDPs according to~\eqref{eq:comp3}.
We can thus use the power method on this system of linear equations to obtain the values of all exit states in $\cE$.

\paragraph{Example 1:} In the 4-room example, the row in $G$ corresponding to $\hat{z}_\cE(2^L)$ contains the element $z^B(2^L)$ in the column for $\hat{z}_\cE(1^B)$, and the element $z^R(2^L)$ in the column for $\hat{z}_\cE(1^R)$, while all other elements equal $0$. While the flat approach requires one run of the power method on a large matrix, %of $100\times 100$ states
our hierachical approach needs five runs of the power method on significantly reduced %$(5+4)\times (5+4)$
 matrices (these runs can be parallelized), and one additional run on a $8\times 8$ matrix, corresponding to~\eqref{eq:exits}.\\ %, to obtain the same global optimal solution.

We remark that we do not explicitly represent the values of states in $\cS\setminus\cE$ since they are given by~\eqref{eq:comp3}.
Since we can now obtain the value $z(s)$ of each state $s\in\cS$, we can define the optimal policy directly in terms of the values $z$ and~\eqref{eq:pi}. Hence unlike most approaches to hierarchical reinforcement learning, the policy does not select among subtasks, but instead depends directly on the decomposed value estimates.

\subsection{Online and Intra-task Learning}

%Alternatively, in the learning setting where there agent is learning by interacting with the environment, we need to maintain 

In the online learning case, we need to maintain estimates $\hat{z}_j^1,\ldots,\hat{z}_j^n$ of the value functions of the base LMDPs associated with each equivalent subtask $\cL_j$.
These estimates can be updated using the Z-learning rule~\eqref{eqn:zlearning-imp} after each transition.
But to make learning more efficient, we can use a single transition $(s_t,r_t,s_{t+1})$ with $s_t\in\cS_j$ to update the values of {\em all} base LMDPs associated with $\cL_j$ simultaneously. This is known in the literature as intra-task learning~\textit{Kaelbling1993,Jonsson2016}.

%In the case of unknown dynamics, we need to maintain estimates $\hat{z}_j^1,\ldots,\hat{z}_j^n$ of the value functions of the base LMDPs associated with each equivalent subtask $\cL_j$. These estimates are updated using the Z-learning rule in~\eqref{eqn:zlearning-imp} after each transition. To make learning more efficient, we can use a single transition $(s_t,r_t,s_{t+1})$ with $s_t\in\cS_j$ to update the values of {\em all} base LMDPs associated with $\cL_j$ simultaneously. This is known in the literature as intra-task learning~\textit{Kaelbling93}.

Given the estimates $\hat{z}_j^1,\ldots,\hat{z}_j^n$, we could then formulate and solve the same system of linear equations in~\eqref{eq:comp3} to obtain the value estimates of exit states. However, it is impractical to solve this system of equations every time we update %the estimates 
$\hat{z}_j^1,\ldots,\hat{z}_j^n$. Instead, we explicitly maintain estimates $\hat{z}_\cE$ of the values of exit states in the set $\cE$, and update these values incrementally. 
For that, we turn~\eqref{eq:comp3} into an update rule:
\begin{align}\label{eq:comprule}
\hat{z}_\cE(s) \leftarrow &(1 - \alpha_\ell) \hat{z}_\cE(s) + \alpha_\ell \sum_{k=1}^n \hat{z}_j^k(s) \hat{z}_\cE(\tSi_k).
\end{align}
The question is when to update the value of an exit state. We propose several alternatives:
\begin{itemize}
\item[$V_1$:] Update the value of an exit state $s\in\cE_i$ each time we take a transition from $s$.
\item[$V_2$:] When we reach a terminal state of the subtask $\cL_i$, update the values of all exit states in $\cE_i$.
\item[$V_3$:] When we reach a terminal state of the subtask $\cL_i$, update the values of all exit states in $\cE_i$ and all exit states of subtasks in the equivalence class $\cC_j$ of $\cL_i$.
\end{itemize}
Again, the estimated policy $\pi$ is defined directly by the value estimates $\hat{z}$ and~\eqref{eq:pi}, and thus does not select among subtasks. Below is the pseudo-code of the proposed algorithm.

% \begin{algorithm}[H]
% \setstretch{1.12}
% \renewcommand{\thealgorithm}{}
% \caption{Online and Intra-Task Learning Algorithm}
% \begin{algorithmic}[1]
% \STATE {\bf Input:} An LMDP ${\cL = \langle \cS, \cT, \cP, \cR, \cJ \rangle}$ and
% a partition $\{\cS_i\}_{i=1}^L$ of $\cS$ \newline
% A set $\{\cC_1,\ldots,\cC_C\}$ of equivalent subtasks and related base LMDPs $\mathcal{L}_j^k = \langle \cS_j, \cT_j, \cP_j, \cR_j, \cJ_j^k \rangle$

% \STATE {\bf Initialization:} \newline
% $\hat z_\cE(s) := 1 \;\; (\forall s \in \cE)$ \COMMENT {high-level Z function approximation} \newline
% $\hat z_j^k(s) := 1$ \COMMENT{base LMDPs $1 \dots |\cT_j|$ for each equivalent subtask $\cL_j$}

% \WHILE{$\text{termination condition is not met}$}
% %\STATE update all $\alpha_\ell_{\ell}$
% \STATE observe transition $s_t, r_t, s_{t+1} \sim \hat\pi(\cdot|s_{t})$, where $s_t\in\cS_i$ and $\cL_i\in\cC_j$
% %\STATE $\text{update high-level estimation } \hat z(s_t)$
% \STATE $\text{update lower-level estimations } \hat z_j^k(s_t)$ \text{using} \eqref{eqn:zlearning-imp}
% \IF[$s_t$ is an exit or $s_{t+1}$ is terminal for current subtask $\cL_j$]{$s_t\in\cE$ or $s_{t+1} \in \cT_j$}
% \STATE $\text{apply \eqref{eq:comprule} to update $\hat z_\cE$ using variant } V_1$, $V_2$ or $V_3$
% \ENDIF
% \ENDWHILE

% \end{algorithmic}
% \end{algorithm}

\begin{figure*}[!htb]
\setlength{\belowcaptionskip}{-10pt}
\centering
%\includegraphics[scale=0.2]{Figures/overall_grid_3_3_1-1.png}
% \includegraphics[scale=0.2]{Figures/nroom_3_3-1.png}
%\includegraphics[scale=0.2]{Figures/overall_grid_5_5_1_3-1.png}
% \includegraphics[scale=0.2]{Figures/nroom_5_5-1.png}
% \includegraphics[scale=0.2]{Figures/nroom_8_8-1.png}\\
%\includegraphics[scale=0.2]{Figures/grid_with_Q.png}\\
%a) \hspace*{3.8cm} b) \hspace*{3.8cm} c) %\hspace*{3.8cm} d)
\caption{ Results for $3\times 3$ rooms of size $5 \times 5$ (left);
$5\times 5$ rooms of size $3 \times 3$ (center); $8 \times 8$ rooms of size $5\times 5$ (right).}
%; and d) $5 \times 5$ with Q-learning ($Q_o$).}
\label{fig:hlmdps_errors_nrooms}
\end{figure*}


\subsection{Analysis}
Let $\cL=\langle\cS,\cT,\cP,\cR,\cJ\rangle$ be an LMDP, and let $\cL_i=\langle\cS_i,\cT_i,\cP_i,\cR_i,\cJ_i\rangle$ be a subtask associated with the partition $\cS_i\subseteq\cS$. Let $z$ denote the optimal value of $\cL$, and let $z_i$ denote the optimal value of $\cL_i$.

\begin{lemma}\label{lemma:same}
If the reward of each terminal state $\tSi\in\cT_i$ equals its optimal value in $\cL$, i.e.~$z_i(\tSi)=z(\tSi)$, the optimal value of each non-terminal state $s\in\cS_i$ equals its optimal value in $\cL$, i.e.~$z_i(s)=z(s)$.
\end{lemma}

\begin{proof}
Since $\cP_i$ and $\cR_i$ are the restriction of $\cP$ and $\cR$ onto $\cS_i$, for each $s\in\cS_i$ we have
\begin{align*}
    z_i(s) &= e^{\cR_i(s)/\lambda} \sum_{s'}\cP_i(s'|s)z_i(s') \\
    &= e^{\cR(s)/\lambda} \sum_{s'}\cP(s'|s)z_i(s'),
\end{align*}
which is %exactly
 the same Bellman equation as for $z(s)$.
Since $z_i(\tSi)=z(\tSi)$ for each terminal state $\tSi\in\cT_i$, we immediately obtain $z_i(s)=z(s)$ for each non-terminal state $s\in\cS_i$.
\end{proof}
As an consequence of Lemma~\ref{lemma:same}, assigning the optimal value $z(\tSi)$ to each exit state $\tSi\in\cE$ yields a solution to~\eqref{eq:exits}, which is thus guaranteed to have a solution with eigenvalue~$1$. Lemma~\ref{lemma:same} also guarantees that we can use \eqref{eq:comp3} to compute the optimal value of any arbitrary state given optimal values of the base LMDPs and the exit states. The only necessary conditions needed for convergence to the optimal value function is that $(i)$ $\{\cS_i\}_{i=1}^L$ is a proper partition of the state space; and $(ii)$ the set of terminal states $\cT_i$ of each subtask $\cL_i$ includes all states reachable in one step from $\cS_i$.

%\setcounter{lemma}{0}

%\setcounter{\xlemma}{0}
\begin{lemma}
The solution to \eqref{eq:exits} is unique.
\end{lemma}

\begin{proof}
By contradiction. Assume that there exists a solution ${\bf z}_\cE'$ which is different from the optimal values ${\bf z}_\cE$. We can extend ${\bf z}$ and ${\bf z}'$ to all states in $\cS$ by applying \eqref{eq:comp3}. Due to the same argument as in the proof of Lemma~\ref{lemma:same}, the solution ${\bf z}'$ satisfies the Bellman optimality equation of all states in $\cS$. Hence ${\bf z}'$ is an optimal value function for the original LMDP $\cL$, which contradicts that ${\bf z}'$ is different from ${\bf z}$ since the Bellman optimality equations have a unique solution.
\end{proof}

\begin{lemma}
For each subtask $\cL_i$ and state $s\in\cS_i^+$, it holds that $z_i^1(s)+\cdots+z_i^n(s)\leq 1$.
\end{lemma}

\begin{proof}
By induction. The base case is given by terminal states $t_\ell\in\cT_i$, in which case $z_i^1(t_\ell)+\cdots+z_i^n(t_\ell) = z_i^\ell(t_\ell) = 1$. For $s\in\cS_i$, the Bellman equation for each base LMDP yields
\begin{align*}
    \sum_{k=1}^n z_i^k(s)=e^{\cR_i(s)/\lambda}\sum_{s'}\cP(s'|s)\sum_{k=1}^n z_i^k(s').
\end{align*}


Since $\cR_i(s)=\cR(s)<0$ holds by assumption, and since $z_i^1(s')+\cdots+z_i^n(s')\leq 1$ holds for each $s'$ by hypothesis of induction, it follows that $z_i^1(s)+\cdots+z_i^n(s)\leq 1$.
\end{proof}
As a consequence, just like the matrix $RP$ in \eqref{eq:matrixz}, the matrix $G$ in \eqref{eq:exits} has spectral radius at most $1$, and hence the power method is guaranteed to converge to the unique solution with largest eigenvalue $1$, corresponding to the optimal values of the exit states.

The convergence rate of the power method is exponential in $\gamma<1$, the eigenvalue of $RP$ or $G$ with second largest value and independent of the state space.
The average running time scales linearly with the number of non-zero elements in $RP$ or $G$~\textit{Todorov2006}, which is drastically reduced compared to the non-hierarchical approach. 
%Assuming a similar convergence rate, the time complexity of the power method depends on the matrix multiplication.
More precisely, given an upper bound $B$ on the support of $\cP$ and a sparse representation, the matrix multiplication in~\eqref{eq:matrixz} has complexity $\mathcal{O}(BS)$. In comparison, the matrix multiplication of the $\mathcal{O}(CN)$ base LMDPs has complexity $\mathcal{O}(BK)$, while the matrix multiplication in \eqref{eq:exits} has complexity $\mathcal{O}(NE)$. Hence the hierarchical approach is competitive whenever $\mathcal{O}(CNBK+NE)$ is smaller than $\mathcal{O}(BS)$. In a $10\times 10$ room example, $CNBK+NE=500+1{,}805=2{,}305$, while $BS=10{,}000$.

%The convergence rate of the power iteration method is $\lambda<1$, where $\lambda$ is the eigenvalue of $RP$ or $G$ with second largest value, i.e.~less than $1$. Assuming a similar convergence rate, the time complexity of the power iteration method depends on the matrix multiplication. Given an upper bound $B$ on the support of $P$ and a sparse representation, the matrix multiplication in~\eqref{eq:matrixz} has complexity $O(BS)$. In comparison, the matrix multiplication of the $O(CN)$ base LMDPs has complexity $O(BK)$, while the matrix multiplication in \eqref{eq:exits} has complexity $O(NE)$. Hence the hierarchical approach is competitive whenever $O(CNBK+NE)$ is smaller than $O(BS)$. In the $10\times 10$ room example, $CNBK+NE=500+1{,}805=2{,}305$, while $BS=10{,}000$.

\section{Experiments}
We now evaluate the proposed learning algorithm in the two previous examples.\footnote{Code available at https://github.com/guillermoim/HRL\_LMDP}
% Vicenc:
% \footnote{texttt{https://github.com/guillermoim/HRL\_LMDP}}
%, the Rooms domain and the Taxi domain.
The objective of this evaluation is to analyze empirically the different update alternatives ($V_1$, $V_2$, and $V_3$), and
to compare against a flat approach which exploits the benefits of LMDPs without the hierarchy (Z-IS), and the hierarchical approach based on options ($Q_o$)~\textit{Sutton1999}. Our main objective is to empirically show that our approach is more sample efficient than the other algorithms. We run each algorithm with four different random seeds to analyze the average MAE (mean absolute error) against the optimal value function (computed separately) and its standard deviation over the number of samples. Since the value functions are different for Q-learning and LMDP methods, we present the self-normalized MAE (Figures \ref{fig:errors_grid} and \ref{fig:errors_taxi}) for different configurations and domains. Further, for a fair comparison between approaches, we only use the exit set for calculating the MAE.

%In this section, we empirically test the model-free algorithm in several experiments. In the first set of experiments, we extend the example from Figure~\ref{fig:ex} to $X\times Y$ rooms. In the second set of experiments, we use the Taxi domain~\textit{dietterich2000hierarchical}.

In all experiments, the learning rates for each abstraction level is $\alpha_\ell(t) = c_\ell / (c_\ell + n)$ where $n$ represents the episode each sample $t$ belongs to. We empirically optimize the constant $c_\ell$ for each domain. For LMDPs, we use a temperature $\lambda=1$, which provides good results. $Q_o$ solves an equivalent MDP with {\em deterministic} actions, which should actually give it an advantage. For fairness, $Q_o$ obtains the same per-step negative reward, exploits the same equivalence classes, learns the same subtasks (i.e.~reach a terminal state), and has knowledge of which options are available in each state.

\subsection{Rooms Domain.}
We analyze the performance for different room sizes and number of rooms (Figure \ref{fig:errors_grid}). In all configurations the proposed hierarchical approach outperfoms Z-IS and $Q_o$. Concretely, $Q_o$ suffers from non-stationarity: initial option executions will incur more negative reward than later executions, which causes high-level Q-learning updates to be {\em incorrect}, and it takes the learner significant time to recover from this.

\begin{figure}[!htb]
\centering
\includegraphics[width=0.32\textwidth]{figures/chapter1/nroom_3_3.png}
\includegraphics[width=0.32\textwidth]{figures/chapter1/nroom_5_5.png}
\includegraphics[width=0.32\textwidth]{figures/chapter1/nroom_8_8.png}\\
\caption{ Results for $3\times 3$ rooms of size $5 \times 5$ (left);
$5\times 5$ rooms of size $3 \times 3$ (center); $8 \times 8$ rooms of size $5\times 5$ (right).}
%; and d) $5 \times 5$ with Q-learning ($Q_o$).}
\label{fig:errors_grid}
\end{figure}

Figure~\ref{fig:errors_grid} (left) shows results for $3\times 3$ rooms of size $5\times 5$ and Figure~\ref{fig:errors_grid} (center) shows results for $5\times 5$ rooms of size $3\times 3$. Both scenarios have $225$ interior states.
The difference between variants $V_1$, $V_2$ and $V_3$ is more pronounced in the second case, when the number of subtasks increases (more rooms) and the partition for each subtask is smaller (smaller rooms). Figure~\ref{fig:errors_grid} (right) shows how the method scales with the number of rooms of size $5\times 5$.
Again, variant $V_3$ has the best performance, in this case by a larger 
margin than before.

% Vicenc:
%Figure~\ref{fig:errors_grid} (left) shows results for $3\times 3$ rooms of size $5\times 5$ and Figure~\ref{fig:errors_grid} (center) shows results for $5\times 5$ rooms of size $3\times 3$. Both scenarios have $225$ interior states.
%The difference between variants $V_1$, $V_2$ and $V_3$ is more pronounced in the second case, when the number of subtasks increases (more rooms) and the partition for each subtask is smaller (smaller rooms). Figure~\ref{fig:errors_grid} (right) shows results for a larger number of rooms of size $5\times 5$. The variant $V_3$ has the best performance, in this case by a larger margin than before.



% Variant $V_3$, updating the exits of all equivalent rooms, converges the fastest, and the difference increases as a function of the number of rooms.

%We run several experiments with different grid sizes. One one side, we kept the room dimensions fixed to $5 \times 5$ to analyze the impact of the number of rooms in two grids of $3 \times 3$ ($246$ states) and $8 \times 8$ ($1696$ states). Alternatively, we reduced the size of individual rooms to $3\times3$ and increased the number of rooms to $5 \times 5$ ($270$ states). The first and the last instances have the same number ($225$) of interior states. This allows us to have a better understanding of different  hierarchical structures.

%In each problem, we run the three variants of the model-free algorithm ($V_1$, $V_2$, $V_3$) and Z-learning without hierarchical decomposition (Z-IS). 


\subsection{Taxi Domain.}
%Since LMDPs have no explicit actions, we adapted the Taxi domain as follows.
To allow comparison between all the methods, we adapted the Taxi domain as follows: when the taxi is at the correct pickup location, it can transition to a state with the passenger in the taxi.
In a wrong pickup location, it can instead transition to a terminal state with large negative reward (simulating an unsuccessful pick-up).
When the passenger is in the taxi, it can be dropped off at any pickup location, successfully completing the task whenever dropped at the correct destination.

\begin{figure}[!h]
\centering
\includegraphics[width=0.49\textwidth]{figures/chapter1/taxi_5.png}
\includegraphics[width=0.49\textwidth]{figures/chapter1/taxi_10.png}
\caption{Results for $5 \times 5$ (left) and $10 \times 10$ (right) grids of Taxi domain.}
\label{fig:hlmdps_errors_taxi}
\end{figure}

%\begin{figure}[H]
%\centering
%\includegraphics[scale=0.3]{Figures/taxi_dom_VF.png}
%\caption{$\hat v(s)$ in a $5 \times 5$ Taxi, passenger (\textit{P}) and destination (\textit{D}).}
%\label{fig:taxi_VF}
%\end{figure}

%the pickup corner is an exit state connected to the partition in which the passenger is on the taxi (and, thus, can be moved to the destination) while the other corners are non-goal terminal states. This way, the agent can drop the passenger at any corners, but it will only successfully terminate the episode in the specified destination. If we disentangle the state space we can represent the value function graphically as in figure \ref{fig:taxi_VF}. In this domain, subtasks are to optimally go to the corners.
%\vspace{-0.1em}
Figure~\ref{fig:hlmdps_errors_taxi} shows results in two instances of size $5 \times 5$ ($408$ states) and $10 \times 10$ ($1608$ states). %, respectively.
Again, the proposed hierarchical approach outperforms Z-IS and $Q_o$.
In this case, the difference between $V_1$, $V_2$ and $V_3$ is less pronounced, even when the grid size increases. One possible explanation is the small number of exit states in this problem.

%Figures \ref{fig:errors_grid} and \ref{fig:errors_taxi} show the Mean Absolute Error (MAE) measured on $\hat v(s)$ against the optimal value function $v^*(s)$ previously computed using the power iteration method. Clearly, our approach outperforms the flat version of Z-learning. Regarding the gridworld domain (figure \ref{fig:errors_grid}) it is observed that the three variants of our algorithm perform similarly for the case of the $3\times3$ grid. However, when the number of rooms is increased ($5\times5$ and $8\times8$ ) and thus, the hierarchical structure can be better exploited, each of the more sophisticated approaches improves the previous one. As for the Taxi domain (figure \ref{fig:errors_taxi}) the three versions converge much faster than flat learning, but yield similar results. In this case, the structure of the partitions (most of them have a single exit state) is the reason that further assumptions do not necessarily improve the performance as the three variants comprise the same effects.

\section{Discussion and Conclusion}

In this paper we have introduced a novel approach to hierarchical reinforcement learning that focuses on the class of linearly-solvable Markov decision processes.
Using subtask compositionality, we can decompose the value function and derive algorithms that converge to the optimal value function.
To the best of our knowledge, our approach is the first to exploit both the concurrent compositionality enabled by LMDPs together with hierarchies and intra-task learning to obtain globally optimal policies efficiently.

The proposed hierarchical decomposition leads to a new form of zero-shot learning that allows to incorporate subtasks that belong to an existing equivalent class without additional learning effort.
For example, adding new rooms in our example.
This is in contrast with existing methods that only exploit linear compositionality of tasks.

Our approach is limited to OR compositionality of subtasks, but there is no fundamental limitation that prevents arbitrary compositions.
The benefits of hierarchies can be combined for example, with the extended value functions proposed in~\textit{NangueTasse2020}.
\chapter{Hierarchical Average-reward Linearly-solvable Markov Decision Pocesses}
\section{Introduction}

Most previous work on hierarchical RL considers either the finite-horizon setting or the infinite-horizon setting with discounted rewards.
The average-reward setting is better suited for cyclical tasks characterized by continuous experience.
In the few works on hierarchical RL in the average-reward setting, either the low-level tasks are assumed to be solved beforehand~\citep{Fruit2017,Fruit2017b,Wan2021a} or they % low-level tasks 
have important restrictions that severely reduce their applicability, e.g.~a single initial state~\citep{Ghavamzadeh2007}. It is therefore an open question how to develop algorithms for hierarchical RL in the average-reward setting
%with the average-reward criterion
in order to learn the low-level and high-level tasks simultaneously.



This chapter describes a novel framework for for hierarchical RL in the average-reward setting that simultaneously solves low-level and high-level tasks. Concretely, considering the class of Linearly-solvable Markov Decision Processes (LMDPs)~\citep{Todorov2006}. The method here developed is an extension of the episodic case, described in the previous chapter, to the average-reward setting. Even though the compositionality property for LMDPs or methods of similar flavor have been proposed in RL~\cite{Hunt2019,Niekerk2019,NangueTasse2020} and in combination with Hierarchical RL in the finite-horizon setting~\cite{Jonsson2016,Saxe2017,Infante2022}, adapting this idea to the average-reward setting requires careful analysis and pose a new challenge.



%% VICEN
% LMDPs are a class of restricted MDPs for which the Bellman equation can be exactly transformed into a linear equation. This class of problems plays a key role in the framework of RL as probabilistic inference~\cite{Kappen2012,Levine2018}.
% One of the properties of LMDPs is compositionality: one can compute the solution to a novel task from the solutions to previously solved tasks without learning~\citep{Todorov2009a}. 
%In addition to many other properties, for LMDPs it is possible to compose a solution to a novel task from the solutions to previously solved task without learning~\citep{Todorov2009a}.


%%% PREVIOUS
% Since the Bellman optimality equations are linear for LMDPs, it is possible to compose a solution to a novel task from the solutions to previously solved task without learning~\citep{Todorov2009a}. Compositionality has been previously exploited for hierarchical reinforcement learning for LMDPs in the finite-horizon setting~\citep{Infante2022}, but adapting the idea to the average-reward setting requires careful analysis.

Similarly to the first-exit case and unlike most frameworks for hierarchical RL, this approach does not decompose the policy, only the value function. Hence the agent never chooses a subtask to solve, and instead uses the subtasks to compose the value function of the high-level task. 
This avoids introducing non-stationarity at the higher level when updating the low-level policies.

\section{Contributions}
%This has the advantage of not causing non-stationarity at the higher level when updating the low-level policies. To the best of our knowledge, our work makes the following novel contributions:
\begin{itemize}
  \item Learning low-level and high-level tasks simultaneously in the average-reward setting, without imposing additional restrictions on the low-level tasks.
    \item Two novel algorithms for solving hierarchical RL in the average reward setting: the first one is based on the eigenvector approach used for solving LMDPs. The second is an online variant in which an agent learns simultaneously the low-level and high-level tasks.
%  \item Representing low-level tasks as finite-horizon rather than average-reward decision processes.
    \item Two main theoretical contributions LMDPs: convergence proofs for both differential soft TD-learning for (non-hierarchical) LMDPs and also for the eigenvector approach in the hierarchical case.
%  \item Proving the convergence of average-reward RL in non-hierarchical LMDPs.
  %\item Applying compositionality for reinforcement learning in the average-reward setting.
\end{itemize}

This work is the first that extends the combination of compositionality and hierarchical RL to the average-reward setting.

\section{Related work}
Most research on hierarchical RL formulates problems as a Semi-Markov Decision Process (SMDP) with options~\citep{Sutton1999} or the MAXQ decomposition~\citep{Dietterich2000}.

\citet{Fruit2017} and~\cite{Fruit2017b} propose algorithms for solving SMDPs with options in the average-reward setting, proving that the regret of their algorithms is polynomial in the size of the SMDP components, which may be smaller than the components of the underlying Markov Decision Process (MDP).
\citet{Wan2021a} present a version of differential Q-learning for SMDPs with options in the average-reward setting, proving that differential Q-learning converges to the optimal policy.
However, the above work assumes that the option policies are given prior to learning.
\citet{Ghavamzadeh2007} propose a framework for hierarchical average-reward RL based on the MAXQ decomposition, in which low-level tasks are also modeled as average-reward decision processes.
However, since the distribution over initial states can change as the high-level policy changes, the authors restrict low-level tasks to have a single initial state.

\citet{Infante2022} combine the compositionality of LMDPs with the equivalence of low-level tasks to develop a framework for hierarchical RL in the finite-horizon setting.
In contrast, our framework based on LMDPs can represent the globally optimal policy.

\section{Alternative method for solving an ALMDP}

An alternative method for solving an ALMDP $\cL$ is to transform it to a first-exit LMDP. 
Given a reference state $s^*$ and an original ALMDP $\cL=\langle\cS,\kernel,\cR\rangle$, define a first-exit LMDP $\cL'=\langle\cS\setminus\{s^*\},\{s^*\}, \kernel',\cR',\cJ'\rangle$, where $\kernel'(s'|s)=\kernel(s'|s)$ for all state pairs $(s,s')\in(\cS\setminus\{s^*\})\times\cS$, and $\cJ(s^*)=0$ (implying $z(s^*)=1$). By inspection of~\eqref{eq:boe_z_lmdp} and~\eqref{eq:boe_z_almdp}, the Bellman optimality equation of $\cL'$ is identical to that of $\cL$ if $\cR'(s)=\cR(s)-\rho$. Even though the agent has no prior knowledge of the exponentiated gain $\Gamma = e^{\eta\rho}$, binary search is performed to find $\Gamma$. For a given estimate $\hat\Gamma$ of $\Gamma$, after solving $\cL'$, $\hat \Gamma z(s^*)$ is compared to $e^{\eta\cR(s^*)}\sum_s\kernel(s|s^*)z(s)$. If $\hat \Gamma z(s^*)$ is greater, then $\hat\Gamma$ is too large, else it is too small.

Alternatively, when $\kernel$ and $\cR$ are not known, an estimate $\hat v$ of the optimal value $v$ and an estimate $\hat\rho$ of the optimal gain $\rho$ are kept using \textit{differential soft TD-learning}, similar to differential Q-learning~\citep{Wan2021}. Samples $(s_t, r_t, s_{t+1})$ generated by the estimated policy $\hat\pi$ derived from $\hat v$ as in \eqref{eq:lmdp_optimal_policy} are collected, and the update rules for $\hat v$ and $\hat \rho$ from \eqref{eq:boe_almdp} are as follows:
\begin{align}
    \hat{v}_{t+1}(s_t) &\gets \hat{v}_t(s_t) + \alpha_t \delta_t,\label{eq:main_v_td_update}\\
    \hat{\rho}_{t+1} &\gets \hat{\rho}_t + \lambda \alpha_t \delta_t.\label{eq:main_rho_td_update}
    \end{align}
Here, the TD error $\delta_t$ is given by
\begin{align*}
\delta_t &= r_t - \hat{\rho}_t - \frac 1 \eta \log \frac {\hat{\pi}_t(s_{t+1}|s_t)} {\kernel(s_{t+1}|s_t)} + \hat{v}_t(s_{t+1}) - \hat{v}_t(s_t)\\
 &= r_t - \hat{\rho}_t + \frac 1 \eta \log \sum_{s'\in\cS} \kernel(s'|s_t) e^{\eta \hat{v}_t(s')} - \hat{v}_t(s_t).
\end{align*}
% The learning rates $\alpha_t$ and $\beta_t$ can be chosen independently.
Note that both updates use the same TD error. At any time, the estimates of $\hat z$ and $\widehat\Gamma$ can be retrieved by exponentiating $\hat v$ and $\hat \rho$, respectively.


\begin{theorem}
    Under mild assumptions, differential soft TD-learning in~\eqref{eq:main_v_td_update} and~\eqref{eq:main_rho_td_update} converges to the optimal values of $v$ and $\rho$ in $\cL$. \label{theo:almdps}
\end{theorem}

\begin{proof}[Proof sketch]
The proof is adapted from the proof of convergence of differential Q-learning~\cite{Abounadi2001,Wan2021}, which requires the ALMDP to be communicating (Assumption~\ref{ass:communicating}).
Define a Bellman operator $T$ as
\[
T(v)(s) = \cR(s) + \frac 1 \eta \log \sum_{s'\in\cS} \kernel(s'|s) e^{ \eta v(s') }.
\]
To adapt the previous proof, it is sufficient to show that $T$ is a non-expansion in the max norm, i.e.~$\infnorm{T(x)-T(y)} \leq \infnorm{x - y}$ for each $x, y\in\real^{|\cS|}$, and that $T$ satisfies $T(x + c\mathds{1}) = T(x) + c\mathds{1}$ for each $x\in\real^{|\cS|}$ and constant $c\in\real$, where $\mathds{1}$ is $|\cS|$-dimensional vector of all ones.
For completeness, the full proof appears in Appendix~\ref{proof:theo_almdps}.
\end{proof}

% Simultaneously, we need to keep an estimate $\widehat\Gamma$ of the exponentiated gain $\Gamma$. We can derive update rules for $\hat z$ and $\widehat \Gamma$ from \eqref{eq:boe_z_almdp} as follows:
% \begin{align}
%   \widehat{z}_{t+1}(s_t)  \gets \widehat{z}_{t}(s_{t}) & + \alpha_t \Bigg(\frac{e^{\eta r_t}}{\widehat\Gamma_t}\dfrac{\cP(s_{t+1}\lvert s_t)}{\widehat\pi_t(s_{t+1}\lvert s_t)}  - \hat z_t(s_t)\Bigg),\label{eq:update_z_e}  \\   %\delta^z_t \\
%   \widehat\Gamma_{t+1}    \gets \widehat\Gamma_t       & + \beta_t \Bigg(\frac{e^{\eta r_t}}{\hat z_t(s_t)}\dfrac{\cP(s_{t+1}\lvert s_t)}{\widehat\pi_t(s_{t+1}\lvert s_t)}  - \widehat \Gamma_t\Bigg).\label{eq:update_gamma}
% \end{align}
% The learning rates $\alpha_t$ and $\beta_t$ can be chosen independently.

\begin{comment}
We also notice that we can perform the updates in the logarithmic space. In this case, the update rules are given by
\begin{align}
  \frac{1}{\eta}\log{\hat z_{t+1}(s_t)}    & \gets \frac{1}{\eta}\log{\hat z_t(s_t)} + \alpha_t \delta_t     \\
  \frac{1}{\eta}\log{\widehat\Gamma_{t+1}} & \gets \frac{1}{\eta}\log{\widehat \Gamma_t} + \beta_t \delta_t.
\end{align}
The TD error $\delta_t$ is given by
\begin{align*}
  \delta_t & = \Big(r_t - \frac{1}{\eta}\log{\widehat\Gamma_t} -\frac{1}{\eta}\log{\frac{\hat\pi(s_{t+1}\lvert s_t)}{\cP(s_{t+1}\lvert s_t)}} + \frac{1}{\eta} \log{\hat z_t(s_{t+1})} - \frac{1}{\eta} \log{\hat z_t(s_{t})}\Big) \\
           & = \Big(r_t- \frac{1}{\eta}\log{\widehat\Gamma_t} + \frac{1}{\eta}\log{G\left[\hat z_t\right](s)} - \frac{1}{\eta} \log{\hat z_t(s_{t})}\Big).
\end{align*}
After exponentiation, the update rules can be rewritten as
\begin{align}
  \hat z_{t+1}(s_t)    & \gets \hat z_t(s_t)e^{\eta\alpha_t\delta_t} =  \hat z_t(s_t)^{1-\alpha_t} \Bigg(\frac{e^{\eta r_t}}{\widehat \Gamma_t}G\left[\hat z_t\right](s_t) \Bigg)^{\alpha_t}  \\
  \widehat\Gamma_{t+1} & \gets \widehat\Gamma_t e^{\eta\beta_t\delta_t} =  \widehat\Gamma_t^{1-\beta_t} \Bigg(e^{\eta r_t}\frac{G\left[\hat z_t\right](s_t)}{\hat z_t(s_t)} \Bigg)^{\beta_t}.
  \label{eq:td_gamma}
\end{align}
{\color{blue} A full derivation of these updates rules can be found in Appendix.} \todo{Should we draw a comparison between the two update rules?}
\end{comment}


\begin{figure*}
  \begin{center}
  \begin{adjustbox}{width=0.4\textwidth}
    \begin{tikzpicture}
      \draw[step=0.4,thin,shift={(0.2,0.2)}] (0.8,0.8) grid (4.8,4.8);
      \draw[ultra thick] (1,1) rectangle (5,5);
      \draw[ultra thick] (3,1) -- (3,1.8);
      \draw[ultra thick] (3,2.2) -- (3,3.8);
      \draw[ultra thick] (3,4.2) -- (3,5);
      \draw[ultra thick] (1,3) -- (1.8,3);
      \draw[ultra thick] (2.2,3) -- (3.8,3);
      \draw[ultra thick] (4.2,3) -- (5,3);

      \draw[ultra thick] (4.2,4.2) rectangle (4.6,4.6);
      \draw[ultra thick] (3.8,2.6) rectangle (4.2,3.4);
      \draw[ultra thick] (1.8,2.6) rectangle (2.2,3.4);
      \draw[ultra thick] (2.6,3.8) rectangle (3.4,4.2);
      \draw[ultra thick] (2.6,1.8) rectangle (3.4,2.2);

      \node (R) at (1.2,4.8) {} ;
      \node (G) at (4.4,4.4) {\tiny $G$};
      \node at (2,3.2) {\tiny $3^T$};
      \node at (2,2.8) {\tiny $1^B$};
      \node at (4,3.2) {\tiny $4^T$};
      \node at (4,2.8) {\tiny $2^B$};
      \node at (2.8,4) {\tiny $2^L$};
      \node at (2.8,2) {\tiny $4^L$};
      \node at (3.2,4) {\tiny $1^R$};
      \node at (3.2,2) {\tiny $3^R$};

    %   \draw[step=0.4,thin,shift={(0.2,0)}] (8.799,1.999) grid (10.8,4);
    %   \draw[ultra thick] (9,3.2) -- (8.6,3.2) -- (8.6,2.8) -- (9,2.8) -- (9,2) -- (9.8,2);
    %   \draw[ultra thick] (9,3.2) -- (9,4) -- (9.8,4) -- (9.8,4.4) -- (10.2,4.4) -- (10.2,4);
    %   \draw[ultra thick] (10.2,4) -- (11,4) -- (11,3.2) -- (11.4,3.2) -- (11.4,2.8) -- (11,2.8);
    %   \draw[ultra thick] (9.8,2) -- (9.8,1.6) -- (10.2,1.6) -- (10.2,2) -- (11,2) -- (11,2.8);
    %   \draw[ultra thick] (10.2,3.2) rectangle (10.6,3.6);

    %   \node at (10.4,3.4) {\small $G$};
    %   \node at (8.8,3)    {\small $L$};
    %   \node at (11.2,3)   {\small $R$};
    %   \node at (10,1.8)   {\small $B$};
    %   \node at (10,4.2)   {\small $T$};

    %   \node at (3,0.7) {\Large a)};
    %   \node at (10,0.7) {\Large b)};

      \draw [->] (G.north) to [out=100,in=70] (R.center);
    \end{tikzpicture}
  \end{adjustbox}
  \end{center}
  \caption{a) An example $4$-room ALMDP; b) a single subtask with 5 terminal states $G,L,R,T,B$ that is equivalent to all 4 room subtasks. Rooms are numbered 1 through 4, left-to-right, then top-to-bottom, and exit state $1^B$ refers to the exit $B$ of room $1$, etc. \\}
  \label{fig:ex}
\end{figure*}

\section{Hierarchical Average-Reward LMDPs}

This section presents the approach for hierarchical average-reward LMDPs. The idea is to take advantage of the similarity of the value functions in the first-exit and average-reward settings, and use compositionality to compose the value functions of the subtask LMDPs without additional learning.

%\subsubsection{Hierarchical Decomposition}
\subsection{Hierarchical Decomposition}
Consider an ALMDP $\langle\cS,\kernel,\cR\rangle$. 
Similarly to~\citet{Infante2022}, the state space $\cS$ is assumed to be partitioned into subsets $\left\{\cS_i\right\}^L_{i=1}$, with each partition $\cS_i$ inducing
a first-exit LMDP $\cL_i = \langle\cS_i, \cT_i, \kernel_i, \cR_i,\cJ_i\rangle$.
The components of each such subtask $\cL_i$ are defined as follows:
\begin{itemize}
  \item The set of states is $\cS_i$.
  \item The set of terminal states $\cT_i=\{\tau\in\cS\setminus\cS_i: \exists s\in\cS_i, \kernel(\tau|s)>0\}$ contains states not in $\cS_i$ that are reachable in one step from any state inside the partition.
  \item The transition function $\kernel_i$ and reward function $\cR_i$ are projections of $\kernel$ and $\cR-\hat\rho$ onto $\cS_i$, where $\hat\rho$ is a gain estimate.
  \item $\cJ_i$ is defined for each $\tau\in\cT_i$ as $\cJ_i(\tau)=\hat v(\tau)$, where $\hat v$ is a current value estimate (hence $z_i(\tau)=e^{\eta\hat v(\tau)} = \hat z(\tau)$ is defined by a current exponentiated value estimate $\hat z$).
\end{itemize}
The Bellman optimality equations of each subtask $\cL_i$ are given by
\begin{equation}\label{eq:subtask}
  z_i(s) = e^{\eta \cR_i(s)} \sum_{s'} \kernel_i(s'|s) z_i(s') \;\; \forall s\in\cS_i.
\end{equation}
By inspection of the Bellman optimality equations in~\eqref{eq:boe_z_almdp} and~\eqref{eq:subtask}, they are equal if $\cR_i(s)=\cR(s)-\rho$. Thus, if $z_i(\tau)=z(\tau)$ for each $\tau\in\cT_i$ then the solution of the subtask $\cL_i$ corresponds to the optimal solution for each $s\in\cS_i$. However, in general neither $\rho$ nor $z(\tau)$ are known prior to learning and, therefore, estimates $\hat\rho$ and $\hat z(\tau)$ must be used instead. Each subtask $\cL_i$ can be seen as being {\it parameterized\/} on the value estimates $\hat z(\tau)$ for each $\tau\in\cT_j$ and the gain estimate $\hat\rho$. Every time that $\hat z(\tau)$, $\tau\in\cT_i$, and $\hat\rho$ change, a new value estimate for each $s\in\cS_i$ is obtained by solving the subtask for the new parameters.

%\subsection{Subtask compositionality}
\subsection{Subtask Compositionality}
It is impractical to solve each subtask $\cL_i$ every time the estimate $\hat z(\tau)$ changes for $\tau\in\cT_j$. In order to alleviate this, compositionality for LMDPs can be leveraged. The key insight is to build a basis of value functions that can be combined to obtain the solution for the subtasks.

Consider a subtask $\cL_i=\langle\cS_i,\cT_i,\kernel_i,\cR_i,\cJ_i\rangle$ and let $n=|\cT_i|$.  Let $\{\cL_i^1,\ldots,\cL_i^n\}$ be $n$ base LMDPs that are first-exit LMDPs and terminate in $\cT_i$. These base LMDPs only differ from $\cL_i$ in the reward of each terminal state $\tau^k\in\cT_i$. For all $s\in\cS_i$, the reward for each $\cL_i^k$ is by definition $\cR_i(s)=\cR(s)-\hat\rho$ for all $s\in\cS_i$, while at terminal states $\tau\in\cT_i$ let the reward function is $z_i^k(\tau;\hat\rho)=1$ if $\tau=\tau^k$ and $z_i^k(\tau;\hat\rho)=0$ otherwise. Thus, the base LMDPs are parameterized by the gain estimate $\hat\rho$. This is equivalent to setting the reward to $\cJ_i^k(\tau)=0$ if $\tau=\tau_k$ and $\cJ_i^k(\tau)=-\infty$ otherwise. Intuitively, each base LMDP solves the subtask of reaching one specific terminal state $\tau_k\in\cT_i$.

Assume that the solution $z_i^1(\cdot;\rho),\ldots,z_i^n(\cdot;\rho)$ for the base-LMDPs (for the optimal gain $\rho$) is available as well as the optimal value $z(\tau^k)$ of the original ALMDP for each terminal state $\tau^k\in\cT_i$. Then by compositionality could represent the value function of each terminal state can be represented as a weighted combination of the subtasks:
\begin{equation}
  z(\tau) = \sum_{k=1}^n w_k z_i^k(\tau;\rho) =  \sum_{k=1}^n z(\tau^k) z_i^k(\tau;\rho) \;\;\forall\tau\in\cT_i.
  \label{eq:comp_terminal}
\end{equation}

Clearly, the RHS in the previous expression evaluates to $z(\tau)$ since $z(\tau^k) z_i^k(\tau;\rho) = z(\tau)\cdot 1$ when $\tau = \tau^k$, and
$z(\tau^k) z_i^k(\tau;\rho) = z(\tau^k)\cdot 0$ otherwise.

Thanks to compositionality, it is also possible to represent the value function for each subtask state $s\in\cS_i$ as
\begin{equation}
  z(s) = \sum_{k=1}^n z(\tau^k) z_i^k(s;\rho)\;\;\forall s\in\cS_i.
  \label{eq:comp_internal}
\end{equation}
It is signficant to remark that the base LMDPs depend on the gain $\rho$ by the definition of the reward function. This parameter is not known prior to learning. The subtasks in practice are solved for the latest estimate $\hat\rho$ and must be re-learned for every update of this parameter until convergence.
\subsection{Efficiency of the value representation}
 Similar to previous work~\citep{Wen2020,Infante2022} the equivalence of subtasks is exploited to learn more efficiently. Let $\cC=\{\cC_1,\ldots,\cC_C\}$, $C\leq L$, be a set of equivalence classes, i.e.~a partition of the set of subtasks $\{\cL_1,\ldots,\cL_L\}$ such that all subtasks in a given partition are equivalent.
As before, a set of exit states as $\cE=\cup_{i=1}^L\cT_i$ is also defined.
Due to the decomposition, there is no need to keep an explicit value estimate $\hat z(s)$ for every state $s\in\cS$. Instead, it is sufficient to keep a value function for exit states $\hat z_\cE: \cE\rightarrow\real$ and a value function for each base LMDP of each equivalence class. This is enough to represent the value for any state $s\in\cS$ using the compositionality expression in~\eqref{eq:comp_internal}.

Letting $K=\max_{i=1} ^L\lvert\cS_i\rvert$,  $N=\max_{i=1} ^L\lvert\cT_i\rvert$ and $E=\lvert\cE\rvert$, $O(KN)$ values are needed to represent the base LMDPs of a subtask, and the value function can be represented with $O(CKN + E)$ values. The decomposition leads to an efficient representation of the value function whenever $CKN + E \ll \lvert\cS\rvert$. This is achieved when there are few equivalence classes, the size of each subtask is small (in terms of the number of states) and there are relatively few exit states.

  % {\bf Example}: Consider the $4$-room example depicted in Figure~\ref{fig:ex}. Here there is a single equivalence class that generalizes all the rooms. Therefore, a single subclass is represented with $5\times 5$ states. This subtask induces $5$ base-LMDPs with terminal states in $G,L,R,T,B$. Besides, there is a total of 

  %{\bf Example 1:} 
  \emph{Example 1}:
  Figure~\ref{fig:ex}a) shows an example 4-room ALMDP. When reaching the state marked $G$, separate from the room but reachable in one step from the highlighted location, the system transitions to a restart state (top left corner) and receives reward $0$. In all other states the reward is $-1$. The rooms are connected via doorways, so the subtask corresponding to each room has two terminal states in other rooms, plus the terminal state $G$ in the top right room. The 9 exit states in $\cE$ are highlighted and correspond to states next to doorways, plus $G$. Figure~\ref{fig:ex}b) shows a single subtask that is equivalent to all four room subtasks, since the dynamics is shared inside rooms and the set of terminal states is the same. % as those of the subtasks.
  %Hence the number of equivalent subtasks is $C=1$, the number of non-terminal and terminal states of subtasks is $K=25$ and $N=5$, respectively, and the number of exit states is $E=9$.
    %In the 4-room example, t
    There are five base LMDPs with value functions $z^G$, $z^L$, $z^R$, $z^T$ and $z^B$, respectively. Given an initial value estimate $\hat{z}_\cE$ for each exit state in $\cE$, a value estimate of any state in the top left room is given by $\hat{z}(s)=\hat{z}_\cE(1^B) z^B(s) + \hat{z}_\cE(1^R) z^R(s)$, where $\hat{z}_\cE(G)=\hat{z}_\cE(L)=\hat{z}_\cE(T)=0$ is used to indicate that the terminal states $G$, $L$ and $T$ are not reachable in the top left room. 
    The total number of values needed to store the optimal value function is $E+CKN=9+125=134$, and the base LMDPs are faster to learn since they have smaller state space.
    %We need $CKN = 125$ values to store the value functions of the 5 base LMDPs, and $E=9$ values to store the value estimates of all exit states. Although this is more than the 100 states of the original LMDP, if we increase the number of rooms to $X\times Y$, the term $CKN$ is a constant as long as all rooms have equivalent dynamics, and the number of exit states is $E=(2X-1)(2Y-1)$, which is much smaller than the $25XY$ total states. For $10\times 10$ rooms, the value function decomposition requires $486$ values to represent the values of $2{,}500$ states.


\section{Algorithms}
In this section two novel algorithms for solving hierarchical ALMDPs are described. The first is a two-stage eigenvector approach that relies on first solving the subtasks. The second is an online algorithm in which an agent simultaneously learns the subtasks, the gain and the exit values from samples $(s_t, r_t, s_{t+1})$.
Once again it is important to remark that the values for states $s\notin\cE$ are not explicitly represented.

\subsection{Eigenvector approach}

In the episodic case, the base LMDPs are only solved once, and the solutions are then reused to compute the value function $z_\cE$ on exit states (see section~\ref{section:hlmdps_eigenvector_episodic}). However, in the case of ALMDPs, the reward functions of base LMDPs depend on the current gain estimate $\hat\rho$, which is initially unknown. 

% I think the duration part is false!!!

%If we could estimate the expected {\em duration} $d_i(s)$ of each subtask $\cL_i$ from each state $s\in\cS_i$, we could use the duration to adjust the value estimate $\hat z(s)$ of $s$ according to the current gain estimate $\widehat\Gamma$ by a factor $\widehat \Gamma^{d_i(s)}$. The duration can be recursively defined as $d_i(\tau)=0$ for each $\tau\in\cT_i$ and
%\[
%d_i(s) = 1 + \sum_{s'} \pi_i(s'|s) d_i(s').
%\]
%However, even though the subtask policy $\pi_i$ can be expressed in terms of the base LMDP policies $\pi_i^k$, we have been unable to formulate $d_i$ as a composition of the durations $d_i^k$ of base LMDPs, and hence the duration would have to be recomputed each time we resolve a subtask $\cL_i$, which is inefficient.

%We describe in Algorithm~\ref{alg:halmdps_eigenvector}. 
The eigenvector approach for solving hierarchical ALMDPs appears in Algorithm~\ref{alg:halmdps_eigenvector}. The intuition is that in each iteration, first the subtasks are solved for the latest estimate of the exponentiated gain $\widehat\Gamma$. For this, the base LMDPs are solved with~\eqref{eq:subtask} using with the current value of $\hat\rho$. Then~\eqref{eq:comp_internal} is applied, restricted to $\cE$ to obtain an estimate of the value for the exit states. This yields the system of linear equations
\begin{equation}
    {\bf z_\cE} = G_\cE {\bf z_\cE}.\label{eq:eigenvector}
\end{equation}
Here, the matrix $G_\cE\in\real^{\lvert\cE\rvert\times \lvert\cE\rvert}$ contains the optimal values of the base LMDPs and has elements defined as in~\eqref{eq:comp_internal}. The previously introduced idea to transform the ALMDP $\cL$ to a first-exit LMDP $\cL'$ parameterized on the estimated gain $\hat\rho$ is used, and  the optimal exponentiated gain $\Gamma$ is found using binary search. There is a reference state $s^*\in\cS$ (which is by definition an exit state) and is used the test described above to decide how to update the search interval. 
\begin{algorithm}[!b]
  \caption{Eigenvector approach to solving a hierarchical ALMDP.}
  \begin{algorithmic}[1]
    %\Require{An ALMDP $\cL=\langle\cS,\cP,\cR\rangle$, a partition $\cS_1,\ldots,\cS_L$ of $\cS$, inducing a set of exit states $\cE$, reference state $s^\star$.}

    %\Procedure{\textsc{SolveALMDP}}{$\cL,\cS_1,\ldots,\cS_L,\cE,\epsilon,\eta$}
    \State{{\bf Input:} $\cL,\cS_1,\ldots,\cS_L,\cE,\epsilon,\eta$}
    \State $\text{lo}\gets 0$, $\text{hi}\gets 1$
    \While {$\text{hi} - \text{lo} > \epsilon$}
    \State $\widehat\Gamma \gets (\text{hi} + \text{lo}) \mathbin{/} 2$
    %\For{each subtask $\cL_i$}
    %\State Form the matrix $G_i = G_j(\eta,\widehat{\Gamma}_k)$
    %\State Solve the Bellman optimality equation $\widehat{\bf z}_i = G_i\widehat{\bf z}_i^+$
    %\EndFor
    \State Solve base LMDPs $\cL_j^1,\ldots,\cL_j^n$ for each equivalence class $\cC_j$
    \State Form the matrix $G_\cE$ from the optimal value functions
    \State Solve the system of equations  ${\bf \hat z_\cE} = G_\cE {\bf\hat z_\cE}$
    \If {$ \widehat\Gamma \hat z_\cE(s^*) > e^{\eta \cR(s^*)} \sum_{s\in\cS} \kernel(s\lvert s^*) \hat z_\cE(s)$}
    \State $\text{hi}\gets \widehat\Gamma$
    \Else \State $\text{lo}\gets \widehat\Gamma$
    \EndIf
    \vspace*{3pt}
    \EndWhile
    \Return value functions of all base LMDPs, ${\bf \hat z_\cE}$
    %\EndProcedure
  \end{algorithmic}
  \label{alg:halmdps_eigenvector}
\end{algorithm}

\begin{theorem}\label{thm:converge}
    Algorithm~\ref{alg:halmdps_eigenvector} converges to the optimal value function $z$ of $\cL$ as $\epsilon\to 0$.
\end{theorem}
% The proof of Theorem~\ref{thm:converge} appears in Appendix~\ref{proof:theo_h}.

First note that the optimal value function $z$ of $\cL$ exists and is unique due to Assumption~\ref{ass:communicating}. Due to the equivalence between $\cL$ and the corresponding first-exit LMDP $\cL'$, this implies that $\cL'$ has a unique solution $z(\cdot;\rho)$ when the estimated gain $\hat\rho$ equals $\rho$, and that this solution equals $z(\cdot;\rho)=z$, the optimal solution to $\cL$.

\begin{lemma}\label{lemma:monotonicity}
     Given a first-exit LMDP $\cL'$ parameterized on $\hat\rho$, the optimal value $z(s;\hat\rho)$ of each non-terminal state $s\in\cS$ is strictly monotonically decreasing in $\hat\rho$.
\end{lemma}

\begin{proof}
Strict monotonicity requires that there exists $\varepsilon>0$ such that $\linebreak{z(s;\hat\rho - \varepsilon) > z(s;\hat\rho) > z(s;\hat\rho+\varepsilon)}$ when $\varepsilon\rightarrow 0$. The first inequality is proven by induction; the second is analogous. The base case is given by the terminal states $\tau\in\cT$, for which $z(\tau;\hat\rho - \varepsilon) = z(\tau;\hat\rho)$. The inductive case is given by
\begin{align*}
    z(s;\hat\rho-\varepsilon) &= e^{\eta(\cR(s) - (\hat\rho -\varepsilon))}\sum_{s'\in\cS}\kernel(s'\lvert s) z(s';\hat\rho- \varepsilon)\\
      &\geq e^{\eta\varepsilon} e^{\eta(\cR(s) - \rho))}\sum_{s'\in\cS}\kernel(s'\lvert s) z(s';\hat\rho)\\
      &= e^{\eta\varepsilon} z(s;\hat\rho) > z(s;\hat\rho).
\end{align*}
This concludes the proof.
\end{proof}

As a consequence of Lemma~\ref{lemma:monotonicity}, $\cL'$ has a unique solution $z(\cdot,\hat\rho)$ for each $\hat\rho\geq\rho$, since the values $z(\cdot,\hat\rho)$ decrease as $\hat\rho$ increases. In contrast, there may be values of $\hat\rho>\rho$ for which power iteration does not converge.

\begin{restatable}{lemma}{optimality}
Given a subtask $\cL_i$, if the optimal value of each terminal state $\tau\in\cT_i$ equals its optimal value in $\cL$, i.e. $z_i(\tau) = z(\tau)$, and the optimal gain $\rho$ in $\cL$ is known, then the optimal value of each non-terminal state $s\in\cS_i$ is unique and equals $z_i(s)=z(s)$.
    \label{lemma:optimality}
\end{restatable}

\begin{proof} Since $\cR_i$ and $\kernel_i$ are restrictions of $\cR-\rho$ and $\kernel$, respectively, to $\cS_i$, then
\begin{align*}
    z_i(s) &= e^{\eta\cR_i(s)}\sum_{s'}\kernel_i(s'\lvert s) z_i(s') = e^{\eta(\cR(s) - \rho)}\sum_{s'}\kernel(s'\lvert s) z_i(s'), \nonumber
\end{align*}
which is the same Bellman equation as for $z(s)$. Assuming that $z_i(\tau) = z(\tau)$ for each $\tau\in\cT$, directly yields that $z_i(s)=z(s)$ for each non-terminal state $s\in\cS_i$.
\end{proof}

\begin{corollary}\label{cor:uniqueness}
    If the optimal gain $\rho$ in $\cL$ is known, each base LMDP $\cL_j^k$ has a unique solution $z_j^k(\cdot;\rho)$.
\end{corollary}

\begin{proof} From \eqref{eq:comp_internal}, the optimal values of subtask states satisfy
\[
  z(s) = \sum_{k=1}^n z(\tau^k) z_i^k(s;\rho)\;\;\forall s\in\cS_i.
\]
Due to Lemma~\ref{lemma:optimality}, the optimal value $z(s)$ is unique, which is only possible if $z_i^k(s;\rho)$ is unique for each $\tau^k\in\cT_i$.
\end{proof}

Combined with Lemma~\ref{lemma:monotonicity}, Corollary \ref{cor:uniqueness} implies that each base LMDP $\cL_j^k$ has a unique solution $z_j^k(\cdot;\hat\rho)$ whenever $\hat\rho\geq\rho$.

\begin{lemma}
For $\hat\rho\geq\rho$, the equation ${\bf \hat z_\cE} = G_\cE {\bf\hat z_\cE}$ has a unique solution that equals $z_\cE(\tau)=z(\tau;\hat\rho)$ for each exit $\tau\in\cE$, where $z(\cdot;\hat\rho)$ is the unique value of the first-exit LMDP $\cL'$ for $\hat\rho$.
\end{lemma}

\begin{proof}
At convergence, due to \eqref{eq:comp_internal} it has to hold for each non-terminal exit $\tau\in\cE$ that
\[
  z_\cE(\tau) = \sum_{k=1}^n z_\cE(\tau^k) z_i^k(s;\hat\rho)\;\;\forall s\in\cS_i,
\]
where each $\tau^k$ is also an exit and $z_i^k(s;\hat\rho)$ is well-defined and unique since $\hat\rho\geq\rho$. This equation is satisfied when $z_\cE(\tau)=z(\tau;\hat\rho)$ for each exit. Since $z(\cdot;\hat\rho)$ is unique, this is the only solution.
\end{proof}

Theorem~\ref{thm:converge} can be now proven. When $\hat\rho\geq\rho$ (or equivalently, $\widehat\Gamma\geq\Gamma$), each base LMDP has a unique solution, and $z_\cE$ is also unique. Moreover, when $\hat\rho>\rho$, the condition on line 8 is true, which causes binary search to discard all values greater than $\hat\rho$. If the base LMDPs or $z_\cE$ do not have a unique solution, it means that $\hat\rho$ is too small, and hence all values less than $\hat\rho$ can be discarded. Since the solution $z(\cdot;\hat\rho)$ is monotonically decreasing in $\hat\rho$, binary search is guaranteed to find the optimal gain $\rho$ within a factor of $\epsilon$.

\begin{figure}[!ht]
  \centering
  \includegraphics*[width=0.7\textwidth]{figures/chapter2/eigenvectors/lmdp-nroom-1.pdf}
  \caption{Results in N-room when varying the number of rooms and the size of the rooms.}
  \label{fig:nrooms}
 \end{figure}

\subsection{Online algorithm}
 In the online case (see Algorithm~\ref{alg:halmdps_online}), the agent keeps an estimate of the exponentiated gain $\widehat\Gamma=e^{\eta\hat\rho}$ which is updated every timestep. It also keeps estimates of the value functions of the base LMDPs $\hat z_i^1(\cdot;\hat\rho),\ldots,\hat z_i^n(\cdot;\hat\rho)$ for each equivalence class $\cC_i$, and estimates of the value function on exit states $\hat z_\cE$.  All the base LMDPs of the same equivalence class can be updated with the same sample using intra-task learning with the appropriate {\it importance sampling weights\/}~\citep{Jonsson2016}. For the estimates of the exit states, they are only updated upon visitation of such states. In that case, the compositionality expression in~\eqref{eq:comp_internal} is used to derive the following update:
\begin{equation}
  \hat z_\cE(s)\gets (1-\alpha_\ell) \hat z_\cE(s) + (1-\alpha_\ell) \sum_{k=1}^n \hat z_\cE(\tau^k) \hat z_i^k(s;\rho).
  \label{eq:td_update_exit}
\end{equation}
Here, $\alpha_\ell$ is the learning rate. Each of the learned components (i.e., gain, base LMDPs and exit state value estimates) maintain independent learning rates.

\begin{algorithm}[!htpb]
  \caption{Online algorithm.}
  \begin{algorithmic}[1]
    \State{{\bf Input:} $\cL,\cS_1,\ldots,\cS_L,\cE,s_0,\epsilon,\eta$}
    \State $t \gets 0$, $\widehat{\Gamma} \gets 1$
    \While {{\it not terminate}}
    \State Observe $(s_{t}, r_{t}, s_{t+1})\sim\hat\pi_t$
    \State Update $\hat z_j^1(\cdot;\hat\rho),\ldots,\hat z_j^n(\cdot;\hat\rho)$ using Equation~\eqref{eq:main_v_td_update}
    \State Compute $\hat z(s_t)$ and $\sum_{s'}\kernel(s'|s_t)\hat z(s')$ using Equation \eqref{eq:comp_internal}
    \State Update $\widehat\Gamma$ using Equation~\eqref{eq:main_rho_td_update}
    \If{$s_t\in\cE$}
    \State Update $\hat z_\cE(s_t)$ using Equation~\eqref{eq:td_update_exit}
    \EndIf
    \EndWhile
  \end{algorithmic}
  \label{alg:halmdps_online}
\end{algorithm}

\section{Experiments}

Algorithm~\ref{alg:halmdps_online} is compared against differential soft TD-learning in the flat representation of the ALMDP. Similarly to the episodic case, the metric used for the experiments Mean Absolute Error (MAE) between the estimated value function $\hat z$ and the true optimal value function $z$. For each algorithm, the results are averaged over five seeds. In addition to the average, the standard deviation is also reported. The learning rates have been optimized independently for each of the instances. The two episodic benchmark tasks~\citep{Infante2022} have been adapted by transforming them into infinite-horizon tasks as follows:

  \subsection{N-room domain}
 As in the episodic case, there are some `goal' states with high reward (i.e.~0). When the agent enters a goal state, the next action causes it to receive the given reward and transition to a {\it restart\/} location. The number of rooms as well as the size of the rooms are varied to obtain different instances of the problem.

  \begin{sidewaysfigure}
  \centering
  \includegraphics*[width=0.32\textwidth]{figures/chapter2/online/nrooms_3_3.pdf}
  \includegraphics*[width=0.32\textwidth]{figures/chapter2/online/nrooms_5_5.pdf}
  \includegraphics*[width=0.32\textwidth]{figures/chapter2/online/nrooms_8_8.pdf}

  \includegraphics*[width=0.32\textwidth]{figures/chapter2/online/nrooms_3_3_subtasks.pdf}
  \includegraphics*[width=0.32\textwidth]{figures/chapter2/online/nrooms_5_5_subtasks.pdf}
  \includegraphics*[width=0.32\textwidth]{figures/chapter2/online/nrooms_8_8_subtasks.pdf}

  \includegraphics*[width=0.32\textwidth]{figures/chapter2/online/nrooms_3_3_gammas.pdf}
  \includegraphics*[width=0.32\textwidth]{figures/chapter2/online/nrooms_5_5_gammas.pdf}
  \includegraphics*[width=0.32\textwidth]{figures/chapter2/online/nrooms_8_8_gammas.pdf}


  \caption{Results in N-room when varying the number of rooms and the size of the rooms.}
  \label{fig:halmdps_nrooms}
  \end{sidewaysfigure}
  


  \subsection{Taxi domain}
  In this variant of the original domain~\citep{Dietterich2000}, once the passenger has been dropped off, the system transitions to a state in which the driver is in the last drop-off location, but a new passenger appears randomly at another location. Hence the driver immediately receives a new assignment after having successfully completed the previous one.
  \begin{figure}[!hb]
    \begin{center}
        \includegraphics*[width=0.49\textwidth]{figures/chapter2/online/taxi_5.pdf}
        \includegraphics*[width=0.49\textwidth]{figures/chapter2/online/taxi_8.pdf}
        % \includegraphics*[width=0.49\textwidth]{figures/chapter2/online/taxi_5_subtasks.pdf}
        % \includegraphics*[width=0.49\textwidth]{figures/chapter2/online/taxi_8_subtasks.pdf}
        \caption{Results for $5 \times 5$ (top) and $8 \times 8$ (bottom) grids of the Taxi domain.}
          \label{fig:halmdps_taxi}
    \end{center}
  \end{figure}
  %In this case, the adaptation to an infinite-horizon task is more natural. Once the passenger has been dropped off, the system transitions to a state in which the driver is in the last dropoff location, but a new passenger appears at another location. %Hence the driver immediately receives a new assignment after having successfully completed the previous one.

\subsection{Results}
Figures~\ref{fig:halmdps_nrooms} and~\ref{fig:halmdps_taxi} show the results. Algorithm~\ref{alg:halmdps_online} is able to speed up 
%The results (Figures~\ref{fig:nrooms} and~\ref{fig:taxi}) show that our algorithm is able to speed up 
learning and converges to the optimal solution faster than flat average-reward reinforcement learning (note the log scale). This is in line with previous results for the episodic case~\citep{Infante2022}. The difference in the error scale in the figures is due to the initialization of the base LMDPs. The average reward setting poses an extra challenge since the `sparsity' of the reward can make the estimates of the gain oscillate. This ultimately has an impact on the estimates of the base LMDPs and the value estimates of the exit states, and it is likely the reason why in Figure~\ref{fig:halmdps_taxi} the error increases before decreasing down to zero.


% \begin{figure*}[!bt]
%   \centering
%   \includegraphics*[width=0.32\textwidth]{pictures/nrooms_3_3-1.png}
%   \includegraphics*[width=0.32\textwidth]{pictures/nrooms_5_5-1.png}
%   \includegraphics*[width=0.32\textwidth]{pictures/nrooms_8_8-1.png}
%   \caption{ Results in N-room when varying the number of rooms and the size of the rooms.}
%   \label{fig:nrooms}
% \end{figure*}

% \begin{figure}[!htp]
%   \centering
%   \includegraphics*[width=0.33\textwidth]{pictures/taxi_5-1.png}
%   \includegraphics*[width=0.33\textwidth]{pictures/taxi_10-1.png}
%   \caption{Results for $5 \times 5$ (left) and $10 \times 10$ (right) grids of the Taxi domain.}
%   \label{fig:taxi}

% \end{figure}


\section{Conclusion}

% In this paper we present a novel framework for hierarchical average-reward reinforcement learning which makes it possible to learn the low-level and high-level tasks simultaneously. 
% We propose an eigenvector approach and an online algorithm for solving problems in our framework, and show that the former converges to the optimal value function. As a by-product of our analysis, we also provide a convergence theorem in the non-hierarchical case for average-reward LMDPs, which to the extent of our knowledge, was not previously done. In the future we would like to prove convergence also for the proposed online algorithm.

\chapter{Compositionality with Successor Features via Policy Basis}
\section{Introduction}
% Autonomous agents that interact with an environment usually face tasks  that comprise complex, entangled behaviors over long horizons. Conventional reinforcement learning (RL) methods have successfully addressed this. Notwithstanding, there are cha  However, a common scenario in RL is that where the agent performs several tasks (which are specified by different reward functions) across similar environments, where the dynamics remain the same but the reward function changes. Training a policy for every task separately is time-consuming and requires a lot of data. In such cases, the agent can utilize a method that has built-in generalization capabilities. One such method relies on the assumption that reward function of these tasks can be decomposed into a linear combination of some feature map~\citep{Barreto2017}. When a new task is presented, it is possible to combine previously learned policies and their successor features to solve a new task. %While the combination of such policies is guaranteed to be better than any previously learned policy, it need not be optimal.
% While combining such policies is guaranteed to be an improvement over any previously learned policy, it may not necessarily be optimal. 

Autonomous agents that interact with an environment usually face tasks that comprise complex, entangled behaviors over long horizons. Conventional reinforcement learning (RL) methods have successfully address this. Notwithstanding, there are some limitations that inherent to classical RL method, even when they are endowed with function approximation techniques. One of such challenges is how to combine previously learned behaviors to produce new policies to unseen tasks. As it was introduced in section~\ref{section:successor_features}, the successor features framework is a way to allow generalization of RL method. Nonetheless, the question of which policies should be in the so-called basis or how to learn them was up to this point unclear, leaving the problem of optimal policy transfer open.

Another challenge is that traditional RL methods rely on Markovian reward functions. Section~\ref{section:non_markovian} describes cases where coming up with these reward specification might be difficult or even not possible at all. In such scenarios, there has been a growing interest in alternative methods for non-Markovian task specification in recent years. The work of~\citet{Camacho2019} and~\citet{Icarte2022} successfully purveyed the use of formal languages for such specifications. 

%While these and other conventional RL methods use Markovian reward functions, expressing a task with such reward function can be difficult and may not even be possible in some cases~\citep{Whitehead1995}. In settings where the reward function cannot be expressed in Markovian terms, task specification has raised especial interest in the last few years~\citep{Icarte2022, Camacho2019}.

The method introduced in this chapter aims to build optimal solutions to non-Markovian reward functions exploiting the generalization capabilities of successor feautres. In this context, prior techniques for tackling similar problems have been proposed. As it was explained in section \ref{section:non_markovian}, a fundamental assumption in this setting is there exist a set of propositional symbols that effectively allows the definition of high-level tasks using logic \citep{ToroIcarte2019, Vaezipoor2021} or finite state automata (FSA) \citep{Icarte2022}. This field is tightly connected to hierarchical RL. This connection arise naturally when we see satisying subtaks in the higher level as solving smaller subproblems.  However, combining optimal solutions for subtasks may potentially result in a suboptimal overall policy, and many of these approaches produce \textit{myopic policies} \citep{Vaezipoor2021} (whcih are equivalent to recursively optimal policies in the context of hierarchical RL).

More broadly, generalization has been introduced by conditioning the policy or the value function on the specification of the whole task~\citep{Schaul2015}. Such approaches were recently also proposed for tasks with non-Markovian reward functions \citep{Vaezipoor2021}. However, the methods that specify the whole task usually rely on a blackbox neural network for planning when determining which subgoal to reach next. This makes it hard to interpret the plan to solve the task and although  they show promising results in practice, %they achieve promising empirical results 
it is unclear whether and when these approaches will generalize to a new task. 

 % In this case, it is also possible to simplify the task description once a subgoal has been achieved to improve the data efficiency and generalization. 

Instead, the approach here described uses task descomposition without sacrifing the global optimality of the solution while at the same times attains preidctable generalization. The propose method is a two-stage algorithm in which (i) a set of \textit{local} (sub)policies are learned for a specific choice of a feature representation and (ii) this set of policies is used in a planning step to retrieve the optimal solution to any problem that can be described with an FSA, without additional learning.

\section{Contributions}

More specifically, this work makes the following contributions

 \begin{itemize}
    \item To use successor features to learn a policy basis that is suitable for planning even in stochastic domains.
    \item To provide a planning framework that uses such policy bases for zero-shot generalization to complex temporal tasks described by an arbitrary FSA.
    \item To prove that if the policies in this basis are optimal, this framework produces a globally optimal solution even in stochastic domains.
    \item To provide empirical results that demonstate the improvement over other baselines.
\end{itemize}


\section{Related Work}
One of the key distinctions of this work compared to prior studies is the optimality of the final solution. As noted by \citep{Dietterich2000}, hierarchical methods usually have the capability to achieve hierarchical, recursive, or global optimality. The challenge that often arises when subtask policies are trained in isolation is that the combination of these locally optimal policies does not lead to a globally optimal policy but a recursively \citep{Dayan1992} or hierarchically optimal policy \citep{Sutton1999, Mann2015, Araki2021}.  To tackle this challenge, our approach relies on acquiring a set of low-level policies for each subtask and employing planning to identify the optimal combination of low-level policies when solving a particular task. By learning the CCS with OLS \citep{Roijers2014} in combination with high-level planning our approach ensures that globally optimal policy is found. In this regard, the work of \citep{Alegre2022} is of particular interest as it was the first work that used OLS and successor features \textit{Barreto2017} for optimal policy transfer learning. However, this method has only applied in a setting with Markovian reward function and has not been used with non-Markovian task specifications or high-level planning. 

On the other hand, many recent approaches proposed to use high-level task specifications in the form of LTL~\citep{Icarte2018b, Kuo2020, Vaezipoor2021, Jothimurugan2021}, or similar formal language specifications~\citep{ToroIcarte2019,Camacho2019, Araki2021, Icarte2022} to learn policies. However, the majority of the methods in this area are designed for single-task solutions, with only several focusing on acquiring a set of policies that is capable of addressing multiple tasks \textit{Icarte2018b, Leon2020, kuo2020encoding, Araki2021, Vaezipoor2021}. But, in contrast to our approach, they do not guarantee optimality of the solution.

From these works, our approach is the most similar to the Logical Options Framework \textit{Araki2021}. The main difference is that LOF trains a single policy for each subgoal, resulting in a set of learned policies that is either smaller than or equal to the set acquired through \textsc{sf-fsa-vi}. While employing one policy per subgoal proves sufficient for obtaining a globally optimal policy through planning in deterministic environments~\citep{Wen2020}, this may not hold true in stochastic environments, as our experiments demonstrate. In such instances, the policies generated by LOF are hierarchically optimal but fall short of global optimality.


Two notable examples from aforementioned works on multi-task learning with formal language specifications are the works of \textit{Icarte2018b} and \textit{Vaezipoor2021}. The former struggles with generalizing to unseen tasks, because it uses LTL progression to determine which subtasks need to be learned to solve given tasks. The Q-functions that are subsequently learned for each LTL subtask will therefore not be useful for a new task if its subtasks were not part of the training set. Such limitation does not apply to the latter as it instead encodes the remaining LTL task specification using a neural network and conditions the policy on this LTL embedding. While this approach may be more adaptable to tasks with numerous propositions or subgoals, it risks generating sub-optimal policies as it relies solely on the neural network to select the next proposition to achieve, without incorporating planning. Additionally, since the planning is implicitly done by the neural network, the policy is less interpretable than when explicit planning is used.

The method we propose can be viewed as a method for composing value functions through successor features, akin to previously proposed approaches for composition of value functions and policies ~\citep{Niekerk2019, Barreto2019, NangueTasse2020, Infante2022}. In the work of ~\citep{Infante2022}, which is the closest to our work, the authors propose to learn a basis of value functions that can be combined to form an optimal policy. However, unlike SF-FSA-VI, their approach only works in a restricted class of linearly-solvable MDPs. Lastly, since our approach uses the values of exit states for planning it is also related to planning with exit profiles \textit{Wen2020}. The CCS that we propose to use as a policy basis in our work can be seen as a collection of policies that are optimal for all possible exit profiles.

\section{Preliminaries}
This method assumes that the low-level is modeled by a family of MDPs (see section~\ref{section:successor_features})
\begin{equation}
  \cM^{\boldsymbol{\phi}}\equiv\{\langle\cS,\cE,\cA,\cR_\w,\mathbb{P}_0, \mathbb{P},\gamma\rangle \lvert \cR_\w = \w^\intercal \boldsymbol{\phi}, \forall\w\in\real^d\},
\end{equation}
where there is a propositonal vocabulary $\cP$ that relates to the feature map $\boldsymbol{\phi}$ and the set of exit states $\cE$. The subsequent sections describe the relationship of the building blocks of the method, namely the policy basis for the low-level, the feature representation and the task specification via Finite State Automata (FSA).

\paragraph{Propositional Logic} Environments are assummed to be endowed with a set of high-level, boolean-valued propositional symbols $\cP$. Wihtout loss of generality, it is assumed that these symbols are observed when the agent transitions into some exit state $s\in\cE$ of the low-level MDP $\cM^{\boldsymbol{\phi}}$, though this may not be the case. Every transition $(s, a, s')\in\cS\times\cA\times\cS$ induces some propositional valuation (assignment of truth values) $2^\cP$. Such a valuation depends solely on the new state $s'$ and occurs under a mapping ${\cO:\cS\rightarrow2^\cP}$ that is known to the agent. This implies that the agent can associate valuations $2^\cP$ to transitions at every moment. Propositional symbols are assumed to be mutually exclusive, and the agent cannot observe two symbols in the same transition. A valuation $\Gamma$ is said to satisfy a propositional symbol $p$, formally $\Gamma\vDash p$, if $p$ is true in $\Gamma$. 

\paragraph{Finite State Automaton} High-level tasks are instructed via finite state automata. These are tuples ${\cF=\langle \cU,u_0,\cT,L,\delta\rangle}$ where $\cU$ is the finite set of states, $u_0\in\cU$ is the initial state, $\cT$ is the set of terminal states with $\cU\cap\cT=\emptyset$, $L:\cU\times(\cU\cup\cT)\rightarrow 2^\cP$ is a labelling function that maps FSA states transitions to truth values for the propositions and $\delta:\cT\rightarrow \{0, 1\}$ is a high-level reward function for terminal states. Each transition among FSA states $(u, u')$ defines a subgoal. The agent has to observe some propositional valuation $L(u, u')$ in order to achieve it and FSA states can only be connected by a subgoal. E.g., in Figure~\ref{fig:office_fsa_disjunction}, the FSA state $u_0$ has two outgoing subgoals: getting mail (labeled as \mail) and getting coffee (labeled as \coffee). Non-existing transitions $(u, u')$ get mapped to $L(u, u')=\bot$. The reward function $\delta$ gives a reward larger than 0 only to terminal states. In other words, such a reward function is $\delta(u)=0\;\forall u\in\cU$ and $\delta(\mathbf{t})=1\;\forall \mathbf{t}\in\cT$. 

\begin{figure}[!hbt]
  \centering
  \begin{subfigure}[h]{0.5\textwidth}
    \centering
    \begin{tikzpicture}[node distance=cm,on grid,every initial by arrow/.style={ultra thick,->, >=stealth}]
    \node[thick,state,initial above] (u_0) at (0,0) {$u_0$};
    \node[thick,state]         (u_1) at (0,-2.8)  {$u_1$};
    \node[thick,state]         (u_2) at (1.5, 0)  {$u_2$};
    \node[thick,state]         (u_3) at (1.5,-1.7)  {$u_3$};
    % \node[circle,draw=black,minimum size=0.26cm,inner sep=0pt,fill=black] (t1) at (2,0)  {};
    % \node[circle,draw=black,minimum size=0.26cm,inner sep=0pt,fill=black] (t2) at (2,-1.8)  {};
    \node[circle,draw=black,minimum size=0.26cm,inner sep=0pt,fill=black] (t3) at (1.5,-2.9)  {};
    \node[text width=1cm] at (1.3,-2.9) {$u_T$};
    \path[thick,->, >=stealth] (u_0) edge node [left] {$A$} (u_1);
    \path[thick,->, >=stealth] (u_1) edge node [left] {$B$} (u_2);
    \path[thick,->, >=stealth] (u_2) edge node [right] {$C$} (u_3);
    \path[thick,->, >=stealth] (u_3) edge node [right] {$H$} (t3);
    
    %\path[ultra thick,->, >=stealth] (q_0) edge [loop left] node {$\tuple{\neg \text{\coffee},0}$} ();
    % \path[ultra thick,->, >=stealth] (q_0) edge node [above] {$\tuple{\text{\decoration},0}$} (t1);
    %\path[ultra thick,->, >=stealth] (q_1) edge [loop left] node {$\tuple{\neg o,0}$} ();
    % \path[ultra thick,->, >=stealth] (q_1) edge node [above] {$\tuple{\text{\decoration},0}$} (t2);
    %\path[ultra thick,->, >=stealth] (u_1) edge node [above]{$o$} (t3);
\end{tikzpicture}
  \end{subfigure}
  \begin{subfigure}[h]{0.5\textwidth}
    \centering
    \begin{tikzpicture}[node distance=cm,on grid,every initial by arrow/.style={ultra thick,->, >=stealth}]
    \node[thick,state,initial above] (u_0) at (0,0) {$u_0$};
    \node[ thick,state]         (u_1) at (-0.7,-1.4)  {$u_1$};
    \node[ thick,state]         (u_2) at (0.7,-1.4)  {$u_2$};
    \node[ thick,state]         (u_3) at (-0.7,-3)  {$u_3$};


    \node[circle,draw=black,minimum size=0.26cm,inner sep=0pt,fill=black] (t3) at (0.9,-3)  {};
    \node[text width=1cm ] at (1.2,-2.7) {$u_T$};

    \path[thick,->, >=stealth] (u_0) edge node [left] {$A$} (u_1);
    \path[thick,->, >=stealth] (u_0) edge node [right] {$B$} (u_2);
    \path[thick,->, >=stealth] (u_1) edge node [left] {$C$} (u_3);
    \path[thick,->, >=stealth] (u_2) edge node [right] {$C$} (u_3);
    \path[thick,->, >=stealth] (u_3) edge node [below] {$H$} (t3);

\end{tikzpicture}
  \end{subfigure}
  \begin{subfigure}[t]{0.5\textwidth}
    \centering
    \begin{tikzpicture}[scale=0.54]

    % Outer box
    \draw[step=0.5cm,lightgray] (0,0) grid (8, 6);
    \draw[ultra thick] (0,0) rectangle (8,6);

    % agent
    \node at (0.3, 2.8) {\miniagent};
    % \draw[ultra thick, ->, >=stealth, draw=blue!70!white] (2.5,1.9) -- (2.5,2.5) -- (1.5,2.5) -- (1.5,3.5) -- (2.5,3.5) -- (2.5,5.5) -- (1.5,5.5) -- (1.5,6.5) -- (2.5,6.5) -- (2.5,7.5) -- (3.5,7.5) -- (3.5,6.8);
    % \draw[ultra thick, ->, >=stealth, draw=blue!70!white] (3.8,6.5) -- (4.3,6.5) -- (4.3,4.8);
    \draw[fill=blue] (7.5,5.5) --  (7.5,6) -- (8,6) -- (8,5.5) -- cycle;
    % Outer box
    \draw[fill=red]  (7.5,0) --  (7.5,0.5) -- (8,0.5) -- (8,0) -- cycle;

    % \node[color=white] at (7.5,5.5) {$\text{G}_1$};
    % \node[color=white] at (7.5,0.5) {$\text{G}_2$};

\end{tikzpicture}

    \label{fig:officetaskcomposite}
  \end{subfigure}
  \caption{Disjunction (a) and composite (b) FSA task specifications for the Office domain.}
 \label{fig:sample_fsas}
\end{figure} 



\begin{figure*}[!tb]
  %\centering
  \centering
  \begin{tikzpicture}[scale=0.6]
    % grid
    \draw[step=1cm,gray] (0,0) grid (10, 10);
    % walls
    \draw[ultra thick, fill=black] (5,10) -- (5, 3);
    \draw[ultra thick, fill=black] (1,5) -- (3, 5);
    \draw[ultra thick, fill=black] (7,5) -- (9, 5);

    % Symbols
    \node at (0.5,0.5) {\small $o^1$};
    \node at (0.5,7.5) {\small $\text{\mail}^2$};
    \node at (2.5,9.5) {\small $\text{\coffee}^1$};
    \node at (6.5,9.5) {\small $\text{\coffee}^2$};
    \node at (5.5,6.5) {\small $o^2$};
    \node at (9.5,0.5) {\small $\text{\mail}^2$};

    % agent
    \node at (3.5,1.5) {\small \agent};
    % Optimal path
    \draw[ultra thick, ->, >=stealth, draw=green!70!red!70] (3.5,1.9) -- (3.5,9.5) -- (2.9,9.5);
    \draw[ultra thick, ->, >=stealth, draw=green!70!red!70] (2.1,9.5) -- (0.5,9.5) -- (0.5,7.8) ;
    \draw[ultra thick, ->, >=stealth, draw=green!70!red!70] (0.5,7.2) -- (0.5,0.8);
    % Suboptimal path
     \draw[ultra thick, ->, >=stealth, draw=red!70!white] (3.8, 1.5) -- (8.5, 1.5) -- (8.5, 0.5) -- (9.1, 0.5);
     \draw[ultra thick, ->, >=stealth, draw=red!70!white] (9.5, 0.8) -- (9.5, 9.5) -- (6.9, 9.5);
     \draw[ultra thick, ->, >=stealth, draw=red!70!white] (6.1, 9.5) -- (5.5, 9.5) -- (5.5, 6.8);
    % \draw[ultra thick, ->, >=stealth, draw=blue!70!white] (2.5,1.9) -- (2.5,2.5) -- (1.5,2.5) -- (1.5,3.5) -- (2.5,3.5) -- (2.5,5.5) -- (1.5,5.5) -- (1.5,6.5) -- (2.5,6.5) -- (2.5,7.5) -- (3.5,7.5) -- (3.5,6.8);
    % \draw[ultra thick, ->, >=stealth, draw=blue!70!white] (3.8,6.5) -- (4.3,6.5) -- (4.3,4.8);

    % Outer box
    \draw[ultra thick] (0,0) rectangle (10,10);

\end{tikzpicture}


  \label{fig:office_domain}
  \caption{Depiction of the Office environment. The propositional symbols are $\cP=\{\text{\coffee}, \text{\mail}, o\}$ while the set of exit states is $\cE=\{\text{\coffee}^1,\text{\coffee}^2, \text{\mail}^1,\text{\mail}^2, o^1, o^2\}$. The red and green paths show a suboptimal and optimal (resp.) trajectories for the task `get coffee and mail in any order, then go to an office location'.}
 \label{fig:domains}
\end{figure*}

\paragraph{Policy basis and convex coverage set} The recent work of~\citep{Alegre2022} solves the optimal policy transfer learning problem. They draw the connection between the SF transfer learning problem and multi-objective RL (MORL). The pivotal fact is that the SF representation in Equation~\eqref{eq:sf} can be interpreted as a multidimensional value function and the construction of the aforementioned set of policies $\Pi$ can be cast as a multi-objective optimization problem.
 
 Consequently, the optimistic linear support (OLS) algorithm is extended with successor features in order to learn a set of policies that constitutes a \textit{convex coverage set} (CCS)~\citep{Roijers2015}. Their main result is the SFOLS algorithm (see Supplementary Material\footnote{Supplementary Material at {\texttt{\url{https://arxiv.org/abs/2403.15301}}}} for a full, technical description) in which a set $\Pi_\text{CCS}$ is built incrementally by adding (new) policies to such a set, until convergence. The set $\Pi_\text{CCS}$ contains all non-dominated policies in terms of their multi-objective value functions, where the dominance relation is defined over scalarized values ${V^\pi_\w = \mathbb{E}_{S_0\sim\mathbb{P}_0}\left[V^\pi_\w(S_0)\right]}$, and is characterized as
\begin{align}
  \Pi_\text{CCS} &= \{\pi\;\lvert\;\exists\w\;\text{s.t.}\;\forall {\boldpsi}^{\pi'},\; \w^\intercal {\boldpsi}^\pi {\;\geq\;} \w^\intercal{\boldpsi}^{\pi'} \} \nonumber\\
  &= \{ \pi \;\lvert\;\exists\w\;\text{s.t.}\;\forall {\pi'},\; {V^\pi_\w}{\;\geq\;} {V^{\pi'}_\w}\}.
\end{align} 
In every iteration $k$, SFOLS proposes a new weight vector $\w^k\in\Delta(d)$ for which an optimal policy (and its corresponding SF representation) is learned and added to $\Pi_\text{CCS}$ since it is sufficient to consider weights in $\Delta(d)$ to learn the full $\Pi_\text{CCS}$. The output of SFOLS is both $\Pi_\text{CCS}$ and the SF representation $\boldpsi^\pi$ for every $\pi\in\Pi_\text{CCS}$.
 
Intuitively, all policies in $\Pi_\text{CCS}$ are optimal in at least one task $\w \in \Delta(d)$.
The set $\Pi_\text{CCS}$ is combined with GPI, see Equation~\eqref{eq:gpi}, and upon convergence, for any (new) given task $\w'\in\real^d$, an optimal policy can be identified~\citep[cf. Theorem 2]{Alegre2022}.


\section{Using Successor Features to Solve non-Markovian Reward Specifications}

We focus on the setting in which a low-level MDP is equipped with a reward structure like in Equation~\eqref{eq:reward_sf}. We let the low-level be represented by a family of MDPs $\cM^{\boldsymbol{\phi}}$, where each weight vector $\w\in\real^d$ specifies a low-level task. The agent receives high-level task specifications in the more flexible form of an FSA which permits the specification of non-Markovian reward structures. The combination of a low-level family of MDPs and a high-level FSA gives rise to a \textit{product MDP} $\cM'=\cF\times\cM^{\boldsymbol{\phi}}$ that satisfies the Markov property, and where the state space is augmented to be $\;\cU\times\cS$.

A product MDP $\cM'$ is a well-defined MDP. The agent now follows a policy $\mu:\cU\times\cS\rightarrow\Delta(\cA)$, that depends both on the FSA state and the underlying MDP state. $\cM'$ can be solved with conventional RL methods such as Q-learning~\citep{Watkins1992} by finding an optimal policy $\mu^*$ that maximizes
\begin{equation*}
Q^\mu(u, s, a) = \EEcp{\sum_{i=t}^\infty \gamma^{i-t}\cR_i}{U_t= u, S_t=s, A_t=a}{\mu}.
\end{equation*}
This is, however, impractical since policies should be retrained every time a new high-level task is specified. Exploiting the problem structure is essential for tractable learning, where components can be reused for new task specifications. The special reward structure of the low-level MDPs and our particular choice of feature vectors, later introduced, allow us to define an algorithm able to achieve a solution by simply planning in the space of augmented exit states $\;\cU\times\cE$. This inherently makes obtaining an optimal policy more efficient than solving the whole product MDP, as we reduce the number of states on which it is necessary to compute the value function.

 When presented with different task specifications (e.g.~Figure~\ref{fig:sample_fsas}), the agent may have to perform the same subtask at different moments of the plan or in different FSAs. We aim to provide agents with a collection of base behaviors that can be combined to retrieve the optimal behavior for the whole task.

In line with the previous reasoning, we introduce a two-step algorithm in which the agent first learns a $\Pi_{\text{CCS}}$ (a set of policies that constitute a CCS) on a well-specified representation of the environment. Then these (sub)policies are used to solve efficiently any FSA task specification on the propositional symbols of the environment. In what follows, we motivate the design of the feature vectors, explain our high-level dynamic programming algorithm and prove that it achieves the optimal solution.

 \paragraph{Feature vectors} For a family of MDPs $\cM^{\boldsymbol{\phi}}$, feature vectors $\boldsymbol\phi(s, a, s')$ are \mbox{$\lvert\cE\rvert$-dimensional}. Each feature component $\boldsymbol\phi_j$ is associated with an exit state $\varepsilon_j\in\cE=\{\varepsilon_1,\ldots,\varepsilon_{\lvert\cE\rvert}\}$. Such vectors are built as follows. At terminal transitions $(s, a, \varepsilon_i)\in T$, $\boldsymbol\phi_{j} = 1$ when $j=i$ and $\boldsymbol\phi_{j}=0$ when $j\neq i$. For non-terminal transitions,  For non-terminal transitions, we just require that $\w^\intercal\boldsymbol\phi(s, a, s')<1$. In the case that ${\boldsymbol\phi(s, a, s')=\textbf{0}\in\real^{\lvert\cE\rvert}}$, the SF representation in Equation~\eqref{eq:sf} of each policy consists of a discounted distribution over the exit states. This indicates how likely it is to reach each exit state following such a policy. Furthermore, we require that $\cE\subset\text{supp}(\mathbb{P}_0)$ so the value functions at exit states are well-defined.

 \paragraph{Example} In the office domain depicted in Figure~\ref{fig:office_domain}, the propositional symbols are $\cP=\{\text{\coffee}, \text{\mail}, \text{o}\}$ while the exit states $\cE=\{\text{\coffee}^1, \text{\coffee}^2,\text{\mail}^1,\text{\mail}^2,o^1,o^2\}$. Consequently, the same propositional symbol is satisfied at different exit locations, this is $\cO(\text{\coffee}^1)\vDash\text{\coffee}$ and $\cO(\text{\coffee}^2)\vDash\text{\coffee}$. In this case, $\boldsymbol\phi(s, a, s')\in\real^6$, is defined as the zero vector in $\real^6$ for every $s'\in\cS\setminus\cE$ and gets the corresponding vector component equal to $1$ when $s'\in\cE$. Figure~\ref{fig:sample_fsas} shows two different FSA task specifications for this domain, note that FSAs use symbols in $\cP$ to define the subgoals. The FSA in Figure~\ref{fig:office_fsa_disjunction} (disjunction) corresponds to `get coffee or mail, and then go to an office' and the one in Figure~\ref{fig:office_fsa_composite} (composite) to `get coffee and mail in any order, then go to an office'. 

\begin{algorithm}[!htb]
  \caption{SF-FSA-VI}
  \textbf{Input:} Low-level MDP $\cM^{\boldsymbol{\phi}}$, task specification $\cF$
  \begin{algorithmic}[1]
    \State Obtain $\Pi_\text{CCS}$ on $\cM^{\boldsymbol{\phi}}$.
    \State Initially  $\w^0(u) = \mathbf{0} \in\real^{\lvert\cE\rvert}\;\;\forall u\in\cU$.
   
    \While{not done}
      \For{$u \in \cU$}
        
        \State Update each $\w^{k+1}_j{(u)}$ with Equation~\eqref{eq:update_rule}.
       
      \EndFor
    \EndWhile
    
    \State \Return $\{\w^*(u)\;\forall u\in\cU\}$
  \end{algorithmic}
  \label{alg:online}
\end{algorithm}

\subsubsection{Algorithm} 

The solution to an FSA task specification implies solving a product MDP $\cM'=\cF\times\cM^{\boldsymbol{\phi}}$. Since we have the CCS, the optimal Q-function can be represented by a weight vector $\w^*$:
\begin{equation}
    Q^*_\w(u,s,a) = \smashoperator{\max_{\pi\in\Pi_\text{CCS}}} \w^*(u)^{\intercal}\boldsymbol{\psi}^\pi(s,a).
    \label{eq:extended_qfunction}
\end{equation}
for all $(u,s,a)\in\cU\times\cS\times\cA$. Here, $\w_j^*(u)$ indicates the optimal value from exit state $\varepsilon_j\in\cE$ for FSA state $u$. Then an optimal policy is defined as
\begin{equation}
    \mu^*_\w(u, s) \in \argmax_{a\in\cA} Q^*_\w(u,s,a)\;\forall(s,u)\in\cU\times\cS.
    \label{eq:hl-policy}    
\end{equation}
Therefore, we observe that finding the optimal weight vectors $\w^*(u)$, ${\forall u\in\cU}$ is enough for retrieving the optimal action value function of the product MDP $\cM'$ and, thus, an optimal policy.
 We can obtain this vector using a dynamic-programming approach similar to value iteration: 
\begin{align}
\w_j^{k+1}(u) =&  \max_a Q^*_\w\bigl(\tau(u,\cO(\varepsilon_j)),a\bigr) \\
              =& 
    \max_{a,\pi} \w^k\bigl(\tau(u,\cO(\varepsilon_j))\bigr)^{\intercal} \boldsymbol{\psi}^\pi (\varepsilon_j,a),  
    \label{eq:update_rule}
\end{align}
where $\tau(u,\cO(\varepsilon))\in\cU$ is the FSA state that results from achieving the valuation $\cO(\varepsilon)$ in $u$. We know that, $\w^k_j(u)=1$ if $\tau(u,\cO(j))=\textbf{t}$, per definition, since the high-level reward function $\delta(\mathbf{t})=1$. 
As a result, we propose SF-FSA-VI (see Algorithm~\ref{alg:online}) to extract an optimal policy for a product MDP. As $k\rightarrow\infty$, SF-FSA-VI converges to the optimal set of weight vectors $\{\w^*(u)\}_{u\in\cU}$ and, hence, to the optimal value function in Equation~\eqref{eq:extended_qfunction}.



\subsubsection{Proof of optimality} We first restate the following theorem from~\citep{Alegre2022}.

\begin{theorem}[Alegre, Bazzan, and Silva, 2022]
Let $\Pi$ be a set of policies such that the set of their expected SFs, $\Psi=\{\boldsymbol{\psi}^\pi\}_{\pi\in\Pi}$, constitutes a CCS. Then, given any weight vector $\w\in\real^d$, the GPI policy $\pi_\w^{GPI}(s) \in \arg \max_{a\in A} \max_{\pi\in\Pi} Q_\w^\pi(s,a)$ is
optimal with respect to ${\w: V_\w^{GPI} = V_\w^*}$.
\end{theorem}

\noindent
Applied to our setting, once the set of policies $\Pi_\text{CCS}$ and associated SFs have been computed, we can define an arbitrary vector $\w$ of rewards on the exit states, and use the CCS to obtain an optimal policy $\mu_\w^*$ and an optimal value function $V_\w^*$ without learning. We can then use composition by setting the reward of the exit states equal to the optimal value.

We aim to show that for each augmented state ${(u,s)\in\cU\times\cS}$, the value function output by our algorithm equals the optimal value of $(u,s)$ in the product MDP $\cM'=\cF\times\cM^{\boldsymbol{\phi}}$, i.e.~that $V_{\w(u)}(s)=V_{\cM'}^*(u,s)$. To do so, it is sufficient to show that the weight vectors $\{\w(u)\}_{u\in\cU}$ are optimal.

 Each element of $\w(u)$ is recursively defined as $\w_j(u)=V_{\w(\tau(u,\cO(\varepsilon_j)))}(\varepsilon_j)$. If all weight vectors are optimal, it holds that $V_{\w(\tau(u,\cO(\varepsilon_j)))}(\varepsilon_j)=V_{\cM'}^*(\w(\tau(u,\cO(\varepsilon_j))),\varepsilon_j)$ for each such exit state. Due to the above theorem, the value function $V_{\w(u)}$ is optimal for $\w(u)$. Due to composition that follows GPE and GPI, this means that the value of each internal state $s$ is optimal, i.e.~that $V_{\w(u)}(s)=V_{\cM'}^*(u,s)$.

It remains to show that the weight vectors $\{\w(u)\}_{u\in\cU}$ returned by the algorithm are indeed optimal. To do so it is sufficient to focus on the set of augmented exit states $\cU\times\cE$. We can state a set of optimality equations on the weight vectors as follows:
\begin{align*}
\w_j^*(u) &= V_{\w^*(\tau(u,\cO(\varepsilon)))}(\varepsilon_j)= \max_a Q^*(\tau(u,\cO(\varepsilon)),\varepsilon_j,a)\\
 &= \max_a\max_\pi {\boldsymbol{\psi}}^\pi(\varepsilon_j,a)^\intercal \w^*(\tau(u,\cO(\varepsilon))),
\end{align*}
where ${\boldsymbol{\psi}}^\pi(\varepsilon_j,a)=\sum_{s'}\mathbb{P}(s'|\varepsilon_j,a)\boldsymbol{\psi}^\pi(\varepsilon_j,a,s')$. Our termination condition implies that all subtasks take at least one time step to complete, and due to the discount factor $\gamma$, we have $\lVert\boldsymbol{\psi}(\varepsilon_j,a)\rVert_1<1$. Hence the update rule in Equation~\eqref{eq:update_rule} is a contraction and converges to the set of optimal weight vectors due to the Contraction Mapping Theorem.

\section{Experiments}
\begin{figure*}[htb]
 \begin{subfigure}[t]{0.5\textwidth}
    \centering
    \includegraphics[scale=0.28]{figures/chapter3/results/delivery_learning.pdf}  
    % \subcaption{}
    % \label{fig:results1}
  \end{subfigure}
  \hfill
   \begin{subfigure}[t]{0.5\textwidth}
    \centering
    \includegraphics[scale=0.28]{figures/chapter3/results/office_learning.pdf}  
    % \subcaption{}
    % \label{fig:results2}
  \end{subfigure}
   \begin{subfigure}[b]{0.5\textwidth}
    \centering
    \includegraphics[scale=0.28]{figures/chapter3/results/readapt_delivery.pdf}  
    % \subcaption{}
    % \label{fig:results3}
  \end{subfigure}
  \hfill
   \begin{subfigure}[b]{0.5\textwidth}
    \centering
    \includegraphics[scale=0.28]{figures/chapter3/results/readapt_office.pdf}  
    % \subcaption{}
    % \label{fig:results4}
  \end{subfigure}
  \caption{Experimental results for learning (Delivery, top-left and Office, bottom-left) and compositionality (Delivery, top-right and Office, bottom-right). Results show the average performance and standard deviation over the three tasks and 5 seeds per task.}
  \label{fig:exp_results}
\end{figure*}
\textsc{sf-fsa-vi} is tested in three complex discrete environments later described. At test time, we change the reward to $-1$ for every timestep and use the cumulative reward as the performance metric. We report two types of results. First, we are interested in observing the performance of the derived optimal policy, in Equation~\eqref{eq:hl-policy}, during the \textbf{learning} phase. For this, we fully retrain the high-level policy (lines 2-6 in Algorithm~\ref{alg:online}) every several interactions with the environment as $\Pi_\text{CCS}$ is being learned. Second, once the base behaviors are learned (this is once a complete $\Pi_\text{CCS}$ has been computed), we measure how many \textbf{planning} iterations SF-FSA-VI needs to converge to an optimal solution for different task specifications. In both cases, we compare against existing baselines.

\subsection{Environments and tasks} 

\paragraph{Office} A simplified version of the original Office domain~\citep{Icarte2018b} is used. In the context of this work, this is a complex environment since there are three propositional symbols $\cP=\{\text{\coffee}, \text{\mail}, o\}$ which can be satisfied at different locations, namely $\cE=\{\text{\coffee}^1, \text{\coffee}^2, \text{\mail}^1, \text{\mail}^1, o^1, o^2\}$. Here, there are no obstacle states and $\boldsymbol{\phi}(s,a,s')=\mathbf{0}\in\real^6$ for non-terminal transitions.

\paragraph{Delivery} Algorithms are tested in the Delivery domain~\citep{Araki2021} as-is (see~\ref{fig:delivery}). In the Delivery domain there is a single low-level state associated with each of the propositional symbols, implying that ${\cE=\cP=\{A,B,C, H\}}$. The feature vectors are consistent with our design choice. For terminal transitions, they correspond to their one-hot encodings of the terminal states. There exist obstacle states (in black) for which, upon entering,  the feature vector is $\boldsymbol{\phi}(s,a,s')=\mathbf{-1000}\in\real^4$. This transforms in a large negative reward when multiplied with a corresponding weight vector $\w\in\real^{4}$. For regular grid cells (in white) $\boldsymbol{\phi}(s,a,s')=\mathbf{0}\in\real^{4}$. 

\begin{figure*}[!tb]
    \centering
    
\begin{tikzpicture}[scale=0.4]
    % grid
    %\draw[step=1cm,gray] (0,0) grid (15, 15);
    % walls
    \fill[black] (0,0) rectangle ++ (3,3); 
    \fill[black] (4,0) rectangle ++ (3,3); 
    \fill[black] (8,0) rectangle ++ (3,3); 
    \fill[black] (12,0) rectangle ++ (3,3); 

    \fill[black] (0,4) rectangle ++ (3,3); 
    \fill[black] (4,4) rectangle ++ (3,3); 
    \fill[black] (8,4) rectangle ++ (3,3); 
    \fill[black] (12,4) rectangle ++ (3,3);

    \fill[black] (0, 4) rectangle ++ (3,3); 
    \fill[black] (4, 4) rectangle ++ (3,3); 
    \fill[black] (8, 4) rectangle ++ (3,3); 
    \fill[black] (12,4) rectangle ++ (3,3);

    \fill[black] (0,8) rectangle ++ (3,3); 
    \fill[black] (4,8) rectangle ++ (3,3); 
    \fill[black] (8,8) rectangle ++ (3,3); 
    \fill[black] (12,8) rectangle ++ (3,3);

    \fill[black] (0,12) rectangle ++ (3,3); 
    \fill[black] (4,12) rectangle ++ (3,3); 
    \fill[black] (8,12) rectangle ++ (3,3); 
    \fill[black] (12,12) rectangle ++ (3,3);

    \fill[red] (1,7) rectangle ++ (1,1);
    \node at (1.5,7.5) {\color{white} A};
    \fill[blue] (3,13) rectangle ++ (1,1);
    \node at (3.5,13.5) {\color{white} C};
    \fill[olive] (11,3) rectangle ++ (1,1);
    \node at (11.5,3.5) {\color{white} B};
    \fill[purple] (7,1) rectangle ++ (1,1);
    \node at (7.5,1.5) {\color{white} H};

    % Symbols

    % agent
    % \node at (3.5,1.5) {\agent};
    % \draw[ultra thick, ->, >=stealth, draw=blue!70!white] (2.5,1.9) -- (2.5,2.5) -- (1.5,2.5) -- (1.5,3.5) -- (2.5,3.5) -- (2.5,5.5) -- (1.5,5.5) -- (1.5,6.5) -- (2.5,6.5) -- (2.5,7.5) -- (3.5,7.5) -- (3.5,6.8);
    % \draw[ultra thick, ->, >=stealth, draw=blue!70!white] (3.8,6.5) -- (4.3,6.5) -- (4.3,4.8);

    % Outer box
    \draw[ thick] (0,0) rectangle (15,15);

\end{tikzpicture}
  \caption{Depiction of the Office (a) and Delivery (b) environments, FSA task specification of the composite task in the Office domain and the FSA task specificiation of the sequential task in the Delivery domain (b). In (a) $\cP=\{\text{\coffee}, \text{\mail}, o\}$ and $\cE=\{\text{\coffee}^1,\text{\coffee}^2, \text{\mail}^1,\text{\mail}^2, o^1, o^2\}$. In (b), $\cE=\cP=\{A, B, C, H\}$.}
 \label{fig:delivery}
\end{figure*}

For each of the environments we define three different tasks: sequential, disjunction and composite (all described in the Supplementary Material). The sequential task is meant to show how our algorithm can indeed be effectively used to plan over long horizons, when the other two tasks show the ability of our method to optimally compose the base (sub)policies in complex settings. In natural language, the tasks in the Delivery domain correspond to: "go to $A$, then $B$, then $C$ and finally $H$"  (sequential), "go to $A$ or $B$, then $C$ and finally $H$" (disjunction) and "go to $A$ and $B$ in any order, then $B$, then $C$ and finally $H$" (composite). The agent has to complete the tasks by avoiding obstacles. The counterpart of these tasks in the Office environment are: "get a coffee, then pick up mail and then go to an office" (sequential), "get a coffee or mail, and then go to an office" (disjunction) and "get a coffee and mail in any order, and then go to an office" (composite). 
Our agent never learns how to solve these tasks, but rather learns the set of (sub)policies that constitutes the CCS. At test time, we provide the agent with the FSA task specification, extract a high-level optimal policy and test its performance on solving the task.

\subsection{Baselines} In the literature, we find the most similar approach to ours in the Logical Options Framework (LOF) ~\citep{Araki2021}. We thus use LOF and flat Q-learning on the product MDP as baselines. LOF trains one option per exit state, which are trained simultaneously using intra-option learning, and then uses a high-level value iteration algorithm to train a meta-policy that decides which option to execute in each of the MDP states. On the other hand, the latter learns the action value function in the flat product MDP, from which it extracts the policy. Under certain conditions, flat Q-learning converges to the optimal value function but, especially for longer tasks, it may take a large number of samples. Additionally, it is trained for a specific task, so it is not able to generalize to other task specifications. For LOF, we followed the implementation details prescribed by the authors. 

\subsection{Results}
\paragraph{A motivating example} As it is shown later in the experimental results, in deterministic environments, it suffices to learn the (sub)policies associated with the extrema weights (i.e. those (sub)policies that reach each of the exit states individually) to find a globally optimal policy via planning. In such cases, it may not be necessary to learn a full CCS. That is why, approaches that use the options framework such as LOF traditionally define one option per subgoal. However, there are scenarios, in which these approaches will not find optimal policy. This is the case for most stochastic environments. For example, consider the very simple domain of Double Slit in and the FSA task specification in Figure~\ref{fig:double_slit}. In this environment, there are two exit states $\cE=\{{\color{blue} \text{blue}}, {\color{red}\text{red}}\}$. The agent starts in the leftmost column and middle row. At every timestep, the agent chooses an action amongst $\{\text{UP}, \text{RIGHT}, \text{DOWN}\}$ and is pushed one column to the right in addition to moving in the chosen direction, except in the last column. If the agent chooses RIGHT, he moves an extra column to the right. At every timestep there is a random wind that can blow the agent away up to three positions in the vertical direction. The FSA task specification represents a task in which the agent is indifferent between achieving either of the goal states. Since the RIGHT action brings the agent closer to both goals, the optimal behavior in this case is to commit to either goal as late as possible. In this setting, methods that use one policy per subgoal, such as LOF, train two policies to reach both goals. This means that the agent has to commit to one of the goals from the very beginning, which hurts the performance as it has to make up for the consequences of the random noise. On the other hand, the CCS used by SF-FSA-VI will contain an additional policy that is indifferent between two goals. This leads to a performance gap as our approach achieves an average accumulated reward of $-19.7\pm3.65$ and LOF $-22.70\pm 5.72$.

\paragraph{Learning} Empirical results for learning are shown in Figure~\ref{fig:exp_results} (top-left and bottom-left). The plots reflect how the different methods (ours, LOF and flat Q-learning) perform at solving an FSA task specification during the learning phase. In the case of SF-FSA-VI and LOF, the learning phase corresponds to obtaining the low level (sub)policies for $\Pi_\text{CCS}$ and the options, while for . Results are averaged over the three tasks (sequential, disjunction and composite) previously described for each environment. Each data point in the plots represent the cumulative reward obtained by a fully retrained policy with the current status of $\Pi_\text{CCS}$ and options. In both environments, SF-FSA-VI is the first to reach optimal performance. There exist, however, some differences between LOF and SF-FSA-VI. LOF trains all options simultaneously with intra-option learning. This means that, every transition $(s_t, a_t, s_{t+1})$ is used to update all options' value functions and policies. The learning of a $\Pi_\text{CCS}$, on the other hand, is done sequentially. A fixed sample budget per (sub)policy is set prior to learning, which can be seen as a hyperparameter. We use a total of $8\cdot 10^3$ samples per (sub)policy in both environments. A experience replay buffer is used to speed up the learning of the policy basis $\Pi_\text{CCS}$. Both options and the SF representation of (sub)policies are learned using Q-learning. Due to the incremental nature, at the beginning of the learning process there might be not enough policies in the basis $\Pi_\text{CCS}$ to construct a feasible solution. This is clearly observed in the Delivery domain (Figure~\ref{fig:exp_results}, top left), where at the early stages of the interaction, SF-FSA-VI achieves very low cumulative reward due to failing at delivering a solution. It is not until when there are enough (sub)policies in the basis that Algorithm~\ref{alg:online} attains a policy that solves the problem, which eventually converges to an optimal policy. Similarly, LOF converges to an optimal policy albeit it takes slightly longer to learn. In the more complex Office environment, results follow the same pattern. However, this environment breaks one of the of LOF requirements for optimality: to have a single exit state associated with each propositional predicate. In this problem, for each predicate there exist two exit states that can satisfy them. This makes LOF prone to converge to suboptimal solutions when SF-FSA-VI attains optimality. This is the case for the composite task, where LOF is short-sighted and returns a longer path (in red, Figure~\ref{fig:office_domain}) in contrast to ours that retrieves the optimal solution (in green, Figure~\ref{fig:office_domain}). This means that SF-FSA-VI is more flexible in the task specification. In this environment, our algorithm also converges faster with a more obvious gap with respect to LOF. In any case, learning (sub)policies or options is faster than learning directly on the flat product MDP, as flat Q-learning takes the longest to converge.

\paragraph{Planning} Figure~\ref{fig:exp_results} top-right and bottom-right show how fast SF-FSA-VI and LOF can plan for an optimal solution. Results are again averaged for the three tasks for each environment. Here, a complete policy basis $\Pi_\text{CCS}$ has been previously computed, as well as the option's optimal policies.  In LOF, the cost of each iteration of value iteration is $\lvert\cU\rvert\times\lvert\cS\rvert\times \lvert\cK\rvert$, where $\cK$ is the set of options, while for the Algorithm~\ref{alg:online} we propose it is $\lvert\cU\rvert\times\lvert\cE\rvert\times\lvert\Pi_\text{CCS}\rvert$. By definition, the number of options is equivalent to the number of exit states $\lvert\cK\rvert=\lvert\cE\rvert$, so a single iteration of SF-FSA-VI is more efficient than LOF whenever $\lvert\Pi_\text{CCS}\rvert \ll \lvert\cS\rvert$. In our experiments, the sizes of the CCS are $15$ and $12$ for the Delivery and Office domains, respectively, while the sizes of the state spaces are of $225$ and $121$. Therefore, since our algorithm needs fewer, shorter iterations during planning, it outperforms LOF in terms of planning speed in both domains when composing the global solution. This can be observed in the plots for both environments. 

% \begin{figure}[!htb]
%   \centering
%   \begin{subfigure}[t]{0.23\textwidth}
%     \centering
%     \input{pictures/double_slit}
%   \end{subfigure}
%   \hfill
%   \begin{subfigure}[t]{0.23\textwidth}
%     \centering
%     \begin{tikzpicture}[node distance=cm,on grid,every initial by arrow/.style={ultra thick,->, >=stealth}]
    \node[thick,state,initial above] (u_0) at (0,0) {$u_0$};
    % \node[circle,draw=black,minimum size=0.26cm,inner sep=0pt,fill=black] (t1) at (2,0)  {};
    % \node[circle,draw=black,minimum size=0.26cm,inner sep=0pt,fill=black] (t2) at (2,-1.8)  {};
    \node[circle,draw=black,minimum size=0.26cm,inner sep=0pt,fill=black] (t1) at (-0.5,-1.5)  {};
    \node[circle,draw=black,minimum size=0.26cm,inner sep=0pt,fill=black] (t2) at (0.5,-1.5)  {};
    \node[text width=1cm ] at (-0.7,-1.5) {$u_1$};
    \node[text width=1cm ] at (1.2,-1.5) {$u_2$};
    \path[thick,->, >=stealth] (u_0) edge node [left] {\color{blue} blue} (t1);
    \path[thick,->, >=stealth] (u_0) edge node [right] {\color{red} red} (t2);
    %\path[ultra thick,->, >=stealth] (q_0) edge [loop left] node {$\tuple{\neg \text{\coffee},0}$} ();
    % \path[ultra thick,->, >=stealth] (q_0) edge node [above] {$\tuple{\text{\decoration},0}$} (t1);
    %\path[ultra thick,->, >=stealth] (q_1) edge [loop left] node {$\tuple{\neg o,0}$} ();
    % \path[ultra thick,->, >=stealth] (q_1) edge node [above] {$\tuple{\text{\decoration},0}$} (t2);
    %\path[ultra thick,->, >=stealth] (u_1) edge node [above]{$o$} (t3);
\end{tikzpicture}
%   \end{subfigure}
%   \caption{Double Slit environment (left) and FSA task specification to reach either goal locations {\color{blue}blue} or {\color{red}red}.}
%  \label{fig:double_slit}
% \end{figure} 

\section{Discussion and Conclusion}

In this work, we address the problem of finding optimal behavior for new non-Markovian goal specifications in known environments. To do so, we introduce a novel approach that uses successor features to learn a policy basis, that can subsequently be used to solve any unseen task specified by an FSA with the set of given predicates $\cP$ by planning. SF-FSA-VI is the first that can provably generalize to such new task specification without sacrificing optimality in both deterministic and stochastic environments.

The experiments show that SF-FSA-VI offers several advantages over previous methods. First, due to the use of SF, it allows for faster composition of the high-level value function since it drastically reduces the number of states to plan on. Secondly, thanks to using a CCS over a set of options SF-FSA-VI achieves optimality even in stochastic environments (as shown in the Double Slit example). Lastly, we do not require that there exists a single exit state per predicate which permits more flexible task specification while at the same time allowing deployment in more complex environments. 

A limitation of our approach could be the need to construct a full CCS if one wants to attain global optimality. While the construction of CCS is not timecomsuming for environments with several exit states presented in our work, the computation cost of finding the full CCS could become too large for environments with many exit states. In such case one could instead learn a partial CCS at the cost of a bounded decrease in performance \textit{Alegre2022} or consider splitting the environment into smaller parts with fewer exit states. While our experiments only considered discrete environments, SF-FSA-VI should also be applicable in continuous environments with minor adjustments. These include: using an contiguous set of states instead of a single exit state and using reward shaping to facilitate learning in sparse reward setting.
% \chapter{Conclusion}
% \input{chapters/conclusion}

\bibliographystyle{plainnat}
\bibliography{bibliography}

**\appendix
\chapter{Proof of Theorem~\ref{theo:almdps}}
\section{Proof Theorem~\ref{theo:almdps}}
\label{proof:theo_almdps}
\subsection{Preliminaries}
We introduce the notation:
\begin{itemize}
    \item $\mathds{1}$ denotes an all-ones vector of length $\lvert\cS\rvert$.
    \item $\indicator{p}$ is the indicator function that takes $1$ when predicate $p$ is true and $0$ otherwise.
\end{itemize}

We assumme an underlying continuing LMDP $\cL=\langle\cS,\cP,\cR\rangle$ where $\cS$ represents the state space, $\cP$ the passive dynamics and $\cR$ the reward function. Similarly to [Section B.1] in~\cite{Wan2021}, we also assume there exists a set-valued process $\{X_t\}$ where $X_t$ is a non-empty subset defined as ${X_t = \{ (s) : s\;\text{component of $v$ was updated at timestep $t$} \}}$.

We recall that the TD updates in the asynchronous case are 
\begin{align}
\widehat{v}_{t+1}(s) &\gets \widehat{v}_t(s) + \alpha_t(s) \delta_t(s) \indicator{s\in X_t},\label{eq:updatev}\\
\widehat{\rho}_{t+1} &\gets \widehat{\rho}_t + \lambda \sum_s \alpha_t(s) \delta_t(s) \indicator{s\in X_t}.\label{eq:updaterho}
\end{align}
The indicator $\indicator{s\in X_t}$ specifies whether the value of state $s$ updates at timestep $t$. The TD error for state $s$ is
\begin{align*}
\delta_t(s) %&= r_t(s) - \widehat{\rho}_t - \frac 1 \eta \log \frac {\widehat{\pi}_t(s_{t+1}|s_t)} {\cP(s_{t+1}|s_t)} + \widehat{v}_t(s_{t+1}) - \widehat{v}_t(s_t)\\
 &= r_t(s) - \widehat{\rho}_t + \frac 1 \eta \log \sum_{s'\in\cS} \cP(s'|s) e^{\eta \widehat{v}_t(s')} - \widehat{v}_t(s).
\end{align*}

We introduce a series of necessary assumptions for convergence. We adapt Assumptions B.1-B.5 in~\cite{Wan2021} to the case of LMDPs. Assumptions~\ref{ass:communicating} and~\ref{ass:uniqueness} are standard in average-reward settings, while Assumption~\ref{ass:stepsize1} is the standard Robbins-Monro conditions for step sizes. Assumptions~\ref{ass:stepsize2} and~\ref{ass:stepsize3} are introduced in the convergence argument of RVI Q-learning by~\citet{Borkar1998} and specify some requirements for the learning rates when asynchronous updates are performed. For more details we refer the reader to Section B.1 of~\cite{Wan2021}.
% \begin{assumption}
%  (Communicating assumption) The LMDP $\cL$ has a single communicating class, that is, each state in $\cL$ is accessible from every other state under some policy.
%  \label{ass:communicating}
% \end{assumption}

% NOT NEEDED since we have unichain

\begin{assumption}
 (Value function uniqueness) There exists a unique solution to $v$ in equation~\eqref{eq:boe_almdp} up to a constant shift.
  \label{ass:uniqueness}
\end{assumption}
\begin{assumption} (Stepsize assumption)
    \begin{equation*}
        \alpha_t > 0,\;\sum_{t=0}^{\infty} \alpha_t = \infty,\;\sum_{t=0}^{\infty} \alpha_t^2 < \infty.
    \end{equation*}
      \label{ass:stepsize1}
\end{assumption}
\begin{assumption}
    (Asynchronous Stepsize 1) Let $[\cdot]$ denote the integer part of $(\cdot)$, for $x\in(0, 1)$
    \begin{equation*}
        \sup_i \frac{\alpha_{[xi]}}{\alpha_i} < \infty
    \end{equation*}
    and
    \begin{equation*}
        \frac{\sum_{j=0}^{[yi]}\alpha_j}{\sum_{j=0}^i\alpha_j}\rightarrow 1
    \end{equation*}
    \label{ass:stepsize2}
    uniformly in $y \in[x, 1]$.
\end{assumption}

\begin{assumption}
    (Asynchronous Stepsize 2) There exists $\Delta>0$ such that
    \begin{equation*}
        {\lim\inf}_{t\rightarrow\infty} \frac{\nu(t, s)}{t+1}\geq \Delta
    \end{equation*}
    almost surely, for all $s\in\cS$. Here $\nu(t, s)$ represents the visitation count for state $s$ up to timestep $t$. Furthermore, for all $x > 0$, let
    \begin{equation*}
        N(t, x) = \min \Big\{m > t: \sum_{i=t+1}^m \alpha_i \geq x \Big\}
    \end{equation*}
    \label{ass:stepsize3}
    the limit 
    \begin{equation*}
       \lim_{t\rightarrow\infty} \frac{\sum_{i=\nu(t, s)}^{\nu(N(t, x), s)} \alpha_i}{\sum_{i=\nu(t, s')}^{\nu(N(t, x), s')} \alpha_i}
    \end{equation*}
    exists for all $s, s'\in\cS$.
\end{assumption}

Under the communication assumption, the system
\begin{align}
    v(s) &= \cR(s) - \rho + \frac{1}{\eta} \log \sum {\cP(s'\lvert s) e^{\eta v(s')}},\;\;\forall s\in\cS, \\
    \rho - \hat\rho_0 &= \lambda \Big(\sum_s v(s) - \sum_s\hat v_0(s)\Big),
\end{align}
has a unique solution for $v$ which we denote as $v_\infty$, where $\rho$ is the optimal gain.

At each timestep the increment to $\hat\rho_t$ is $\lambda$ times the increment to $\hat v_t$, and thus, to $\sum_s \hat v_t(s)$. The cumulative increment at $t$ can be expressed as
\begin{align}
     \hat\rho_t - \hat\rho_0 &= \lambda \sum_{i=0}^{t-1}\sum_s \alpha_i(s)\delta_i(s) \indicator{s\in X_t}\nonumber\\
                   &= \lambda \Big(\sum_s\hat v_t(s) - \sum_s\hat v_0 (s)\Big)\nonumber \\
    \implies \hat\rho_t &= \lambda \sum_s \hat v_t(s) - \lambda\sum_s \hat v_0 (s) + \hat\rho_0 \\
    &= \lambda\sum_s \hat v_t(s) - c, \label{eq:cum_rho}\\
    \text{where}\;c &= \lambda \sum_s \hat v_0(s) - \hat\rho_0.
\end{align}

If we replace~\ref{eq:cum_rho} in~\ref{eq:updatev}, we obtain
\begin{equation}
    \widehat{v}_{t+1}(s) \gets \widehat{v}_t(s) + \alpha_t(s) \widetilde\delta_t(s)  \mathbb{I}\{s\in X_t\},~~\forall{s\in\cS},
    \label{eq:td_asynchronous_full}
\end{equation}
where
\begin{equation}
    \widetilde\delta_t(s) = r_t(s) + c - \lambda\sum_s \hat v_t(s) - \frac 1 \eta \log \sum_{s'\in\cS} \cP(s'|s) e^{\eta \widehat{v}_t(s')} - \widehat{v}_t(s).
\end{equation}

This can be interpreted as the TD error of an alternative LMDP $\widetilde\cL=\langle\cS,\cP,\widetilde\cR\rangle$ in which the reward is defined as $\widetilde\cR(s) = \cR(s) + c$ and the gain estimate equals $\lambda \sum_{s\in\cS} \widehat{v}_t(s)$.
%$\lambda \sum_{s'\in\cS} \widehat{v}_t(s)$.
The gain of $\widetilde\cL$ satisfies 
\begin{equation}
    \widetilde\rho = \rho + c.
    \label{eq:alternative_reward}
\end{equation}

The former expression, combined with~\eqref{eq:cum_rho} gives
\begin{equation}
    \widetilde\rho = \lambda \sum_s v_\infty.
    \label{eq:extended_rho}
\end{equation}
It is easy to verify that $v_\infty$ is not only the solution for the original LMDP $\cL$, but also for the alternative LMDP $\widetilde\cL$,
\begin{align*}
     v_\infty(s) &= \cR(s) - \rho + \frac{1}{\eta} \log \sum_{s'} {\cP(s'\lvert s) e^{\eta v_\infty(s')}}\;\;\forall s\in\cS\\
     &= \cR(s) - \widetilde\rho + c + \frac{1}{\eta} \log \sum_{s'} {\cP(s'\lvert s) e^{\eta v_\infty(s')}}\;\;\forall s\in\cS\;(\text{by}~\eqref{eq:alternative_reward})  \\
      &= \widetilde\cR(s) - \widetilde\rho + \frac{1}{\eta} \log \sum_{s'} {\cP(s'\lvert s) e^{\eta v_\infty(s')}}\;\;\forall s\in\cS.
\end{align*}

Now consider $\hat\rho_t$. If we can prove that $\hat v_t\rightarrow v_\infty$ then by~\eqref{eq:cum_rho} we have $\hat\rho_t\rightarrow\lambda\sum v_\infty - c$. By~\eqref{eq:extended_rho}, we know that $\lambda\sum v_\infty = \widetilde\rho$, then we have $\hat\rho_t\rightarrow\widetilde\rho-c$. Using~\eqref{eq:alternative_reward}, we get 
\begin{equation*}
    \hat\rho_t\rightarrow\rho\;\text{almost surely as}\;t\rightarrow\infty.
\end{equation*}

The idea is to prove the convergence of differential soft TD-learning for the alternative LMDP $\widetilde\cL$, which is the same solution as for the original LMDP $\cL$.

We adapt Theorem B.2 in~\cite{Wan2021}.

\begin{theorem} (Convergence of differential TD learning)
    For any $v_0\in\real^{\lvert\cS\rvert}$, let $r_t$, $X_t$, $\alpha_t$ be properly defined and consider the update rule
    \begin{equation}
        \widehat{v}_{t+1}(s) \gets \widehat{v}_t(s) + \alpha_t(s) \big( r_t(s) - \lambda\sum_s \hat v_t(s) - \frac 1 \eta \log \sum_{s'\in\cS} \cP(s'|s) e^{\eta \widehat{v}_t(s')} - \widehat{v}_t(s)\big)  \mathbb{I}\{s\in X_t\},
        \label{eq:td_update_theo}
    \end{equation}
    
    \begin{enumerate}
        \item Assumptions~\ref{ass:unichain} and~\ref{ass:uniqueness}-\ref{ass:stepsize3} hold.
        \item $f:\real^{\lvert\cS\rvert}\rightarrow\real$ is Lipschitz and there exists some $u>0$ such that $\forall c\in\real$ and $x\in\real^{\lvert\cS\rvert}$, $f(\mathds{1})=u$, $f(x + c\mathds{1})=f(x)+cu$ and $f(cx) = c f(x)$
    \end{enumerate}
    then $\hat v_t$ converges almost surely to $v_\infty$.
    \label{theo:our_theorem}
\end{theorem}

We observe that~\eqref{eq:td_update_theo} is in the same form of equationB.24 in~\cite{Wan2021} and equation7.1 in~\cite{Borkar2009}. Thus the results in Section 7.4 in~\cite{Borkar2009} and Theorem 3.2~\cite{Borkar1998} apply to show convergence of~\eqref{eq:td_update_theo}. Due to Assumptions~\ref{ass:stepsize2} and ~\ref{ass:stepsize3}, to show convergence of \eqref{eq:td_asynchronous_full} suffices to show convergence of the following synchronous update
\begin{equation}
    \widehat{v}_{t+1} \gets \widehat{v}_t + \alpha_t \pa{ T(\widehat{v}_t) - f(\widehat{v}_t) \mathds{1} - \widehat{v}_t + M_{t+1}}.
    \label{eq:synchronous}
\end{equation}

Here, $\hat v_t\in\real^{\lvert\cS\rvert}$ is interpreted as a vector, and the operator $T$ is a mapping $T:\real^d\rightarrow\real^d$ defined for each state $s$ as
\begin{equation*}
    T(v)(s) = \cR(s) + \frac 1 \eta \log \sum_{s'\in\cS} \cP(s'|s) e^{ \eta v(s') }.
\end{equation*}
We also define the following operators, 
\begin{align*}
    T^1(v) &= T(v) - \rho\mathds{1} \\
    T^2(v) &= T(v) - f(v)\mathds{1} =  T^1(v) + (\rho - f(v))\mathds{1}.
\end{align*}

The function $f$ is given by $f(v) = \lambda\sum_s v(s)$, which satisfies condition 2 of Theorem~\ref{theo:our_theorem}. The error term $M_{t+1}=r_t-\cR$ is only needed in case the reward vector $r_t$ is sampled from a distribution with mean $\cR$; if reward is deterministic, $M_{t+1}$ can be omitted. If the reward is sampled from a distribution, it should be easy to show that $M_{t+1}$ has zero mean and bounded variance.

The operator $T$ is a non-expansion in the max-norm:
\begin{align*}
T(x)(s) - T(y)(s) &= \frac 1 \eta \log \sum_{s'} \cP(s'|s) e^{\eta x(s')} - \frac 1 \eta \log \sum_{s'} \cP(s'|s) e^{\eta y(s')}\\
 &= \frac 1 \eta \log \frac {\sum_{s'} \cP(s'|s) e^{\eta x(s')}} {\sum_{s'} \cP(s'|s) e^{\eta y(s')}}\\
 &\leq \frac 1 \eta \log \frac {\sum_{s'} \cP(s'|s) e^{\eta \pa{ y(s') + \infnorm{x-y} }}} {\sum_{s'} \cP(s'|s) e^{\eta y(s')}}\\
 &= \frac 1 \eta \log e^{\eta \infnorm{x-y} } \frac {\sum_{s'} \cP(s'|s) e^{\eta y(s')}} {\sum_{s'} \cP(s'|s) e^{\eta y(s')}}\\
 &= \infnorm{x - y}.
\end{align*}
Hence we have
\[
\infnorm{T(x)-T(y)} = \max_s |T(x)(s) - T(y)(s)| \leq \infnorm{x - y}.
\]

We also show the following property of $T$:
\begin{align*}
T(x + c\mathds{1})(s) &= \cR(s) + \frac 1 \eta \log \sum_{s'} \cP(s'|s) e^{\eta \pa{ x(s') + c }}\\
 &= \cR(s) + \frac 1 \eta \log e^{\eta c} \sum_{s'} \cP(s'|s) e^{\eta x(s')}\\
 &= \cR(s) + \frac 1 \eta \sum_{s'} \cP(s'|s) e^{\eta x(s')} + c\\
 &= T(x)(s) + c.
\end{align*}
Hence it follows that $T(x + c\mathds{1}) = T(x) + c\mathds{1}$.

We consider the following ordinary differential equations (ODEs),
\begin{align}
    \dot{y}_t &= T^1(y_t) - y_t \label{eq:ode1} \\
    \dot{x}_t &= T^2(x_t) - x_t \label{eq:ode2}.
\end{align}
Such equations are well-defined since both RHS's are Lipschitz thanks to the properties of $f$ and $T$.

\noindent We complete the proof as a succession of lemmas.

\begin{lemma}
    Let $\bar{y}$ be equilibrium point of the ODE defined in~\eqref{eq:ode1}. Then $\infnorm{y_t - \bar{y}}$ is non-increasing and $y_t\rightarrow y_\infty$ for some equilibrium point of \eqref{eq:ode2}.
\end{lemma}

\begin{proof}
    See Lemma 3.1 in~\cite{Abounadi2001}.
\end{proof} 

\begin{lemma}
    Equation~\eqref{eq:ode2} has a unique equilibrium at $v_\infty$.
\end{lemma}

\begin{proof}
    See Lemma 3.2 in~\cite{Abounadi2001}.
\end{proof}

\begin{lemma}
    Let $x_0=y_0$, then $x_t = y_t + z_t\mathds{1}$ satisfies the ODE $\dot{z}_t = -u z_t + (\rho - f(y_t))$.
\end{lemma}
\begin{proof}
    See Lemma 3.3 in~\cite{Abounadi2001}.
\end{proof} 

\begin{lemma}
    $v_\infty$ is the globally asymptotically stable equilibrium for~\eqref{eq:ode2}
\end{lemma}
\begin{proof}
    See Lemma B.4 in~\cite{Wan2021}.
\end{proof}  

\begin{lemma}
    Equation~\eqref{eq:synchronous} converges almost surely $\hat v_t$ to $v_\infty$ as $t\rightarrow\infty$.
\end{lemma} 
\begin{proof}
    See Lemma B.5 in~\cite{Wan2021} and Lemma 3.8 in~\cite{Abounadi2001}.
\end{proof}

Therefore, stability and convergence of equation~\eqref{eq:td_update_theo} is proved.

\backmatter 
\printindex
% \begin{appendices}
% \section{Proof Theorem~\ref{theo:almdps}}
\label{proof:theo_almdps}
\subsection{Preliminaries}
We introduce the notation:
\begin{itemize}
    \item $\mathds{1}$ denotes an all-ones vector of length $\lvert\cS\rvert$.
    \item $\indicator{p}$ is the indicator function that takes $1$ when predicate $p$ is true and $0$ otherwise.
\end{itemize}

We assumme an underlying continuing LMDP $\cL=\langle\cS,\cP,\cR\rangle$ where $\cS$ represents the state space, $\cP$ the passive dynamics and $\cR$ the reward function. Similarly to [Section B.1] in~\cite{Wan2021}, we also assume there exists a set-valued process $\{X_t\}$ where $X_t$ is a non-empty subset defined as ${X_t = \{ (s) : s\;\text{component of $v$ was updated at timestep $t$} \}}$.

We recall that the TD updates in the asynchronous case are 
\begin{align}
\widehat{v}_{t+1}(s) &\gets \widehat{v}_t(s) + \alpha_t(s) \delta_t(s) \indicator{s\in X_t},\label{eq:updatev}\\
\widehat{\rho}_{t+1} &\gets \widehat{\rho}_t + \lambda \sum_s \alpha_t(s) \delta_t(s) \indicator{s\in X_t}.\label{eq:updaterho}
\end{align}
The indicator $\indicator{s\in X_t}$ specifies whether the value of state $s$ updates at timestep $t$. The TD error for state $s$ is
\begin{align*}
\delta_t(s) %&= r_t(s) - \widehat{\rho}_t - \frac 1 \eta \log \frac {\widehat{\pi}_t(s_{t+1}|s_t)} {\cP(s_{t+1}|s_t)} + \widehat{v}_t(s_{t+1}) - \widehat{v}_t(s_t)\\
 &= r_t(s) - \widehat{\rho}_t + \frac 1 \eta \log \sum_{s'\in\cS} \cP(s'|s) e^{\eta \widehat{v}_t(s')} - \widehat{v}_t(s).
\end{align*}

We introduce a series of necessary assumptions for convergence. We adapt Assumptions B.1-B.5 in~\cite{Wan2021} to the case of LMDPs. Assumptions~\ref{ass:communicating} and~\ref{ass:uniqueness} are standard in average-reward settings, while Assumption~\ref{ass:stepsize1} is the standard Robbins-Monro conditions for step sizes. Assumptions~\ref{ass:stepsize2} and~\ref{ass:stepsize3} are introduced in the convergence argument of RVI Q-learning by~\citet{Borkar1998} and specify some requirements for the learning rates when asynchronous updates are performed. For more details we refer the reader to Section B.1 of~\cite{Wan2021}.
% \begin{assumption}
%  (Communicating assumption) The LMDP $\cL$ has a single communicating class, that is, each state in $\cL$ is accessible from every other state under some policy.
%  \label{ass:communicating}
% \end{assumption}

% NOT NEEDED since we have unichain

\begin{assumption}
 (Value function uniqueness) There exists a unique solution to $v$ in equation~\eqref{eq:boe_almdp} up to a constant shift.
  \label{ass:uniqueness}
\end{assumption}
\begin{assumption} (Stepsize assumption)
    \begin{equation*}
        \alpha_t > 0,\;\sum_{t=0}^{\infty} \alpha_t = \infty,\;\sum_{t=0}^{\infty} \alpha_t^2 < \infty.
    \end{equation*}
      \label{ass:stepsize1}
\end{assumption}
\begin{assumption}
    (Asynchronous Stepsize 1) Let $[\cdot]$ denote the integer part of $(\cdot)$, for $x\in(0, 1)$
    \begin{equation*}
        \sup_i \frac{\alpha_{[xi]}}{\alpha_i} < \infty
    \end{equation*}
    and
    \begin{equation*}
        \frac{\sum_{j=0}^{[yi]}\alpha_j}{\sum_{j=0}^i\alpha_j}\rightarrow 1
    \end{equation*}
    \label{ass:stepsize2}
    uniformly in $y \in[x, 1]$.
\end{assumption}

\begin{assumption}
    (Asynchronous Stepsize 2) There exists $\Delta>0$ such that
    \begin{equation*}
        {\lim\inf}_{t\rightarrow\infty} \frac{\nu(t, s)}{t+1}\geq \Delta
    \end{equation*}
    almost surely, for all $s\in\cS$. Here $\nu(t, s)$ represents the visitation count for state $s$ up to timestep $t$. Furthermore, for all $x > 0$, let
    \begin{equation*}
        N(t, x) = \min \Big\{m > t: \sum_{i=t+1}^m \alpha_i \geq x \Big\}
    \end{equation*}
    \label{ass:stepsize3}
    the limit 
    \begin{equation*}
       \lim_{t\rightarrow\infty} \frac{\sum_{i=\nu(t, s)}^{\nu(N(t, x), s)} \alpha_i}{\sum_{i=\nu(t, s')}^{\nu(N(t, x), s')} \alpha_i}
    \end{equation*}
    exists for all $s, s'\in\cS$.
\end{assumption}

Under the communication assumption, the system
\begin{align}
    v(s) &= \cR(s) - \rho + \frac{1}{\eta} \log \sum {\cP(s'\lvert s) e^{\eta v(s')}},\;\;\forall s\in\cS, \\
    \rho - \hat\rho_0 &= \lambda \Big(\sum_s v(s) - \sum_s\hat v_0(s)\Big),
\end{align}
has a unique solution for $v$ which we denote as $v_\infty$, where $\rho$ is the optimal gain.

At each timestep the increment to $\hat\rho_t$ is $\lambda$ times the increment to $\hat v_t$, and thus, to $\sum_s \hat v_t(s)$. The cumulative increment at $t$ can be expressed as
\begin{align}
     \hat\rho_t - \hat\rho_0 &= \lambda \sum_{i=0}^{t-1}\sum_s \alpha_i(s)\delta_i(s) \indicator{s\in X_t}\nonumber\\
                   &= \lambda \Big(\sum_s\hat v_t(s) - \sum_s\hat v_0 (s)\Big)\nonumber \\
    \implies \hat\rho_t &= \lambda \sum_s \hat v_t(s) - \lambda\sum_s \hat v_0 (s) + \hat\rho_0 \\
    &= \lambda\sum_s \hat v_t(s) - c, \label{eq:cum_rho}\\
    \text{where}\;c &= \lambda \sum_s \hat v_0(s) - \hat\rho_0.
\end{align}

If we replace~\ref{eq:cum_rho} in~\ref{eq:updatev}, we obtain
\begin{equation}
    \widehat{v}_{t+1}(s) \gets \widehat{v}_t(s) + \alpha_t(s) \widetilde\delta_t(s)  \mathbb{I}\{s\in X_t\},~~\forall{s\in\cS},
    \label{eq:td_asynchronous_full}
\end{equation}
where
\begin{equation}
    \widetilde\delta_t(s) = r_t(s) + c - \lambda\sum_s \hat v_t(s) - \frac 1 \eta \log \sum_{s'\in\cS} \cP(s'|s) e^{\eta \widehat{v}_t(s')} - \widehat{v}_t(s).
\end{equation}

This can be interpreted as the TD error of an alternative LMDP $\widetilde\cL=\langle\cS,\cP,\widetilde\cR\rangle$ in which the reward is defined as $\widetilde\cR(s) = \cR(s) + c$ and the gain estimate equals $\lambda \sum_{s\in\cS} \widehat{v}_t(s)$.
%$\lambda \sum_{s'\in\cS} \widehat{v}_t(s)$.
The gain of $\widetilde\cL$ satisfies 
\begin{equation}
    \widetilde\rho = \rho + c.
    \label{eq:alternative_reward}
\end{equation}

The former expression, combined with~\eqref{eq:cum_rho} gives
\begin{equation}
    \widetilde\rho = \lambda \sum_s v_\infty.
    \label{eq:extended_rho}
\end{equation}
It is easy to verify that $v_\infty$ is not only the solution for the original LMDP $\cL$, but also for the alternative LMDP $\widetilde\cL$,
\begin{align*}
     v_\infty(s) &= \cR(s) - \rho + \frac{1}{\eta} \log \sum_{s'} {\cP(s'\lvert s) e^{\eta v_\infty(s')}}\;\;\forall s\in\cS\\
     &= \cR(s) - \widetilde\rho + c + \frac{1}{\eta} \log \sum_{s'} {\cP(s'\lvert s) e^{\eta v_\infty(s')}}\;\;\forall s\in\cS\;(\text{by}~\eqref{eq:alternative_reward})  \\
      &= \widetilde\cR(s) - \widetilde\rho + \frac{1}{\eta} \log \sum_{s'} {\cP(s'\lvert s) e^{\eta v_\infty(s')}}\;\;\forall s\in\cS.
\end{align*}

Now consider $\hat\rho_t$. If we can prove that $\hat v_t\rightarrow v_\infty$ then by~\eqref{eq:cum_rho} we have $\hat\rho_t\rightarrow\lambda\sum v_\infty - c$. By~\eqref{eq:extended_rho}, we know that $\lambda\sum v_\infty = \widetilde\rho$, then we have $\hat\rho_t\rightarrow\widetilde\rho-c$. Using~\eqref{eq:alternative_reward}, we get 
\begin{equation*}
    \hat\rho_t\rightarrow\rho\;\text{almost surely as}\;t\rightarrow\infty.
\end{equation*}

The idea is to prove the convergence of differential soft TD-learning for the alternative LMDP $\widetilde\cL$, which is the same solution as for the original LMDP $\cL$.

We adapt Theorem B.2 in~\cite{Wan2021}.

\begin{theorem} (Convergence of differential TD learning)
    For any $v_0\in\real^{\lvert\cS\rvert}$, let $r_t$, $X_t$, $\alpha_t$ be properly defined and consider the update rule
    \begin{equation}
        \widehat{v}_{t+1}(s) \gets \widehat{v}_t(s) + \alpha_t(s) \big( r_t(s) - \lambda\sum_s \hat v_t(s) - \frac 1 \eta \log \sum_{s'\in\cS} \cP(s'|s) e^{\eta \widehat{v}_t(s')} - \widehat{v}_t(s)\big)  \mathbb{I}\{s\in X_t\},
        \label{eq:td_update_theo}
    \end{equation}
    
    \begin{enumerate}
        \item Assumptions~\ref{ass:unichain} and~\ref{ass:uniqueness}-\ref{ass:stepsize3} hold.
        \item $f:\real^{\lvert\cS\rvert}\rightarrow\real$ is Lipschitz and there exists some $u>0$ such that $\forall c\in\real$ and $x\in\real^{\lvert\cS\rvert}$, $f(\mathds{1})=u$, $f(x + c\mathds{1})=f(x)+cu$ and $f(cx) = c f(x)$
    \end{enumerate}
    then $\hat v_t$ converges almost surely to $v_\infty$.
    \label{theo:our_theorem}
\end{theorem}

We observe that~\eqref{eq:td_update_theo} is in the same form of equationB.24 in~\cite{Wan2021} and equation7.1 in~\cite{Borkar2009}. Thus the results in Section 7.4 in~\cite{Borkar2009} and Theorem 3.2~\cite{Borkar1998} apply to show convergence of~\eqref{eq:td_update_theo}. Due to Assumptions~\ref{ass:stepsize2} and ~\ref{ass:stepsize3}, to show convergence of \eqref{eq:td_asynchronous_full} suffices to show convergence of the following synchronous update
\begin{equation}
    \widehat{v}_{t+1} \gets \widehat{v}_t + \alpha_t \pa{ T(\widehat{v}_t) - f(\widehat{v}_t) \mathds{1} - \widehat{v}_t + M_{t+1}}.
    \label{eq:synchronous}
\end{equation}

Here, $\hat v_t\in\real^{\lvert\cS\rvert}$ is interpreted as a vector, and the operator $T$ is a mapping $T:\real^d\rightarrow\real^d$ defined for each state $s$ as
\begin{equation*}
    T(v)(s) = \cR(s) + \frac 1 \eta \log \sum_{s'\in\cS} \cP(s'|s) e^{ \eta v(s') }.
\end{equation*}
We also define the following operators, 
\begin{align*}
    T^1(v) &= T(v) - \rho\mathds{1} \\
    T^2(v) &= T(v) - f(v)\mathds{1} =  T^1(v) + (\rho - f(v))\mathds{1}.
\end{align*}

The function $f$ is given by $f(v) = \lambda\sum_s v(s)$, which satisfies condition 2 of Theorem~\ref{theo:our_theorem}. The error term $M_{t+1}=r_t-\cR$ is only needed in case the reward vector $r_t$ is sampled from a distribution with mean $\cR$; if reward is deterministic, $M_{t+1}$ can be omitted. If the reward is sampled from a distribution, it should be easy to show that $M_{t+1}$ has zero mean and bounded variance.

The operator $T$ is a non-expansion in the max-norm:
\begin{align*}
T(x)(s) - T(y)(s) &= \frac 1 \eta \log \sum_{s'} \cP(s'|s) e^{\eta x(s')} - \frac 1 \eta \log \sum_{s'} \cP(s'|s) e^{\eta y(s')}\\
 &= \frac 1 \eta \log \frac {\sum_{s'} \cP(s'|s) e^{\eta x(s')}} {\sum_{s'} \cP(s'|s) e^{\eta y(s')}}\\
 &\leq \frac 1 \eta \log \frac {\sum_{s'} \cP(s'|s) e^{\eta \pa{ y(s') + \infnorm{x-y} }}} {\sum_{s'} \cP(s'|s) e^{\eta y(s')}}\\
 &= \frac 1 \eta \log e^{\eta \infnorm{x-y} } \frac {\sum_{s'} \cP(s'|s) e^{\eta y(s')}} {\sum_{s'} \cP(s'|s) e^{\eta y(s')}}\\
 &= \infnorm{x - y}.
\end{align*}
Hence we have
\[
\infnorm{T(x)-T(y)} = \max_s |T(x)(s) - T(y)(s)| \leq \infnorm{x - y}.
\]

We also show the following property of $T$:
\begin{align*}
T(x + c\mathds{1})(s) &= \cR(s) + \frac 1 \eta \log \sum_{s'} \cP(s'|s) e^{\eta \pa{ x(s') + c }}\\
 &= \cR(s) + \frac 1 \eta \log e^{\eta c} \sum_{s'} \cP(s'|s) e^{\eta x(s')}\\
 &= \cR(s) + \frac 1 \eta \sum_{s'} \cP(s'|s) e^{\eta x(s')} + c\\
 &= T(x)(s) + c.
\end{align*}
Hence it follows that $T(x + c\mathds{1}) = T(x) + c\mathds{1}$.

We consider the following ordinary differential equations (ODEs),
\begin{align}
    \dot{y}_t &= T^1(y_t) - y_t \label{eq:ode1} \\
    \dot{x}_t &= T^2(x_t) - x_t \label{eq:ode2}.
\end{align}
Such equations are well-defined since both RHS's are Lipschitz thanks to the properties of $f$ and $T$.

\noindent We complete the proof as a succession of lemmas.

\begin{lemma}
    Let $\bar{y}$ be equilibrium point of the ODE defined in~\eqref{eq:ode1}. Then $\infnorm{y_t - \bar{y}}$ is non-increasing and $y_t\rightarrow y_\infty$ for some equilibrium point of \eqref{eq:ode2}.
\end{lemma}

\begin{proof}
    See Lemma 3.1 in~\cite{Abounadi2001}.
\end{proof} 

\begin{lemma}
    Equation~\eqref{eq:ode2} has a unique equilibrium at $v_\infty$.
\end{lemma}

\begin{proof}
    See Lemma 3.2 in~\cite{Abounadi2001}.
\end{proof}

\begin{lemma}
    Let $x_0=y_0$, then $x_t = y_t + z_t\mathds{1}$ satisfies the ODE $\dot{z}_t = -u z_t + (\rho - f(y_t))$.
\end{lemma}
\begin{proof}
    See Lemma 3.3 in~\cite{Abounadi2001}.
\end{proof} 

\begin{lemma}
    $v_\infty$ is the globally asymptotically stable equilibrium for~\eqref{eq:ode2}
\end{lemma}
\begin{proof}
    See Lemma B.4 in~\cite{Wan2021}.
\end{proof}  

\begin{lemma}
    Equation~\eqref{eq:synchronous} converges almost surely $\hat v_t$ to $v_\infty$ as $t\rightarrow\infty$.
\end{lemma} 
\begin{proof}
    See Lemma B.5 in~\cite{Wan2021} and Lemma 3.8 in~\cite{Abounadi2001}.
\end{proof}

Therefore, stability and convergence of equation~\eqref{eq:td_update_theo} is proved.

% \end{appendices}

\end{document}


%NUMBER OF THE EXTERNAL PAGE EXCEPT IN THE FIRST PAGE OF EACH CAPITAL
\usepackage{fancyhdr}
\pagestyle{fancy}
\fancyfoot{}
\fancyfoot[RO]{\thepage}
\fancyfoot[LE]{\thepage}


%MULTIPLE INDEX
%In the preamble
\usepackage{multind}
\makeindex{authors}
%Introduction to form entries
\index{authors}{Einstein}
%Situation of the Index
\printindex{authors}{Author index}
%The \ usepakage {makeidx} \ makeindex \ printindex commands must be removed
%You need to exacute from the command line makeindex authors